\section*{*Generalization via Right Derived Functors}
\begin{proposition}[Invariant Submodule Functor]
	Let $R \in \mathsf{CRing}$ and $\mathfrak{g} \in {_{R}\mathsf{LieAlg}}$. Then there exists a left exact functor
	\begin{equation*}
		(-)^\mathfrak{g} : {_{\mathfrak{g}}}\mathsf{Mod} \to {_{R}}\mathsf{Mod}.
	\end{equation*}
\end{proposition}

\begin{proof}
	The proof is divided into three steps.
	\begin{enumerate}[label = \textit{Step} \arabic*:, wide = 0pt]
		\item \textit{Definition on objects.} Let $M \in {_{\mathfrak{g}}}\mathsf{Mod}$. Define 
			\begin{equation*}
				M^\mathfrak{g} := \cbr[0]{m \in M : \forall x \in \mathfrak{g}(xm = 0)}.
			\end{equation*}
			Then $M^\mathfrak{g} \leq M$ in $_{R}\mathsf{Mod}$. Indeed, $M^\mathfrak{g} \neq \varnothing$ since $0 \in M^\mathfrak{g}$ and if $m,n \in M^\mathfrak{g}$ and $r \in R$, we have that
			\begin{equation*}
				x(m - n) = xm - xn = 0 \qquad \text{and} \qquad x(rm) = r(xm) = 0,
			\end{equation*}
			\noindent for all $x \in \mathfrak{g}$ by bilinearity of the product. The $R$-submodule $M^\mathfrak{g}$ is called the \bld{invariant submodule of $M$}.
		\item \textit{Definition on morphisms.} Let $f \in {_\mathfrak{g}}\mathsf{Mod}(M,N)$. Then simply let $f^\mathfrak{g} := f\vert_{M^\mathfrak{g}}$. This is well defined. Indeed, if $m \in M^\mathfrak{g}$, then for any $x \in \mathfrak{g}$ we have that
			\begin{equation*}
				xf\vert_{M^\mathfrak{g}}(m) = xf(m) = f(xm) = f(0) = 0.
			\end{equation*}
		\item \textit{Left exactness.} Consider the functor $\varepsilon : {_{R}}\mathsf{Mod} \to {_{\mathfrak{g}}}\mathsf{Mod}$ defined by sending $M$ to the \bld{trivial $\mathfrak{g}$-module}, i.e. $xm := 0$ for all $x \in \mathfrak{g}$ and $m \in M$. Then $\varepsilon \dashv (-)^\mathfrak{g}$. Thus $(-)^\mathfrak{g}$ preserves limits by \cite[159]{leinster:basic_category:2016} and hence since a left exact functor equivalently preserves kernels (see \cite[65]{freyd:abelian_categories:1964}), we have that $(-)^\mathfrak{g}$ is left exact.
	\end{enumerate}
\end{proof}


There is an intrinsic connection between $\mathfrak{g}$-modules and the notion of universal envelopping algebras for Lie algebras. Recall, that for $R \in \mathsf{CRing}$ and $\mathfrak{g} \in {_{R}\mathsf{LieAlg}}$, the \bld{universal envelopping algebra $U\mathfrak{g}$ of $\mathfrak{g}$} is defined to be the quotient of the \bld{tensor algebra $T\mathfrak{g}$}
\begin{equation*}
	T\mathfrak{g} := \bigoplus_{n \in \omega} \mathfrak{g}^{\otimes n}
\end{equation*}
\noindent by the $2$-sided ideal generated by the relations
\begin{equation*}
	\iota[x,y] = \iota(x)\iota(y) - \iota(y)\iota(x),
\end{equation*}
\noindent for all $x,y \in \mathfrak{g}$, where $\iota : \mathfrak{g} \hookrightarrow T\mathfrak{g}$ denotes inclusion. 

\begin{theorem}
	\label{thm:universal_envelopping_algebra}
	Let $R \in \mathsf{CRing}$ and $\mathfrak{g} \in {_{R}\mathsf{LieAlg}}$. Then 
	\begin{equation*}
		_{\mathfrak{g}}\mathsf{Mod} \cong {_{U\mathfrak{g}}}\mathsf{Mod}
	\end{equation*}
	\noindent naturally.
\end{theorem}



\begin{definition}[Injective]
	Let $\mathcal{A}$ be an abelian category. A object $I \in \mathcal{A}$ is said to be \bld{injective}, iff it satisfies the following universal lifting property:
	\begin{equation*}
		\begin{tikzcd}
			0 \arrow[r] & X\arrow[r,"\forall"]\arrow[d,"\forall"'] & Y \arrow[dl,dashed,"\exists"]\\
			& I
		\end{tikzcd}
	\end{equation*}
	We say that \bld{$\mathcal{A}$ has enough injectives}, iff for all $X \in \mathcal{A}$ there exists an exact sequence
	\begin{equation*}
		\begin{tikzcd}
			0 \arrow[r] & X \arrow[r] & I,
		\end{tikzcd}
	\end{equation*}
	\noindent with $I$ injective.
\end{definition}

\begin{definition}[Injective Resolution]
	Let $\mathcal{A}$ be an abelian category and $X \in \mathcal{A}$. An exact sequence
	\begin{equation*}
		\begin{tikzcd}
			0 \arrow[r] & X \arrow[r] & I_0 \arrow[r,"d_0"] & I_1 \arrow[r,"d_1"] & \dots
		\end{tikzcd}
	\end{equation*}
	\noindent with $I_n$ injective for all $n \in \omega$ is called a \bld{injective resolution of $X$}.
\end{definition}

\begin{definition}[Right Derived Functor]
	\label{def:right_derived_functors}
	Let $\mathcal{A}$ and $\mathcal{B}$ be abelian categroies and $F : \mathcal{A} \to \mathcal{B}$ left exact. Moreover, assume that $\mathcal{A}$ has enough injectives. For $n \in \omega$, define the \bld{right derived functors of $F$}, written $(\mathcal{R}^nF)_{n \in \omega}$, by
	\begin{equation*}
		\mathcal{R}^nF(X) := H_n(F(I_0 \to I_1 \to \dots)).
	\end{equation*}
	\noindent where $I_0 \to I_1 \to \dots$ is an injective resolution of $X$.
\end{definition}

\begin{remark}
	One can show that the definition \ref{def:right_derived_functors} of right derived functors does not depend of the choice of an injective resolution (this invokes the so called \emph{comparison theorem} of homological algebra).
\end{remark}

\begin{definition}[Cohomology of Lie Algebras]
	\label{def:cohomology_of_Lie_algebras}
	Let $R \in \mathsf{CRing}$, $\mathfrak{g} \in {_{R}\mathsf{LieAlg}}$ and $M \in {_{\mathfrak{g}}}\mathsf{Mod}$. For $n \in \omega$, define the \bld{$n$-th cohomology group of $\mathfrak{g}$ with coefficients in $M$} by
	\begin{equation*}
		H^n(\mathfrak{g},M) := \mathcal{R}^n(-)^\mathfrak{g}(M) \cong \Ext^n_{U\mathfrak{g}}(R,M) =: \mathcal{R}^n\Hom_{U\mathfrak{g}}(k,-)(M).
	\end{equation*}
\end{definition}

\begin{remark}
	The definition of cohomology of Lie algebras \ref{def:cohomology_of_Lie_algebras} actually makes sense since by theorem \ref{thm:universal_envelopping_algebra}, $_{\mathfrak{g}}\mathsf{Mod}$ has enough injectives and thus by \cite[40]{weibel:homological_algebra:1994}, every object admits an injective resolution. Moreover, the isomorphism in definition \ref{def:cohomology_of_Lie_algebras} follows by
	\begin{equation*}
		M^\mathfrak{g} \cong \Hom_{\mathfrak{g}}(R,M),
	\end{equation*}
	\noindent where we consider $R$ as a trivial $\mathfrak{g}$-module.
\end{remark}

