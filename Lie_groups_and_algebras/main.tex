%%%%%%%%%%%%%%%%%%%%%%%%%%%%%%%%%%%%%%%%%%%%%%%%%%%%%%%%%%%%%%%%%%%%%%%%%%
%Author:																 %
%-------																 %
%Yannis Baehni at University of Zurich									 %
%baehni.yannis@uzh.ch													 %
%																		 %
%Version log:															 %
%------------															 %
%06/02/16 . Basic structure												 %
%04/08/16 . Layout changes including section, contents, abstract.		 %
%%%%%%%%%%%%%%%%%%%%%%%%%%%%%%%%%%%%%%%%%%%%%%%%%%%%%%%%%%%%%%%%%%%%%%%%%%

%Page Setup
\documentclass[
	12pt, 
	oneside, 
	a4paper,
	reqno,
	final
]{amsart}

\usepackage[
	left = 3cm, 
	right = 3cm, 
	top = 3cm, 
	bottom = 3cm
]{geometry}

\newcommand\hmmax{0}
\newcommand\bmmax{0}

%Headers and footers
\usepackage{fancyhdr}
	\pagestyle{fancy}
	%Clear fields
	\fancyhf{}
	%Header right
	\fancyhead[R]{
		\footnotesize
		Yannis B\"{a}hni\\
		\href{mailto:yannis.baehni@uzh.ch}{yannis.baehni@uzh.ch}
	}
	%Header left
	\fancyhead[L]{
		\footnotesize
		MAT579:	Lie groups and Lie algebras\\
		Spring 2018
	}
	%Page numbering in footer
	\fancyfoot[C]{\thepage}
	%Separation line header and footer
	\renewcommand{\headrulewidth}{0.4pt}
	%\renewcommand{\footrulewidth}{0.4pt}
	
	\setlength{\headheight}{19pt} 

%Title
\usepackage[foot]{amsaddr}
\usepackage{newtxtext}
\usepackage[subscriptcorrection,nofontinfo,mtpcal,mtphrb]{mtpro2}
\usepackage{mathtools}
\usepackage{bm}
\usepackage{xspace}
\makeatletter
\def\@textbottom{\vskip \z@ \@plus 1pt}
\let\@texttop\relax
\usepackage{etoolbox}
\patchcmd{\abstract}{\scshape\abstractname}{\textbf{\abstractname}}{}{}

%Section, subsection and subsubsection font
%------------------------------------------
	\renewcommand{\@secnumfont}{\bfseries}
	\renewcommand\section{\@startsection{section}{1}%
  	\z@{.7\linespacing\@plus\linespacing}{.5\linespacing}%
  	{\normalfont\boldmath\bfseries\centering}}
	\renewcommand\subsection{\@startsection{subsection}{2}%
    	\z@{.5\linespacing\@plus.7\linespacing}{-.5em}%
    	{\normalfont\bfseries}}%
	\renewcommand\subsubsection{\@startsection{subsubsection}{3}%
    	\z@{.5\linespacing\@plus.7\linespacing}{-.5em}%
    	{\normalfont\bfseries}}%
%Formatting title of TOC
\renewcommand{\contentsnamefont}{\bfseries}
%Table of Contents
\setcounter{tocdepth}{3}

% Add bold to \section titles in ToC and remove . after numbers
\renewcommand{\tocsection}[3]{%
	\indentlabel{\@ifnotempty{#2}{\bfseries\ignorespaces#1 #2\quad}}\boldmath\bfseries#3}
\renewcommand{\tocappendix}[3]{%
  	\indentlabel{\@ifnotempty{#2}{\bfseries\ignorespaces#1 #2: }}\bfseries#3}
% Remove . after numbers in \subsection
\renewcommand{\tocsubsection}[3]{%
  \indentlabel{\@ifnotempty{#2}{\ignorespaces#1 #2\quad}}#3}
%\let\tocsubsubsection\tocsubsection% Update for \subsubsection
%...

\newcommand\@dotsep{4.5}
\def\@tocline#1#2#3#4#5#6#7{\relax
  \ifnum #1>\c@tocdepth % then omit
  \else
    \par \addpenalty\@secpenalty\addvspace{#2}%
    \begingroup \hyphenpenalty\@M
    \@ifempty{#4}{%
      \@tempdima\csname r@tocindent\number#1\endcsname\relax
    }{%
      \@tempdima#4\relax
    }%
    \parindent\z@ \leftskip#3\relax \advance\leftskip\@tempdima\relax
    \rightskip\@pnumwidth plus1em \parfillskip-\@pnumwidth
    #5\leavevmode\hskip-\@tempdima{#6}\nobreak
    \leaders\hbox{$\m@th\mkern \@dotsep mu\hbox{.}\mkern \@dotsep mu$}\hfill
    \nobreak
    \hbox to\@pnumwidth{\@tocpagenum{\ifnum#1=1\bfseries\fi#7}}\par% <-- \bfseries for \section page
    \nobreak
    \endgroup
  \fi}
\AtBeginDocument{%
\expandafter\renewcommand\csname r@tocindent0\endcsname{0pt}
}
\def\l@subsection{\@tocline{2}{0pt}{2.5pc}{5pc}{}}
\def\l@subsubsection{\@tocline{2}{0pt}{4.5pc}{5pc}{}}
\makeatother

\advance\footskip0.4cm
\textheight=54pc    %a4paper
\textheight=50.5pc %letterpaper
\advance\textheight-0.4cm
\calclayout
%Font settings
%\usepackage{anyfontsize}
%Footnote settings
\usepackage{footmisc}
%	\renewcommand*{\thefootnote}{\fnsymbol{footnote}}
\usepackage{commath}
%Further math environments
%Further math fonts (loads amsfonts implicitely)
%Redefinition of \text
%\usepackage{amstext}
\usepackage{upref}
%Graphics
%\usepackage{graphicx}
%\usepackage{caption}
%\usepackage{subcaption}
%Frames
\usepackage{mdframed}
\allowdisplaybreaks
%\usepackage{interval}
\newcommand{\toup}{%
  \mathrel{\nonscript\mkern-1.2mu\mkern1.2mu{\uparrow}}%
}
\newcommand{\todown}{%
  \mathrel{\nonscript\mkern-1.2mu\mkern1.2mu{\downarrow}}%
}
\AtBeginDocument{\renewcommand*\d{\mathop{}\!\mathrm{d}}}
\renewcommand{\Re}{\operatorname{Re}}
\renewcommand{\Im}{\operatorname{Im}}
\DeclareMathOperator\Log{Log}
\DeclareMathOperator\Arg{Arg}
\DeclareMathOperator\id{id}
\DeclareMathOperator\sech{sech}
\DeclareMathOperator\Aut{Aut}
\DeclareMathOperator\h{h}
\DeclareMathOperator\sgn{sgn}
\DeclareMathOperator\arctanh{arctanh}
\DeclareMathOperator*\esssup{ess.sup}
\DeclareMathOperator\ob{ob}
\DeclareMathOperator\coker{coker}
\DeclareMathOperator\im{im}
\DeclareMathOperator\Ch{Ch}
\DeclareMathOperator\Ext{Ext}
\DeclareMathOperator\Hom{Hom}
\DeclareMathOperator\tr{tr}
%\usepackage{hhline}
%\usepackage{booktabs} 
%\usepackage{array}
%\usepackage{xfrac} 
%\everymath{\displaystyle}
%Enumerate
\usepackage{tikz-cd}
\usepackage{enumitem} 
%\renewcommand{\labelitemi}{$\bullet$}
%\renewcommand{\labelitemii}{$\ast$}
%\renewcommand{\labelitemiii}{$\cdot$}
%\renewcommand{\labelitemiv}{$\circ$}
%Colors
%\usepackage{color}
%\usepackage[cmtip, all]{xy}
%Main style theorem environment
\newtheoremstyle{main} 		             	 		%Stylename
  	{}	                                     		%Space above
  	{}	                                    		%Space below
  	{\itshape}			                     		%Body font
  	{}        	                             		%Indent
  	{\boldmath\bfseries}   	                         		%Head font
  	{.}            	                        		%Head punctuation
  	{ }           	                         		%Head space 
  	{\thmname{#1}\thmnumber{ #2}\thmnote{ (#3)}}	%Head specification
\theoremstyle{main}
\newtheorem{definition}{Definition}[section]
\newtheorem{proposition}{Proposition}[section]
\newtheorem{corollary}{Corollary}[section]
\newtheorem{theorem}{Theorem}[section]
\newtheorem{lemma}{Lemma}[section]
\newtheoremstyle{nonit} 		             	 		%Stylename
  	{}	                                     		%Space above
  	{}	                                    		%Space below
  	{}			                     		%Body font
  	{}        	                             		%Indent
	{\boldmath\bfseries}   	                         		%Head font
  	{.}            	                        		%Head punctuation
  	{ }           	                         		%Head space 
  	{\thmname{#1}\thmnumber{ #2}\thmnote{ (#3)}}	%Head specification
\theoremstyle{nonit}
\newtheorem{remark}{Remark}[section]
\newtheorem{example}{Example}[section]
\newtheorem{examples}{Examples}[section]
\newtheoremstyle{ex} 		             	 		%Stylename
  	{}	                                     		%Space above
  	{}	                                    		%Space below
  	{\small}			                     		%Body font
  	{}        	                             		%Indent
  	{\bfseries\boldmath}   	                         		%Head font
  	{.}            	                        		%Head punctuation
  	{ }           	                         		%Head space 
  	{\thmname{#1}\thmnumber{ #2}\thmnote{ (#3)}}	%Head specification
\theoremstyle{ex}
\newtheorem{exercise}[theorem]{Exercise}
%German non-ASCII-Characters
%Graphics-Tool
%\usepackage{tikz}
%\usepackage{tikzscale}
%\usepackage{bbm}
%\usepackage{bera}
%Listing-Setup
%Bibliographie
\usepackage[backend=bibtex, style=alphabetic]{biblatex}
%\usepackage[babel, german = swiss]{csquotes}
\bibliography{bibliography}
%PDF-Linking
%\usepackage[hyphens]{url}
\usepackage[bookmarksopen=true,bookmarksnumbered=true]{hyperref}
%\PassOptionsToPackage{hyphens}{url}\usepackage{hyperref}
\urlstyle{rm}
\hypersetup{
  colorlinks   = true, %Colours links instead of ugly boxes
  urlcolor     = blue, %Colour for external hyperlinks
  linkcolor    = blue, %Colour of internal links
  citecolor    = blue %Colour of citations
}
\newcommand{\bld}[1]{\boldmath\textit{\textbf{#1}}\unboldmath}


\setcounter{section}{1}

\title{Lie Algebra Cohomology}
\author{Yannis B\"{a}hni}
\address[Yannis B\"{a}hni]{University of Zurich, R\"{a}mistrasse 71, 8006 Zurich}
\email[Yannis B\"{a}hni]{\href{mailto:yannis.baehni@uzh.ch}{\nolinkurl{yannis.baehni@uzh.ch}}}

\begin{document}

\begin{abstract}
	Aim of this talk is to give a short overview of the \emph{cohomology of Lie algebras with coefficients in modules}. We follow the original construction of Chevalley-Eilenberg via complexes. We then state two results concerning \emph{semisimple} Lie algebras, known as the \emph{first and second Whitehead lemma}, and calculate an example. 
\end{abstract}

\maketitle

\tableofcontents

\section*{Introduction}
The archetypal example of a \emph{cohomology theory} arises in differential topology: The \emph{de Rham cohomology}. Given a smooth manifold $M$, we define $\Omega^n(M) := \Gamma(\Lambda^n T^\vee M)$ for each $n \in \omega$, the space of \emph{smooth differential $n$-forms on $M$}. Moreover, is a sequence of mappings $\del[1]{d^n : \Omega^n(M) \to \Omega^{n + 1}(M)}_{n \in \omega}$, called \emph{exterior differentiation operators}, which roughly speaking generalize the notion of a differential of a function. They do satisfy the relation $d^n \circ d^{n - 1} = 0$ and thus we can define the \emph{$n$-th de Rham cohomology group} to be the quotient space
\begin{equation*}
	H^n_{\mathrm{dR}}(M) := \ker d^n/\im d^{n - 1}.
\end{equation*}

\section*{The Chevalley-Eilenberg Complex}
The definition of the $n$-th de Rham cohomology group $H^n_{\mathrm{dR}}(M)$ can actually be thought of a two-stage process: First we go from $\mathsf{Diff}$ to an intermediate category and then we apply a \emph{homology functor}, which is a purely algebraic construct, to go from this intermediate category to $_{\mathbb{R}}\mathsf{Vect}$. Aim of this section is to give the definition of this intermediate category and then to define such a functor explicitely for the case of Lie algebras.

\begin{definition}[Chain Complex in $_{R}\mathsf{Mod}$]
	Let $R \in \mathsf{Ring}$. A \bld{$\mathbb{Z}$-graded chain complex in $_{R}\mathsf{Mod}$} is defined to be a tuple $(C_\bullet,\partial_\bullet)$, consisting of an infinite sequence
	\begin{equation*}
		\begin{tikzcd}
			\dots \arrow[r] & C_{n + 1} \arrow[r,"\partial_{n + 1}"] & C_n \arrow[r,"\partial_n"] & C_{n - 1} \arrow[r] & \dots
		\end{tikzcd}
	\end{equation*}
	\noindent in $_{R}\mathsf{Mod}$ such that $\partial_n\circ \partial_{n + 1} = 0$ holds for all $n \in \mathbb{Z}$.\\
	Dually, a \bld{$\mathbb{Z}$-graded cochain complex in $_{R}\mathsf{Mod}$} is simply a $\mathbb{Z}$-graded chain complex in $_{R}\mathsf{Mod}^{\mathrm{op}}$, i.e. a tuple $(C^\bullet,d^\bullet)$ consisting of an infinite sequence
	\begin{equation*}
		\begin{tikzcd}
			\dots \arrow[r] & C^{n - 1} \arrow[r,"d^n"] & C^n \arrow[r,"d^{n + 1}"] & C^{n + 1} \arrow[r] & \dots
		\end{tikzcd}
	\end{equation*}
	\noindent such that $d^{n + 1} \circ d^n = 0$ holds for all $n \in \mathbb{Z}$.
\end{definition}

In a more general setting, this translates as follows.

\begin{definition}[Chain Complex]
	Let $\mathcal{A}$ be an abelian category. A \bld{$\mathbb{Z}$-graded chain complex in $\mathcal{A}$} is defined to be a tuple $\del[0]{(C_n)_{n \in \mathbb{Z}},(\partial_n)_{n \in \mathbb{Z}}}$, consisting of a sequence $(C_n)_{n \in \mathbb{Z}}$ of objects in $\mathcal{A}$ and a sequence $(\partial_n)_{n \in \mathbb{Z}}$ of morphisms in $\mathcal{A}$, such that 
	\begin{equation*}
		\partial_n \in \Hom_\mathcal{A}(C_n,C_{n - 1}) \qquad \text{and} \qquad \partial_n \circ \partial_{n + 1} = 0
	\end{equation*}
	\noindent for all $n \in \mathbb{Z}$.\\
	Dually, a \bld{$\mathbb{Z}$-graded cochain complex in $\mathcal{A}$} is a $\mathbb{Z}$-graded chain complex in $\mathcal{A}^\mathrm{op}$. 
\end{definition}

\begin{remark}
	For notational simplicity, we will write $(C_\bullet,\partial_\bullet)$ for a chain complex in $\mathcal{A}$.
\end{remark}

\begin{remark}
	The notions of chain and cochain complexes are actually equivalent. Indeed, changing from one to the other amounts simply to a reversing of the $\mathbb{Z}$-grading.
\end{remark}

\begin{remark}
	For each abelian category $\mathcal{A}$, there is an abelian category $\Ch(\mathcal{A})$ of chain complexes in $\mathcal{A}$ (see \cite[7]{weibel:homological_algebra:1994}).
\end{remark}

\begin{definition}[Non-Negative Chain Complex]
	Let $\mathcal{A}$ be an abelian category. A chain complex $(C_\bullet,\partial_\bullet) \in \Ch(\mathcal{A})$ is said to be \bld{non-negative}, iff $C_n = 0$ for all $n < 0$. We denote by $\Ch_{\geq 0}(\mathcal{A})$ the full subcategory of $\Ch(\mathcal{A})$ of non-negative chain complexes.
\end{definition}

In what follows, suppose that all rings admit an identity element and all modules are unital. Recall, that for $R \in \mathsf{CRing}$ and $\mathfrak{g} \in {_{R}\mathsf{LieAlg}}$, the \bld{universal envelopping algebra $U\mathfrak{g}$ of $\mathfrak{g}$} is defined to be the quotient of the \bld{tensor algebra $T\mathfrak{g}$}
\begin{equation*}
	T\mathfrak{g} := \bigoplus_{n \in \omega} \mathfrak{g}^{\otimes n}
\end{equation*}
\noindent by the $2$-sided ideal generated by the relations
\begin{equation*}
	\iota[x,y] = \iota(x)\iota(y) - \iota(y)\iota(x),
\end{equation*}
\noindent for all $x,y \in \mathfrak{g}$, where $\iota : \mathfrak{g} \hookrightarrow T\mathfrak{g}$ denotes inclusion. Moreover, $U$ is a functor from $_{R}\mathsf{LieAlg}$ to associative $R$-algebras. Hence we can define a $U\mathfrak{g}$-action on $U\mathfrak{g} \otimes_R \Lambda^n \mathfrak{g}$ simply by
\begin{equation*}
	u(v \otimes x_1 \wedge \dots \wedge x_n) := uv \otimes x_1 \wedge \dots \wedge x_n,
\end{equation*}
\noindent for all $n \in \omega$.

\begin{definition}[Chevalley-Eilenberg Complex]
	\label{def:CEc}
	Let $R \in \mathsf{CRing}$ and $\mathfrak{g} \in {_{R}}\mathsf{LieAlg}$ which is free as an $R$-module. Denote by $U\mathfrak{g}$ the universal envelopping algebra of $\mathfrak{g}$. Define a non-negative chain complex $(C_\bullet,\partial_\bullet) \in \Ch_{\geq 0}({_{U\mathfrak{g}}}\mathsf{Mod})$ by
	\begin{equation*}
		C_n := U\mathfrak{g} \otimes_R\Lambda^n \mathfrak{g}
	\end{equation*}
	\noindent for all $n \in \omega$ and
	\begin{align*}
		\partial_n(u \otimes x_1 \wedge \dots \wedge x_n) := \ccases{
			ux_1 & n = 1,\\
			\theta_1 + \theta_2 & n > 1,
		}
	\end{align*}
	\noindent where
	\begin{equation*}
		\theta_1 := \sum_{i = 0}^n (-1)^{i + 1} ux_i \otimes x_1 \wedge \dots \wedge \what{x_i} \wedge \dots \wedge x_n,
	\end{equation*}
	\noindent and
	\begin{equation*}
		\theta_2 := \sum_{1 \leq i < j \leq n} (-1)^{i + j} u \otimes [x_i,x_j] \wedge x_1 \wedge \dots \wedge \what{x_i} \wedge \dots \wedge \what{x_j} \wedge \dots \wedge x_n.
	\end{equation*}
\end{definition}

\begin{remark}
	It is by no means obvious, that $\partial_n \circ \partial_{n + 1} = 0$ holds for the Chevalley-Eilenberg complex \ref{def:CEc}. However, it is a tedious computation, and we will only demonstrate the case $n = 1$. In this case
	\begin{align*}
		(\partial_1 \circ \partial_2)(u \otimes x \wedge y) &= \partial_1(ux \otimes y - uy \otimes x - u \otimes [x,y])\\
		&= u(xy - yx) - u[x,y]\\
		&= 0,
	\end{align*}
	\noindent for all $u \in U\mathfrak{g}$ and $x,y \in \mathfrak{g}$.
\end{remark}

\begin{remark}
	The definition of the boundary map $\partial_n$ in the Chevalley-Eilenberg complex \ref{def:CEc} is not as arbitrary at it might seem at first sight. Given $\alpha \in \Omega^n(M)$ for a smooth manifold $M$, then we have that $d\alpha(X_1,\dots,X_{n + 1})$ is of the same form for any $X_1,\dots,X_{n + 1} \in \mathfrak{X}(M)$. Actually, this formula can be used to give an invariant definition of the exterior derivative $d$ in the de Rham theory (see \cite[370--372]{lee:smooth_manifolds:2013}).
\end{remark}

\section*{Left $\mathfrak{g}$-Modules and the Cohomology of Lie Algebras}

\begin{definition}[Category of Left $\mathfrak{g}$-Modules]
	Let $R \in \mathsf{CRing}$ and $\mathfrak{g} \in {_{R}\mathsf{LieAlg}}$. The \bld{category of left $\mathfrak{g}$-modules}, written $_{\mathfrak{g}}\mathsf{Mod}$, is defined to be the category with objects \bld{left $\mathfrak{g}$-modules}, i.e. modules $M \in {_{R}}\mathsf{Mod}$ equipped with an $R$-bilinear product $\mathfrak{g} \times M \to M$, $(x,m) \mapsto xm$, such that 
	\begin{equation*}
		[x,y]m = x(ym) - y(xm)
	\end{equation*}
	\noindent holds for all $x,y \in \mathfrak{g}$ and $m \in M$, and \bld{left $\mathfrak{g}$-module homomorphisms} as morphisms, i.e. morphisms $f \in \Hom_R(M,N)$ such that 
	\begin{equation*}
		f(xm) = xf(m)
	\end{equation*}
	\noindent holds for all $x \in \mathfrak{g}$ and $m \in M$.
\end{definition}

\begin{examples}
	~
	\begin{enumerate}[label = \textup{(}\alph*\textup{)},wide=0pt]
		\item A \bld{trivial $\mathfrak{g}$-module} is an $R$-module $M$ where $xm := 0$ for all $x \in \mathfrak{g}$ and $m \in M$. 
		\item The Lie bracket makes $\mathfrak{g}$ itself into a left $\mathfrak{g}$-module by the Jacobi identity. This module is usually called the \bld{adjoint representation of $\mathfrak{g}$}.
	\end{enumerate}
\end{examples}

\begin{proposition}
	Let $R \in \mathsf{CRing}$ and $\mathfrak{g} \in {_{R}\mathsf{LieAlg}}$. Then $_{\mathfrak{g}}\mathsf{Mod}$ is an abelian category.
\end{proposition}

\begin{proof}
	See \cite[220]{weibel:homological_algebra:1994}.
\end{proof}

We follow \cite[178]{kashiwara:categories:2006}.

\begin{proposition}
	\label{lem:existence_morphism}
	Let $\mathcal{A}$ be an abelian category and $(C_\bullet,\partial_\bullet) \in \Ch(\mathcal{A})$. Then for every $n \in \mathbb{Z}$, there exists a unique monic 
	\begin{equation*}
		\im \partial_{n + 1} \to \ker \partial_n,
	\end{equation*}
	\noindent where $\im \partial_{n + 1} := \ker(\coker \partial_{n + 1})$.
\end{proposition}

\begin{exercise}
	Prove proposition \ref{lem:existence_morphism}. \emph{Hint:} Use that $\im \partial_{n + 1} \to C_n$ is monic by \cite[199]{maclane:categories:1978}.
\end{exercise}

\begin{definition}[Homology of a Chain Complex]
	Let $\mathcal{A}$ be an abelian category and $(C_\bullet,\partial_\bullet) \in \Ch(\mathcal{A})$. Moreover, let $n \in \mathbb{Z}$ and $\im \partial_{n + 1} \to \ker \partial_n$ be the unique morphism assured by lemma \ref{lem:existence_morphism}. Then we define the \bld{$n$-th homology object}, written $H_n(C_\bullet,\partial_\bullet)$, by
	\begin{equation*}
		H_n(C_\bullet,\partial_\bullet) := \coker(\im \partial_{n + 1} \to \ker \partial_n) \in \ob(\mathcal{A}).
	\end{equation*} 
\end{definition}

Explicitely, in our setting this translates as follows.

\begin{definition}[Homology of a Chain Complex in $_{R}\mathsf{Mod}$] Let $R \in \mathsf{Ring}$ and $n \in \mathbb{Z}$. The \bld{$n$-th homology module} of a chain complex $(C_\bullet,\partial_\bullet)$, written $H_n(C_\bullet,\partial_\bullet)$, is defined to be
	\begin{equation*}
		H_n(C_\bullet,\partial_\bullet) := \ker \partial_n/\im \partial_{n + 1}.
	\end{equation*}
	Dually, the \bld{$n$-th cohomology module} of a cochain complex $(C^\bullet,d^\bullet)$ in $_{R}\mathsf{Mod}$, written $H^n(C^\bullet,d^\bullet)$, is defined to be
	\begin{equation*}
		H^n(C^\bullet,d^\bullet) := \ker d^{n + 1}/\im d^n. 
	\end{equation*}
\end{definition}

Observe, that for each $n \in \omega$, we have that $C_n \in {_{\mathfrak{g}}}\mathsf{Mod}$ via $\iota : \mathfrak{g} \hookrightarrow U\mathfrak{g}$.

\begin{definition}[Cohomology of Lie Algebras]
	Let $R \in \mathsf{CRing}$ and $\mathfrak{g} \in {_{R}}\mathsf{LieAlg}$ which is free as an $R$-module. Moreover, let $M \in {_{\mathfrak{g}}}\mathsf{Mod}$ and $(C_\bullet,\partial_\bullet)$ denote the Chevalley-Eilenberg complex \ref{def:CEc}. For $n \in \omega$, define the \bld{$n$-th cohomology group of $\mathfrak{g}$ with coefficients in $M$}, written $H^n(\mathfrak{g},M)$, to be the $n$-th cohomology module of the cochain complex $\Hom_\mathfrak{g}((C_\bullet,\partial_\bullet),M)$.
\end{definition}

\begin{remark}
	Actually, for $n \in \omega$ we have that 
	\begin{equation*}
		\Hom_\mathfrak{g}(C_n,M) \cong \Hom_R(\Lambda^n \mathfrak{g},M),
	\end{equation*}
	\noindent in $_{R}\mathsf{Mod}$. Indeed, if $n \geq 1$ and $\varphi \in \Hom_\mathfrak{g}(C_n,M)$, define $\wbar{\varphi} \in \Hom_R(\Lambda^n \mathfrak{g},M)$ by 
	\begin{equation*}
		\wbar{\varphi}(x_1 \wedge \dots \wedge x_n) := \varphi(1 \otimes x_1 \wedge \dots \wedge x_n),
	\end{equation*}
	\noindent and conversly, if $\varphi \in \Hom_R(\Lambda^n \mathfrak{g},M)$, define $\wbar{\varphi} \in \Hom_\mathfrak{g}(C_n,M)$ by
	\begin{equation*}
		\wbar{\varphi}(u \otimes x_1 \wedge \dots \wedge x_n) := u \varphi(x_1 \wedge \dots \wedge x_n).
	\end{equation*}
	\noindent This is possible, since every left $\mathfrak{g}$-module is naturally a left $U\mathfrak{g}$-module (see \cite[224--225]{weibel:homological_algebra:1994}). If $n = 0$, we define $\wbar{\varphi} \in \Hom_R(R,M)$ by 
	\begin{equation*}
		\wbar{\varphi}(r) := r\varphi(1),
	\end{equation*}
	\noindent for $\varphi \in \Hom_\mathfrak{g}(U\mathfrak{g} \otimes_R R,M) \cong \Hom_\mathfrak{g}(U\mathfrak{g},M)$ and for $\varphi \in \Hom_R(R,M)$ define similarly $\wbar{\varphi} \in \Hom_\mathfrak{g}(U\mathfrak{g},M)$ by
	\begin{equation*}
		\wbar{\varphi}(u) := u\varphi(1).
	\end{equation*}
	Hence we get an induced morphism 
	\begin{equation*}
		\begin{tikzcd}
			\Hom_\mathfrak{g}(C_{n - 1},M) \arrow[r,"d^n"]\arrow[d,"\wbar{\cdot}"'] & \Hom_\mathfrak{g}(C_n,M)\arrow[d,"\wbar{\cdot}"]\\
			\Hom_R(\Lambda^{n - 1}\mathfrak{g},M) \arrow[r,dashed] & \Hom_R(\Lambda^n\mathfrak{g},M).
		\end{tikzcd}
	\end{equation*}
	Explicitely
	\begin{align*}
		d^nf(x_1,\dots,x_n) =& \sum_{i = 1}^n(-1)^{i + 1}x_i f(x_1,\dots,\what{x_i},\dots,x_n)\\
		&+ \sum_{1 \leq i < j \leq n}(-1)^{i + j} f([x_i,x_j],x_1,\dots,\what{x_i},\dots,\what{x_j},\dots,x_n),
	\end{align*}
	\noindent for $n > 1$ and 
	\begin{equation*}
		d^1f(x_1) = x_1f(1).
	\end{equation*}
\end{remark}

\begin{remark}
	There is a more general approach to the definition of the cohomology of Lie algebras via the notion of \emph{right derived functors} which does not use the intermediate step of the Chevalley-Eilenberg complex. 
\end{remark}

\begin{example}[$H^3(\mathfrak{sl}_2,k)$]
	\label{ex:H^3_sl_2_k}
	Let $k$ be a field with characteristic not equal to two and consider the \emph{special linear Lie algebra over $k$}, i.e. $A \in M_2(k)$ such that $\tr A = 0$. This is a three dimensional Lie algebra with ordered basis 	
	\begin{equation*}
		e_1 := \begin{pmatrix}
			1 & 0\\
			0 & -1
		\end{pmatrix} \qquad
		e_2 := \begin{pmatrix}
			0 & 1\\
			0 & 0
		\end{pmatrix} \qquad
		e_3 := \begin{pmatrix}
			0 & 0\\
			1 & 0
		\end{pmatrix}.
	\end{equation*}
	Hence the crucial portion of the Eilenberg-Chevalley cochain complex is given by
	\begin{equation*}
		\begin{tikzcd}
			\dots \arrow[r] & \Hom_k(\Lambda^2\mathfrak{sl}_2,k) \arrow[r,"d"] & \Hom_k(\Lambda^3\mathfrak{sl}_2,k) \arrow[r] & 0.
		\end{tikzcd}
	\end{equation*}
	We compute
	\begin{align*}
		df(e_1,e_2,e_3) =& e_1f(e_2,e_3) - e_2f(e_1,e_3) + e_3f(e_1,e_2)\\
		&- f([e_1,e_2],e_3) + f([e_1,e_3],e_2) - f([e_2,e_3],e_1)\\
		=& -2f(e_2,e_3) - 2f(e_3,e_2) - f(e_1,e_1)\\
		=& -2f(e_2,e_3) + 2f(e_2,e_3)\\
		=& 0,
	\end{align*}
	\noindent since $k$ is interpreted as a trivial $\mathfrak{sl}_2$-module and by the alternating $k$-multilinear properties of $f$. Hence
	\begin{equation*}
		H^3(\mathfrak{sl}_2,k) \cong \Hom_k(\Lambda^3 \mathfrak{sl}_2,k) \cong k,
	\end{equation*}
	\noindent since $\dim \Hom_k(\Lambda^3\mathfrak{sl}_2,k) = 1$.
\end{example}

\section*{The Whitehead Lemmas}
Let $M \in {_\mathfrak{g}}\mathsf{Mod}$ and denote by $\mathrm{Der}(\mathfrak{g},M)$ the \emph{derivations from $\mathfrak{g}$ to $M$}, i.e. mappings $D \in {_R}\mathsf{Mod}$, such that $D[x,y] = x(Dy) - y(Dx)$ holds for all $x,y \in \mathfrak{g}$. If $m \in M$, define $D_m(x) := xm$ for $x \in \mathfrak{g}$. This is called an \emph{inner derivation}. The $R$-module of all inner derivations from $\mathfrak{g}$ to $M$ is denoted by $\mathrm{Der}_{\mathrm{Inn}}(\mathfrak{g},M)$. Clearly, $\mathrm{Der}_{\mathrm{Inn}}(\mathfrak{g},M)$ is an $R$-submodule of $\mathrm{Der}(\mathfrak{g},M)$. By \cite[230]{weibel:homological_algebra:1994} we have that
\begin{equation*}
	H^1(\mathfrak{g},M) \cong \mathrm{Der}(\mathfrak{g},M)/\mathrm{Der}_{\mathrm{Inn}}(\mathfrak{g},M).
\end{equation*}
Let $M \in {_R}\mathsf{LieAlg}$ be abelian. An \emph{extension of $\mathfrak{g}$ by $M$} is a short exact sequence
\begin{equation*}
	\begin{tikzcd}
		0 \arrow[r] & M \arrow[r] & \mathfrak{e} \arrow[r,"\pi"] & \mathfrak{g} \arrow[r] & 0
	\end{tikzcd}
\end{equation*}
\noindent in $_{R}\mathsf{LieAlg}$. There is a notion of equivalence classes of extensions of $\mathfrak{g}$ by $M$. Denote by $\mathrm{Ext}(\mathfrak{g},M)$ the set of all such equivalence classes. Then there is a one-to-one correspondence
\begin{equation*}
	\begin{tikzcd}
		\mathrm{Ext}(\mathfrak{g},M) \arrow[r,leftrightarrow,"1\colon 1"] & H^2(\mathfrak{g},M)
	\end{tikzcd}
\end{equation*}
\noindent if $M \in {_\mathfrak{g}}\mathsf{Mod}$ (see \cite[235]{weibel:homological_algebra:1994}).\\
However, the situation gets much more easier if we make some assumptions.

\begin{theorem}[Whitehead's First Lemma]
	\label{thm:Wfl}
	Let $k$ be a field of characteristic zero and $\mathfrak{g} \in {_{k}}\mathsf{LieAlg}$ semisimple. Then for any finite-dimensional $M \in {_{\mathfrak{g}}}\mathsf{Mod}$ we have that
	\begin{equation*}
		H^1(\mathfrak{g},M) = 0.
	\end{equation*}
	That is, every derivation from $\mathfrak{g}$ into $M$ is an inner derivation.
\end{theorem}

\begin{theorem}[Whitehead's Second Lemma]
	\label{thm:Wfl}
	Let $k$ be a field of characteristic zero and $\mathfrak{g} \in {_{k}}\mathsf{LieAlg}$ semisimple. Then for any finite-dimensional $M \in {_{\mathfrak{g}}}\mathsf{Mod}$ we have that
	\begin{equation*}
		H^2(\mathfrak{g},M) = 0.
	\end{equation*}
\end{theorem}

\begin{remark}
	There cannot be a third Whitehead lemma, since
	\begin{equation*}
		H^3(\mathfrak{sl}_2,k) \cong k,
	\end{equation*}
	\noindent by exercise \ref{ex:H^3_sl_2_k}.
\end{remark}

There is a quite strong result obtained via the second Whitehead lemma, which gives rise to the \emph{structure theory of Lie algebras}.

\begin{theorem}[Levi's Theorem]
	Let $k$ be a field of characteristic zero and $\mathfrak{g} \in {_k}\mathsf{LieAlg}$ finite dimensional. Then there exists a semisimple Lie subalgebra $\mathcal{L} \subseteq \mathfrak{g}$ (called a \bld{Levi factor}), such that
	\begin{equation*}
		\mathfrak{g} \cong \mathcal{L} \ltimes \mathrm{rad}(\mathfrak{g}).
	\end{equation*}
\end{theorem}

\begin{proof}
	We only provide a sketch. For full details see \cite[247]{weibel:homological_algebra:1994}. We know that $\mathfrak{g}/\mathrm{rad}(\mathfrak{g})$ is semisimple, so it suffices to show that the Lie algebra extension
	\begin{equation*}
		\begin{tikzcd}
			0 \arrow[r] & \mathrm{rad}(\mathfrak{g}) \arrow[r] & \mathfrak{g} \arrow[r] & \mathfrak{g}/\mathrm{rad}(\mathfrak{g}) \arrow[r] & 0
		\end{tikzcd}
	\end{equation*}
	\noindent splits. If $\mathrm{rad}(\mathfrak{g})$ is abelian, we are done by Whitehead's second lemma \ref{thm:Wfl}. If $\mathrm{rad}(\mathfrak{g})$ is not abelian, proceed by induction on the derived length of $\mathrm{rad}(\mathfrak{g})$.
\end{proof}

\printbibliography
\end{document}
