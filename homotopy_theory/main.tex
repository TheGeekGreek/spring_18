%%%%%%%%%%%%%%%%%%%%%%%%%%%%%%%%%%%%%%%%%%%%%%%%%%%%%%%%%%%%%%%%%%%%%%%%%%
%Author:																 %
%-------																 %
%Yannis Baehni at University of Zurich									 %
%baehni.yannis@uzh.ch													 %
%																		 %
%Version log:															 %
%------------															 %
%06/02/16 . Basic structure												 %
%04/08/16 . Layout changes including section, contents, abstract.		 %
%%%%%%%%%%%%%%%%%%%%%%%%%%%%%%%%%%%%%%%%%%%%%%%%%%%%%%%%%%%%%%%%%%%%%%%%%%

%Page Setup
\documentclass[
	12pt, 
	oneside, 
	a4paper,
	reqno,
	final
]{amsart}

\usepackage[
	left = 3cm, 
	right = 3cm, 
	top = 3cm, 
	bottom = 3cm
]{geometry}

\newcommand\hmmax{0}
\newcommand\bmmax{0}

%Headers and footers
\usepackage{fancyhdr}
	\pagestyle{fancy}
	%Clear fields
	\fancyhf{}
	%Header right
	\fancyhead[R]{
		\footnotesize
		Yannis B\"{a}hni\\
		\href{mailto:yannis.baehni@uzh.ch}{yannis.baehni@uzh.ch}
	}
	%Header left
	\fancyhead[L]{
		\footnotesize
		MAT579:	Lie groups and Lie algebras\\
		Spring 2018
	}
	%Page numbering in footer
	\fancyfoot[C]{\thepage}
	%Separation line header and footer
	\renewcommand{\headrulewidth}{0.4pt}
	%\renewcommand{\footrulewidth}{0.4pt}
	
	\setlength{\headheight}{19pt} 

%Title
\usepackage[foot]{amsaddr}
\usepackage{newtxtext}
\usepackage[subscriptcorrection,nofontinfo,mtpcal,mtphrb]{mtpro2}
\usepackage{mathtools}
\usepackage{bm}
\usepackage{xspace}
\makeatletter
\def\@textbottom{\vskip \z@ \@plus 1pt}
\let\@texttop\relax
\usepackage{etoolbox}
\patchcmd{\abstract}{\scshape\abstractname}{\textbf{\abstractname}}{}{}

%Section, subsection and subsubsection font
%------------------------------------------
	\renewcommand{\@secnumfont}{\bfseries}
	\renewcommand\section{\@startsection{section}{1}%
  	\z@{.7\linespacing\@plus\linespacing}{.5\linespacing}%
  	{\normalfont\boldmath\bfseries\centering}}
	\renewcommand\subsection{\@startsection{subsection}{2}%
    	\z@{.5\linespacing\@plus.7\linespacing}{-.5em}%
    	{\normalfont\bfseries}}%
	\renewcommand\subsubsection{\@startsection{subsubsection}{3}%
    	\z@{.5\linespacing\@plus.7\linespacing}{-.5em}%
    	{\normalfont\bfseries}}%
%Formatting title of TOC
\renewcommand{\contentsnamefont}{\bfseries}
%Table of Contents
\setcounter{tocdepth}{3}

% Add bold to \section titles in ToC and remove . after numbers
\renewcommand{\tocsection}[3]{%
	\indentlabel{\@ifnotempty{#2}{\bfseries\ignorespaces#1 #2\quad}}\boldmath\bfseries#3}
\renewcommand{\tocappendix}[3]{%
  	\indentlabel{\@ifnotempty{#2}{\bfseries\ignorespaces#1 #2: }}\bfseries#3}
% Remove . after numbers in \subsection
\renewcommand{\tocsubsection}[3]{%
  \indentlabel{\@ifnotempty{#2}{\ignorespaces#1 #2\quad}}#3}
%\let\tocsubsubsection\tocsubsection% Update for \subsubsection
%...

\newcommand\@dotsep{4.5}
\def\@tocline#1#2#3#4#5#6#7{\relax
  \ifnum #1>\c@tocdepth % then omit
  \else
    \par \addpenalty\@secpenalty\addvspace{#2}%
    \begingroup \hyphenpenalty\@M
    \@ifempty{#4}{%
      \@tempdima\csname r@tocindent\number#1\endcsname\relax
    }{%
      \@tempdima#4\relax
    }%
    \parindent\z@ \leftskip#3\relax \advance\leftskip\@tempdima\relax
    \rightskip\@pnumwidth plus1em \parfillskip-\@pnumwidth
    #5\leavevmode\hskip-\@tempdima{#6}\nobreak
    \leaders\hbox{$\m@th\mkern \@dotsep mu\hbox{.}\mkern \@dotsep mu$}\hfill
    \nobreak
    \hbox to\@pnumwidth{\@tocpagenum{\ifnum#1=1\bfseries\fi#7}}\par% <-- \bfseries for \section page
    \nobreak
    \endgroup
  \fi}
\AtBeginDocument{%
\expandafter\renewcommand\csname r@tocindent0\endcsname{0pt}
}
\def\l@subsection{\@tocline{2}{0pt}{2.5pc}{5pc}{}}
\def\l@subsubsection{\@tocline{2}{0pt}{4.5pc}{5pc}{}}
\makeatother

\advance\footskip0.4cm
\textheight=54pc    %a4paper
\textheight=50.5pc %letterpaper
\advance\textheight-0.4cm
\calclayout
%Font settings
%\usepackage{anyfontsize}
%Footnote settings
\usepackage{footmisc}
%	\renewcommand*{\thefootnote}{\fnsymbol{footnote}}
\usepackage{commath}
%Further math environments
%Further math fonts (loads amsfonts implicitely)
%Redefinition of \text
%\usepackage{amstext}
\usepackage{upref}
%Graphics
%\usepackage{graphicx}
%\usepackage{caption}
%\usepackage{subcaption}
%Frames
\usepackage{mdframed}
\allowdisplaybreaks
%\usepackage{interval}
\newcommand{\toup}{%
  \mathrel{\nonscript\mkern-1.2mu\mkern1.2mu{\uparrow}}%
}
\newcommand{\todown}{%
  \mathrel{\nonscript\mkern-1.2mu\mkern1.2mu{\downarrow}}%
}
\AtBeginDocument{\renewcommand*\d{\mathop{}\!\mathrm{d}}}
\renewcommand{\Re}{\operatorname{Re}}
\renewcommand{\Im}{\operatorname{Im}}
\DeclareMathOperator\Log{Log}
\DeclareMathOperator\Arg{Arg}
\DeclareMathOperator\id{id}
\DeclareMathOperator\sech{sech}
\DeclareMathOperator\Aut{Aut}
\DeclareMathOperator\h{h}
\DeclareMathOperator\sgn{sgn}
\DeclareMathOperator\arctanh{arctanh}
\DeclareMathOperator*\esssup{ess.sup}
\DeclareMathOperator\ob{ob}
\DeclareMathOperator\coker{coker}
\DeclareMathOperator\im{im}
\DeclareMathOperator\Ch{Ch}
\DeclareMathOperator\Ext{Ext}
\DeclareMathOperator\Hom{Hom}
\DeclareMathOperator\tr{tr}
%\usepackage{hhline}
%\usepackage{booktabs} 
%\usepackage{array}
%\usepackage{xfrac} 
%\everymath{\displaystyle}
%Enumerate
\usepackage{tikz-cd}
\usepackage{enumitem} 
%\renewcommand{\labelitemi}{$\bullet$}
%\renewcommand{\labelitemii}{$\ast$}
%\renewcommand{\labelitemiii}{$\cdot$}
%\renewcommand{\labelitemiv}{$\circ$}
%Colors
%\usepackage{color}
%\usepackage[cmtip, all]{xy}
%Main style theorem environment
\newtheoremstyle{main} 		             	 		%Stylename
  	{}	                                     		%Space above
  	{}	                                    		%Space below
  	{\itshape}			                     		%Body font
  	{}        	                             		%Indent
  	{\boldmath\bfseries}   	                         		%Head font
  	{.}            	                        		%Head punctuation
  	{ }           	                         		%Head space 
  	{\thmname{#1}\thmnumber{ #2}\thmnote{ (#3)}}	%Head specification
\theoremstyle{main}
\newtheorem{definition}{Definition}[section]
\newtheorem{proposition}{Proposition}[section]
\newtheorem{corollary}{Corollary}[section]
\newtheorem{theorem}{Theorem}[section]
\newtheorem{lemma}{Lemma}[section]
\newtheoremstyle{nonit} 		             	 		%Stylename
  	{}	                                     		%Space above
  	{}	                                    		%Space below
  	{}			                     		%Body font
  	{}        	                             		%Indent
	{\boldmath\bfseries}   	                         		%Head font
  	{.}            	                        		%Head punctuation
  	{ }           	                         		%Head space 
  	{\thmname{#1}\thmnumber{ #2}\thmnote{ (#3)}}	%Head specification
\theoremstyle{nonit}
\newtheorem{remark}{Remark}[section]
\newtheorem{example}{Example}[section]
\newtheorem{examples}{Examples}[section]
\newtheoremstyle{ex} 		             	 		%Stylename
  	{}	                                     		%Space above
  	{}	                                    		%Space below
  	{\small}			                     		%Body font
  	{}        	                             		%Indent
  	{\bfseries\boldmath}   	                         		%Head font
  	{.}            	                        		%Head punctuation
  	{ }           	                         		%Head space 
  	{\thmname{#1}\thmnumber{ #2}\thmnote{ (#3)}}	%Head specification
\theoremstyle{ex}
\newtheorem{exercise}[theorem]{Exercise}
%German non-ASCII-Characters
%Graphics-Tool
%\usepackage{tikz}
%\usepackage{tikzscale}
%\usepackage{bbm}
%\usepackage{bera}
%Listing-Setup
%Bibliographie
\usepackage[backend=bibtex, style=alphabetic]{biblatex}
%\usepackage[babel, german = swiss]{csquotes}
\bibliography{bibliography}
%PDF-Linking
%\usepackage[hyphens]{url}
\usepackage[bookmarksopen=true,bookmarksnumbered=true]{hyperref}
%\PassOptionsToPackage{hyphens}{url}\usepackage{hyperref}
\urlstyle{rm}
\hypersetup{
  colorlinks   = true, %Colours links instead of ugly boxes
  urlcolor     = blue, %Colour for external hyperlinks
  linkcolor    = blue, %Colour of internal links
  citecolor    = blue %Colour of citations
}
\newcommand{\bld}[1]{\boldmath\textit{\textbf{#1}}\unboldmath}


\title{Whitehead Product}
\author{Yannis B\"{a}hni}
\address[Yannis B\"{a}hni]{University of Zurich, R\"{a}mistrasse 71, 8006 Zurich}
\email[Yannis B\"{a}hni]{\href{mailto:yannis.baehni@uzh.ch}{\nolinkurl{yannis.baehni@uzh.ch}}}

\begin{document}

\begin{abstract}
	Aim of this paper is to give a short overview of the definition and the basic properties of the non-generalized \emph{Whitehead product}.
\end{abstract}

\maketitle

\tableofcontents

\section{Introduction}
In the category of \emph{compactly generated spaces}, suppose $G$ is an $H$-group, i.e. a space satisfying the group axioms up to homotopy, then $[X,G]$ is a group for any space $X$. This group need not be abelian. Thus a natural question is, if $[X,G]$ is \emph{nilpotent}. As the notion of nilpotence is based on the behaviour of \emph{commutators}, it is natural to consider certain related products: First of all the \emph{commutator product} or \emph{Samelson product} defined as follows: If $[\alpha] \in [X,G]$ and $[\beta] \in [Y,G]$, define $\gamma : X \times Y \to G$ by
\begin{equation*}
	\gamma(x,y) := \alpha(x)\beta(y)\del[1]{\alpha(x)}^{-1}\del[1]{\beta(y)}^{-1}.
\end{equation*}
Then $\gamma\vert_{X \vee Y}$ is nullhomotopic and thus yields a map $\gamma : X \wedge Y \to G$, whose homotopy class is defined to be the product of $[\alpha]$ and $[\beta]$. When $X = \mathbb{S}^n$, $Y = \mathbb{S}^m$ and $G = \Omega X$, then $[\mathbb{S}^n,G]$ is identified with $\pi_n(G)$ since the $\pi_1$ action is trivial for $H$-spaces, and the Samelson product
\begin{equation*}
	\pi_n(G) \otimes \pi_m(G) \to \pi_{n + m}(G)
\end{equation*}
\noindent translates to a pairing
\begin{equation*}
	\pi_{n + 1}(X) \otimes \pi_{m + 1}(X) \to \pi_{n + m + 1}(X),
\end{equation*}
\noindent the \emph{Whitehead product}, since $\pi_n(G) \cong \pi_{n + 1}(X)$ (see \cite[456--457]{whitehead:homotopy_theory:1978}).

\section{Definition of the Whitehead Product}
Notice, that for any $(X,x_0),(Y,y_0) \in \mathsf{Top}_*$, their coproduct is given by
\begin{equation*}
	\textstyle X \coprod Y = (X \times \cbr[0]{y_0}) \cup (\cbr[0]{x_0} \times Y) \subseteq X \times Y,
\end{equation*}
\noindent with basepoint $(x_0,y_0)$. 

\begin{lemma}
	\label{lem:attaching_cell_wedge_sum_spheres}
	Let $n,m \in \omega$, $n,m \geq 1$. The space $\mathbb{S}^n \times \mathbb{S}^m$ can be obtained from $\mathbb{S}^n \vee \mathbb{S}^m$ by attaching an $n + m$-cell.
\end{lemma}

\begin{proof}
	Observe, that $\mathbb{D}^{n + m} \cong \mathbb{D}^n \times \mathbb{D}^m$ and hence
	\begin{equation*}
		\mathbb{S}^{n + m - 1} = \partial \mathbb{D}^{n + m} \cong (\partial \mathbb{D}^n \times \mathbb{D}^m) \cup (\mathbb{D}^n \times \partial \mathbb{D}^m) = (\mathbb{S}^{n - 1} \times \mathbb{D}^m) \cup (\mathbb{D}^n \times \mathbb{S}^{m - 1}).
	\end{equation*}
	Let 
	\begin{equation*}
		f_1 : \mathbb{S}^{n - 1} \times \mathbb{D}^m \to (\mathbb{S}^{n - 1} \times \mathbb{D}^m) / (\mathbb{S}^{n - 1} \times \mathbb{S}^{m - 1}) \cong \ast \times \mathbb{S}^m
	\end{equation*}
	\noindent and
	\begin{equation*}
		f_2 : \mathbb{D}^n \times \mathbb{S}^{m - 1} \to (\mathbb{D}^n \times \mathbb{S}^{m - 1}) / (\mathbb{S}^{n - 1} \times \mathbb{S}^{m - 1}) \cong \mathbb{S}^n \times \ast
	\end{equation*}
	\noindent be the quotient maps. An application of the gluing lemma thus yields a map
	\begin{equation*}
		f : \mathbb{S}^{n + m - 1} \to \mathbb{S}^n \vee \mathbb{S}^m.
	\end{equation*}
	Moreover, define 
	\begin{equation*}
		q : \mathbb{D}^n \times \mathbb{D}^m \to \mathbb{D}^n/\mathbb{S}^{n - 1} \times \mathbb{D}^m/\mathbb{S}^{m - 1} \cong \mathbb{S}^n \times \mathbb{S}^m	
	\end{equation*}
	\noindent to be the product of the two quotient maps
	\begin{equation*}
		\mathbb{D}^n \to \mathbb{D}^n/\mathbb{S}^{n - 1} \qquad \text{and} \qquad \mathbb{D}^m \to \mathbb{D}^m/\mathbb{S}^{m - 1}.
	\end{equation*}
	Thus we get a commutative diagram
	\begin{equation*}
		\begin{tikzcd}
			\mathbb{S}^{n + m - 1} \arrow[r,"f"]\arrow[d,hook] & \mathbb{S}^n \vee \mathbb{S}^m \arrow[d,hook]\\
			\mathbb{D}^{n + m} \arrow[r,"q"'] & \mathbb{S}^n \times \mathbb{S}^m
		\end{tikzcd}
	\end{equation*}
	Suppose $(X,g,h)$ is another cocone in $\mathsf{Top}$ for the pushout diagram:
	\begin{equation*}
		\begin{tikzcd}
			\mathbb{S}^{n + m - 1} \arrow[r,"f"]\arrow[d,hook] & \mathbb{S}^n \vee \mathbb{S}^m \arrow[d,hook]\arrow[ddr,bend left,"h"]\\
			\mathbb{D}^{n + m} \arrow[r,"q"']\arrow[drr,bend right,"g"'] & \mathbb{S}^n \times \mathbb{S}^m\\
			& & X.
		\end{tikzcd}
	\end{equation*}
	By \cite[186]{munkres:topology:2000}, $q$ is a quotient map. Moreover, for $(x,y) \in \mathbb{S}^{n - 1} \times \mathbb{S}^{m - 1}$, we have that
	\begin{equation*}
		g(x,y) = (h \circ f)(x,y) = h(\ast,\ast).
	\end{equation*}
	Thus $g$ passes to the quotient by \cite[72]{lee:topological_manifolds:2011} to yield a unique map
	\begin{equation*}
		\wtilde{g} : \mathbb{S}^n \times \mathbb{S}^m \to X,
	\end{equation*}
	\noindent such that $g = \wtilde{g} \circ q$. Finally, it is easy to check that
	\begin{equation*}
		\begin{tikzcd}
			\mathbb{S}^{n + m - 1} \arrow[r,"f"]\arrow[d,hook] & \mathbb{S}^n \vee \mathbb{S}^m \arrow[d,hook]\arrow[ddr,bend left,"h"]\\
			\mathbb{D}^{n + m} \arrow[r,"q"']\arrow[drr,bend right,"g"'] & \mathbb{S}^n \times \mathbb{S}^m\arrow[dr,"\wtilde{g}"]\\
			& & X.
		\end{tikzcd}
	\end{equation*}
	\noindent commutes.
\end{proof}

For $n,m \in \omega$, $n,m \geq 1$, consider the map $f$ from lemma \ref{lem:attaching_cell_wedge_sum_spheres}. Let $(X,p) \in \mathsf{Top}_*$. If $[\alpha] \in \pi_n(X,p)$ and $[\beta] \in \pi_m(X,p)$, we get two pointed maps
\begin{equation*}
	\alpha : \mathbb{S}^n \to X \qquad \text{and} \qquad \beta : \mathbb{S}^m \to X.
\end{equation*}
Forming their wedge $\alpha \vee \beta : \mathbb{S}^n \vee \mathbb{S}^m \to X$, defined by
\begin{align*}
	(\alpha \vee \beta) (x,y) := \ccases{
		\alpha(x) & y = \ast,\\
		\beta(y) & x = \ast,
	}
\end{align*}
\noindent and precomposing with $f$, yields a pointed map
\begin{equation*}
	(\alpha \vee \beta) \circ f : \mathbb{S}^{n + m - 1} \to X.
\end{equation*}
Explicitely, if we consider
\begin{equation*}
	\alpha : (\mathbb{D}^n,\mathbb{S}^{n - 1}) \to (X,p) \qquad \text{and} \qquad \beta : (\mathbb{D}^m,\mathbb{S}^{m - 1}) \to (X,p),
\end{equation*}
\noindent we get that
\begin{align}
	\label{eq:Whitehead_product}
	\del[1]{(\alpha \vee \beta) \circ f}(x,y) = \ccases{
		\alpha(x) & x \in \mathbb{D}^n, y \in \mathbb{S}^{m - 1},\\
		\beta(y) & x \in \mathbb{S}^{n - 1}, y \in \mathbb{D}^m.
	}
\end{align}
Hence if $F : \alpha \simeq_{\mathbb{S}^{n - 1}} \alpha'$ and $F' : \beta \simeq_{\mathbb{S}^{m - 1}} \beta'$, we get that 
\begin{equation*}
	H : \del[0]{(\alpha \vee \beta) \circ f} \simeq_\ast \del[0]{(\alpha' \vee \beta') \circ f},
\end{equation*}
\noindent where $H : \mathbb{S}^{n + m - 1} \times I \to X$ is defined by
\begin{align*}
	H(x,y,t) := \ccases{
		F(x,t) & x \in \mathbb{D}^n, y \in \mathbb{S}^{m - 1},\\
		F'(y,t) & x \in \mathbb{S}^{n - 1}, y \in \mathbb{D}^m.
	}
\end{align*}
Thus we get a well defined map $[-,-] : \pi_n(X) \times \pi_m(X) \to \pi_{n + m - 1}(X)$, defined by
\begin{equation*}
	[\alpha,\beta] := [(\alpha \vee \beta) \circ f].
\end{equation*}

\begin{definition}[Whitehead Product]
	Let $n,m \in \omega$, $n,m \geq 1$, and $(X,p) \in \mathsf{Top}_\ast$. The product
	\begin{equation*}
		[-,-] : \pi_n(X,p) \times \pi_m(X,p) \to \pi_{n + m - 1}(X,p)
	\end{equation*}
	\noindent defined by
	\begin{equation*}
		[\alpha,\beta] := [(\alpha \vee \beta) \circ f], 
	\end{equation*}
	\noindent is called the \bld{Whitehead product} and $[-,-]$ is called the \bld{Whitehead bracket}.
\end{definition}

\section{The Whitehead Product and the Conjugation Action}
In this section, we want to have a closer look at $[-,-] : \pi_1(X) \times \pi_n(X) \to \pi_n(X)$. If $n = 1$, the definition of the Whitehead product in equation (\ref{eq:Whitehead_product}) results in figure \ref{fig:n=1} and using that $\mathbb{S}^1$ is parametrized by $\theta \mapsto e^{i\theta}$, i.e. oriented counter clockwise, we get that
\begin{equation*}
	[\alpha,\beta] = [\alpha][\beta][\alpha]^{-1}[\beta]^{-1},
\end{equation*}
\noindent since any reparametrization of a path is homotopic relative to $\partial I$ to the original path (a reparametrization of a path $f$ in $X$ is just a path $f \circ \varphi$, where $\varphi : I \to I$ is continuous and $\varphi\vert_{\partial I} = \id_{\partial I}$). Thus $[\alpha,\beta]$ coincides with the notation of a commutator in $\pi_1(X)$.\\
Let $n > 1$. In the previous case we implicitely used an explicit parametrization of the unit sphere $\mathbb{S}^1$, which resulted in a geometric notion called \emph{orientation}. This can also be done in higher dimensions, however the geometric picture there is not that clear. We define an \bld{orientation of $\mathbb{D}^n$} simply to be a choice of a generator of $H_n(\mathbb{D}^n,\mathbb{S}^{n - 1}) \cong \mathbb{Z}$ (follows from the long exact sequence axiom).  

\begin{lemma}
	\label{lem:order_reversing}
	Let $n \in \omega$, $n > 1$, $[\alpha] \in \pi_n(X)$ and $h : (\mathbb{D}^n,\mathbb{S}^{n - 1}) \to (\mathbb{D}^n,\mathbb{S}^{n - 1})$ an orientation reversing homeomorphism, i.e. $h$ is a homeomorphism and $H_n(h)\langle e \rangle = -\langle e \rangle$ for a generator $\langle e \rangle$ of $H_n(\mathbb{D}^n,\mathbb{S}^{n - 1})$. Then
	\begin{equation*}
		[\alpha \circ h] = -[\alpha].
	\end{equation*}
\end{lemma}

\begin{proof}
 	Let $\rho : \pi_n(Y,A) \to H_n(Y,A)$ denote the \bld{Hurewicz homomorphism} defined by
	\begin{equation*}
		\rho[f] := H_n(f)\langle e \rangle,
	\end{equation*}
	\noindent for a generator $\langle e \rangle$ of $H_n(\mathbb{D}^n,\mathbb{S}^{n - 1})$ (see \cite[166]{whitehead:homotopy_theory:1978}). Using that 
	\begin{equation*}
		\rho : \pi_n(\mathbb{D}^n,\mathbb{S}^{n - 1}) \to H_n(\mathbb{D}^n,\mathbb{S}^{n - 1})
	\end{equation*}
	\noindent is an isomorphism for $n >1$ (see \cite[168]{whitehead:homotopy_theory:1978}), we compute
	\begin{align*}
		[\alpha \circ h] &= \pi_n(\alpha)[h]\\
		&= \pi_n(\alpha)\rho^{-1}\rho[h]\\ 
		&= \pi_n(\alpha)\rho^{-1}(H_n(h)\langle e \rangle)\\
		&= -\pi_n(\alpha)\rho^{-1}\langle e \rangle\\
		&= -\pi_n(\alpha)[\id_{\mathbb{D}^n}]\\
		&= - [\alpha].
	\end{align*}
\end{proof}

The definition of the Whitehead product in equation (\ref{eq:Whitehead_product}) results in \ref{fig:n>1}. Thus	using lemma \ref{lem:order_reversing} yields
\begin{equation*}
	[\alpha,\beta] = [\alpha \cdot \beta] - [\beta],
\end{equation*}
\noindent where $\alpha \cdot \beta$ denotes the \emph{conjugation action}, i.e. the action of $\pi_1(X)$ on $\pi_n(X)$, since the boundary of the cylinder $I \times \mathbb{D}^n$ is oriented coherently with $1 \times \mathbb{D}^n$ and therefore discoherently with $0 \times \mathbb{D}^n$. Indeed, this can be seen as follows: If $e^n$ and $e^m$ are orientations of $\mathbb{D}^n$ and $\mathbb{D}^m$, respectively, then their \emph{cross product} $e^n \times e^m$ is an orientation of $\mathbb{D}^n \times \mathbb{D}^m$ (see \cite[64]{whitehead:homotopy_theory:1978} and \cite[268--278]{hatcher:algebraic_topology:2001} for a more involved treatment). Moreover, we have that 
\begin{equation*}
	\partial(e^n \times e^m) = \partial e^n \times e^m + (-1)^n e^n \times \partial e^m.
\end{equation*}
Therefore
\begin{equation*}
	\partial(e^1 \times e^n) = 1 \times e^n - 0 \times e^n - e^1 \times \partial e^n.
\end{equation*}
And we conclude by observing that the sum $[\alpha] + [\beta]$ in $\pi_n(X)$ is given by the pointed homotopy class of the composition
\begin{equation*}
	\begin{tikzcd}
		\mathbb{S}^n \arrow[r,"c"] & \mathbb{S}^n \vee \mathbb{S}^n \arrow[r,"\alpha \vee \beta"] & X,
	\end{tikzcd}
\end{equation*}
\noindent where $c : \mathbb{S}^n \to \mathbb{S}^n \vee \mathbb{S}^n$ denotes the mapping which collapses the equatorial $\mathbb{S}^{n - 1}$ in $\mathbb{S}^n$ to a point, depicted in figure \ref{fig:collapsing_map} (see \cite[341]{hatcher:algebraic_topology:2001}).

\begin{figure}[h!tb]
	\begin{subfigure}[b]{0.5\textwidth}
		\centering
		\begin{tikzpicture}[scale = 6]
			\draw [thick] (0,0) -- (1,0);
			\draw [thick] (0,0) -- (0,1);
			\draw [thick] (1,0) -- (1,1);
			\draw [thick] (0,1) -- (1,1);
			\draw [thick,->] (0,0) -- (.5,0);
			\draw [thick,->] (1,0) -- (1,.5);
			\draw [thick,->] (0,1) -- (.5,1);
			\draw [thick,->] (0,0) -- (0,.5);
			\draw (.5,0) node[below] {$\alpha$};
			\draw (1,.5) node[right] {$\beta$};
			\draw (.5,1) node[above] {$\alpha$};
			\draw (0,.5) node[left] {$\beta$};
			\node at (0,0) {$\bullet$};
		\end{tikzpicture}
		\caption{$n = 1$.}
    	\label{fig:n=1}
	\end{subfigure}
	~
	\begin{subfigure}[b]{0.5\textwidth}
		\centering
		\begin{tikzpicture}[scale = 3]
			\draw [thick]
  				(180:7.5mm) coordinate (A)
				-- ++(0,-12.5mm) coordinate [midway] (E) coordinate (B) node [midway,left] {$\alpha$}
				arc (180:360:7.5mm and 2.625mm) coordinate (D) node [midway,below] {$\beta$}
				-- (A -| D) coordinate [midway] (F) coordinate (C) node [midway,right] {$\alpha$} arc (0:180:7.5mm and 2.625mm) node [midway,above] {$\beta$};
  			\draw [thick]
  				(0,0) coordinate (T) circle (7.5mm and 2.625mm);
  			\draw [thick,densely dashed] (D) arc (0:180:7.5mm and 2.625mm);
			\draw [thick,->] (B) -- (E);
			\draw [thick,->] (D) -- (F);
			\path [] (A) -- (C) node [midway] {$\circlearrowleft$};
			\path [] (B) -- (D) node [midway] {$\circlearrowleft$};
			\node at (B) {$\bullet$};
		\end{tikzpicture}
		\caption{$n > 1$.}
		\label{fig:n>1}
	\end{subfigure}
	\caption{Whitehead bracket and the conjugation action.}
	\label{fig:conjugation_action}
\end{figure}

\begin{figure}[h!tb]
	\centering
	\begin{tikzpicture}
	\draw [->,thick] (0,0) -- (1,0) node[midway,above] {$c$};
		\draw[thick] (-3,0) circle (2cm);
		\draw[thick] (-5,0) arc (180:360:2 and 0.6);
  		\draw[dashed,thick] (-1,0) arc (0:180:2 and 0.6);
		\draw[thick] (3,1) circle (1cm);
		\draw[thick] (2,1) arc (180:360:1 and 0.3);
  		\draw[dashed,thick] (4,1) arc (0:180:1 and 0.3);
		\draw[thick] (3,-1) circle (1cm);
		\draw[thick] (2,-1) arc (180:360:1 and 0.3);
  		\draw[dashed,thick] (4,-1) arc (0:180:1 and 0.3);
	\end{tikzpicture}
	\caption{The collapsing map $c : \mathbb{S}^n \to \mathbb{S}^n \vee \mathbb{S}^n$.}
	\label{fig:collapsing_map}
\end{figure}

\section{Grading}
Let $(X,p) \in \mathsf{Top}_\ast$. For $n \in \omega$ let $L^n := \pi_{n + 1}(X,p)$ and define
\begin{equation*}
	L := \bigoplus_{n \in \omega} L^n.
\end{equation*}
Moreover, define $[-,-] : L \times L \to L$ by
\begin{equation*}
	\sbr[3]{\sum_i \alpha_i, \sum_j \beta_j} := \sum_{i,j} [\alpha_i,\beta_j].
\end{equation*}
Then clearly $L^nL^m \subseteq L^{n + m}$ holds. It also turns out, that we have a Lie algebra-like structure on $L$, i.e. the bracket is bilinear, alternating and there is a Jacobi identity (for more details see \cite[474--478]{whitehead:homotopy_theory:1978}). 

\begin{proposition}
	Let $n,m \in \omega$, $n \geq 1$, $[\alpha_1], [\alpha_2] \in \pi_{n + 1}(X)$ and $[\beta] \in \pi_{m + 1}(X)$. Then
	\begin{equation*}
		[\alpha_1 + \alpha_2, \beta] = [\alpha_1,\beta] + [\alpha_2,\beta] \qquad \text{and} \qquad [\beta,\alpha_1 + \alpha_2] = [\beta,\alpha_1] + [\beta,\alpha_2].
	\end{equation*}
\end{proposition}

Recall, that for $n \geq 1$ we have that $H_n(\mathbb{S}^n) \cong \mathbb{Z}$. Thus if we are given any continuous map $f : \mathbb{S}^n \to \mathbb{S}^n$, the induced map $H_n(f)$ is simply a multiplication by a unique integer. This integer is defined to be the \bld{degree of $f$}, written $\deg f$.

\begin{proposition}
	Let $n,m \in \omega$, $[\alpha] \in \pi_{n + 1}(X)$ and $[\beta] \in \pi_{m + 1}(X)$. Then
	\begin{equation*}
		[\beta,\alpha] = (-1)^{(n + 1)(m + 1)}[\alpha,\beta].
	\end{equation*}
\end{proposition}

\begin{proof}
	Consider the \bld{permutation map} $\sigma : \mathbb{S}^{m + n + 1} \to \mathbb{S}^{n + m + 1}$ defined by
	\begin{equation*}
		(y_1,\dots,y_{m + 1},x_1,\dots,x_{n + 1}) \mapsto (x_1, \dots,x_{n + 1},y_1,\dots,y_{m + 1}).
	\end{equation*}
	Then clearly $\deg \sigma = (-1)^{(n + 1)(m + 1)}$, since $\sigma$ is the composition of permutations and hence orthogonal transformations. Using lemma \ref{lem:order_reversing}, we compute
	\begin{equation*}
		[\beta,\alpha] = [(\beta \vee \alpha) \circ f] = [(\alpha \vee \beta) \circ f \circ \sigma] = (-1)^{(n + 1)(m + 1)}[\alpha,\beta].
	\end{equation*}
\end{proof}

\begin{proposition}
	Let $n,m,r \in \omega$, $n,m,r \geq 1$, $[\alpha] \in \pi_{n + 1}(X)$, $[\beta] \in \pi_{m + 1}(X)$ and $[\gamma] \in \pi_{r + 1}(X)$. Then
	\begin{equation*}
		(-1)^{r(n + 1)}[\alpha,[\beta,\gamma]] + (-1)^{n(m + 1)}[\beta,[\gamma,\alpha]] + (-1)^{m(r + 1)}[\gamma,[\alpha,\beta]] = 0
	\end{equation*}
\end{proposition}

\printbibliography
\end{document}
