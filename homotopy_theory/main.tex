\input{header.tex}

\title{Whitehead Product}
\author{Yannis B\"{a}hni}
\address[Yannis B\"{a}hni]{University of Zurich, R\"{a}mistrasse 71, 8006 Zurich}
\email[Yannis B\"{a}hni]{\href{mailto:yannis.baehni@uzh.ch}{\nolinkurl{yannis.baehni@uzh.ch}}}

\begin{document}

\begin{abstract}
\end{abstract}

\maketitle

\tableofcontents

\section{Definition of the Whitehead Product}
Notice, that for any $(X,x_0),(Y,y_0) \in \mathsf{Top}_*$, their coproduct is given by
\begin{equation*}
	X \coprod Y = (X \times \cbr[0]{y_0}) \cup (\cbr[0]{x_0} \times Y) \subseteq X \times Y,
\end{equation*}
\noindent with basepoint $(x_0,y_0)$. 

\begin{lemma}
	\label{lem:attaching_cell_wedge_sum_spheres}
	Let $n,m \in \omega$, $n,m \geq 1$. The space $\mathbb{S}^n \times \mathbb{S}^m$ can be obtained from $\mathbb{S}^n \vee \mathbb{S}^m$ by attaching an $n + m$-cell.
\end{lemma}

\begin{proof}
	Observe, that $\mathbb{D}^{n + m} \cong \mathbb{D}^n \times \mathbb{D}^m$ and hence
	\begin{equation*}
		\mathbb{S}^{n + m - 1} = \partial \mathbb{D}^{n + m} \cong (\partial \mathbb{D}^n \times \mathbb{D}^m) \cup (\mathbb{D}^n \times \partial \mathbb{D}^m) = (\mathbb{S}^{n - 1} \times \mathbb{D}^m) \cup (\mathbb{D}^n \times \mathbb{S}^{m - 1}).
	\end{equation*}
	Let 
	\begin{equation*}
		f_1 : \mathbb{S}^{n - 1} \times \mathbb{D}^m \to (\mathbb{S}^{n - 1} \times \mathbb{D}^m) / (\mathbb{S}^{n - 1} \times \mathbb{S}^{m - 1}) \cong \ast \times \mathbb{S}^m
	\end{equation*}
	\noindent and
	\begin{equation*}
		f_2 : \mathbb{D}^n \times \mathbb{S}^{m - 1} \to (\mathbb{D}^n \times \mathbb{S}^{m - 1}) / (\mathbb{S}^{n - 1} \times \mathbb{S}^{m - 1}) \cong \mathbb{S}^n \times \ast
	\end{equation*}
	\noindent be the quotient maps. An application of the gluing lemma thus yields a map
	\begin{equation*}
		f : \mathbb{S}^{n + m - 1} \to \mathbb{S}^n \vee \mathbb{S}^m.
	\end{equation*}
	Moreover, define 
	\begin{equation*}
		q : \mathbb{D}^n \times \mathbb{D}^m \to \mathbb{D}^n/\mathbb{S}^{n - 1} \times \mathbb{D}^m/\mathbb{S}^{m - 1} \cong \mathbb{S}^n \times \mathbb{S}^m	
	\end{equation*}
	\noindent to be the product of quotient maps. Thus we get a commutative diagram
	\begin{equation*}
		\begin{tikzcd}
			\mathbb{S}^{n + m - 1} \arrow[r,"f"]\arrow[d,hook] & \mathbb{S}^n \vee \mathbb{S}^m \arrow[d,hook]\\
			\mathbb{D}^{n + m} \arrow[r,"q"'] & \mathbb{S}^n \times \mathbb{S}^m
		\end{tikzcd}
	\end{equation*}
	Suppose $(X,g,h)$ is another cocone for the pushout diagram:
	\begin{equation*}
		\begin{tikzcd}
			\mathbb{S}^{n + m - 1} \arrow[r,"f"]\arrow[d,hook] & \mathbb{S}^n \vee \mathbb{S}^m \arrow[d,hook]\arrow[ddr,bend left,"h"]\\
			\mathbb{D}^{n + m} \arrow[r,"q"']\arrow[drr,bend right,"g"'] & \mathbb{S}^n \times \mathbb{S}^m\\
			& & X.
		\end{tikzcd}
	\end{equation*}
	By \cite[186]{munkres:topology:2000}, $q$ is a quotient map. Moreover, for $(x,y) \in \mathbb{S}^{n - 1} \times \mathbb{S}^{m - 1}$, we have that
	\begin{equation*}
		g(x,y) = (h \circ f)(x,y) = h(\ast,\ast).
	\end{equation*}
	Thus $g$ passes to the quotient by \cite[72]{lee:topological_manifolds:2011} to yield a unique map
	\begin{equation*}
		\wtilde{g} : \mathbb{S}^n \times \mathbb{S}^m \to X,
	\end{equation*}
	\noindent such that $g = \wtilde{g} \circ q$. Finally, it is easy to check that
	\begin{equation*}
		\begin{tikzcd}
			\mathbb{S}^{n + m - 1} \arrow[r,"f"]\arrow[d,hook] & \mathbb{S}^n \vee \mathbb{S}^m \arrow[d,hook]\arrow[ddr,bend left,"h"]\\
			\mathbb{D}^{n + m} \arrow[r,"q"']\arrow[drr,bend right,"g"'] & \mathbb{S}^n \times \mathbb{S}^m\arrow[dr,"\wtilde{g}"]\\
			& & X.
		\end{tikzcd}
	\end{equation*}
	\noindent commutes.
\end{proof}

For $n,m \in \omega$, $n,m \geq 1$, consider the map $f$ from lemma \ref{lem:attaching_cell_wedge_sum_spheres}. Let $(X,p) \in \mathsf{Top}_*$. If $[\alpha] \in \pi_n(X,p)$ and $[\beta] \in \pi_m(X,p)$, we get two pointed maps
\begin{equation*}
	\alpha : \mathbb{S}^n \to X \qquad \text{and} \qquad \beta : \mathbb{S}^m \to X.
\end{equation*}
Forming their wedge $\alpha \vee \beta : \mathbb{S}^n \vee \mathbb{S}^m \to X$, defined by
\begin{align*}
	(\alpha \vee \beta) (x,y) := \ccases{
		\alpha(x) & y = \ast,\\
		\beta(y) & x = \ast,
	}
\end{align*}
\noindent and precomposing with $f$, yields a pointed map
\begin{equation*}
	(\alpha \vee \beta) \circ f : \mathbb{S}^{n + m - 1} \to X.
\end{equation*}
Moreover, it is easy to check that above map is well behaved under pointed homotopies, hence gives raise to a class $[(\alpha \vee \beta) \circ f]$.

\begin{definition}[Whitehead Product]
	Let $n,m \in \omega$, $n,m \geq 1$, and $(X,p) \in \mathsf{Top}_\ast$. The product
	\begin{equation*}
		\pi_n(X,p) \times \pi_m(X,p) \to \pi_{n + m - 1}(X,p)
	\end{equation*}
	\noindent defined by
	\begin{equation*}
		([\alpha],[\beta]) \mapsto [(\alpha \vee \beta) \circ f]
	\end{equation*}
	\noindent is called the \bld{Whitehead product}.
\end{definition}
\end{document}
