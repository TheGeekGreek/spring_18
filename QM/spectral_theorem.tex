\section*{Postulates of Quantum Mechanics}
\begin{figure}[h!tb]
	\centering
	\begin{tabular}{c|c}
		\bld{Quantum mechanical system} & Hilbert space $\mathcal{H}$\\
		\hline
		\hline
		\bld{State} & $\psi \in \mathcal{H}$, $\norm{\psi} = 1$\\
		\hline
		\hline
		\bld{Observables} & Self-adjoint operators on $\mathcal{H}$\\
		\hline
		\hline
		\bld{Expected Value} of an observable $A$ in the state $\psi$ & $\langle \psi ,A \psi \rangle$\\
		\hline
		\hline
		\bld{Variance} of an observable $A$ in the state $\psi$ & $\Delta A_\psi := \langle \psi, A^2 \psi\rangle - \langle \psi,A\psi\rangle^2$
	\end{tabular}
\end{figure}

\begin{lemma}[Heisenberg Uncertainity Principle]
	Let $A$ and $B$ two self-adjoint operators on a Hilbert space $\mathcal{H}$. Then for any state $\psi$
	\begin{equation*}
		\Delta A_\psi \Delta B_\psi \geq \frac{1}{4} \abs[0]{\langle \psi,[A,B]\psi\rangle}^2.
	\end{equation*}
\end{lemma}

\section*{Unbounded Operators}
\begin{definition}[Linear Operator]
	Let $\mathcal{H}$ be a Hilbert space. A \bld{(linear) operator on $\mathcal{H}$} is simply a linear map $A : D(A) \to \mathcal{H}$, where $D(A)$ is a linear subspace of $\mathcal{H}$.
\end{definition}

\begin{definition}[Closed Operator]
	An operator $A$ on $\mathcal{H}$ is said to be \bld{closed}, iff $\Gamma_A$ is closed in $\mathcal{H} \times \mathcal{H}$.
\end{definition}

\begin{definition}[Closable Operator]
	An operator $A$ on $\mathcal{H}$ is said to be \bld{closable}, iff $\wbar{\Gamma}_A$ is a linear graph, i.e. $(0,y) \in \Gamma_A$ implies $y = 0$. The corresponding operator associated to $\wbar{\Gamma}_A$ is denoted by $\wbar{A}$ and called the \bld{closure} of $A$. Clearly $A \subseteq \wbar{A}$.
\end{definition}

\begin{definition}[Adjoint]
	Let $A$ be a densly defined operator on $\mathcal{H}$. Set
	\begin{equation*}
		D(A^*) := \cbr[0]{\psi \in \mathcal{H} : \exists \eta \in \mathcal{H} \text{ s.t. } \forall \varphi \in D(A) \>\langle A \varphi,\psi \rangle = \langle \varphi,\eta \rangle},
	\end{equation*}
	\noindent and $A^*\psi := \eta$. The operator $A^*$ is called the \bld{adjoint} of $A$.
\end{definition}

\begin{theorem}
	Let $A$ be a densly defined operator on $\mathcal{H}$. Then:
	\begin{enumerate}[label = \textup{(}\alph*\textup{)},wide=0pt]
		\item $A^*$ is closed.
		\item $A$ is closable if and only if $D(A^*)$ is dense.
		\item If $A$ is closable, then $(\wbar{A})^* = A^*$.
	\end{enumerate}
\end{theorem}

\begin{definition}[Symmetric Operator]
	A densly defined operator $A$ is said to be \bld{symmetric}, iff $A \subseteq A^*$.
\end{definition}

\begin{definition}[Self-adjoint Operator]
	A densly defined operator is said to be \bld{self-adjoint}, iff $A = A^*$.
\end{definition}

\begin{definition}[Essentially Self-adjoint Operator]
	A A symmetric operator $A$ is said to be \bld{essentially self-adjoint}, iff $\wbar{A}$ is self-adjoint.
\end{definition}

\section*{The Spectral Theorem}
\subsection*{Projection Valued Measures}

\begin{definition}[Projection Valued Measure]
	Let $\mathcal{H}$ be a Hilbert space. A function $P : \mathcal{B}(\mathbb{R}) \to \mathcal{L}(\mathcal{H})$ is said to be a \bld{projection valued measure}, iff
	\begin{enumerate}[label = \textup{(}\roman*\textup{)},wide=0pt]
		\item For all $\Omega \in \mathcal{B}(\mathbb{R})$, $P(\Omega)$ is an orthogonal projection, i.e. 
			\begin{equation*}
				P(\Omega)^2 = P(\Omega) = P(\Omega)^*.
			\end{equation*}
		\item $P(\mathbb{R}) = \id_\mathcal{H}$.
		\item If $(\Omega_n)_{n \in \omega}$ is a sequence of pairwise disjoint elements of $\mathcal{B}(\mathbb{R})$, then
			\begin{equation*}
				P(\Omega)\psi = \sum_{n \in \omega} P(\Omega_n)\psi,
			\end{equation*}
			\noindent for all $\psi \in \mathcal{H}$.
	\end{enumerate}
\end{definition}

\begin{definition}[Resolution of the Identity]
	Let $\mathcal{H}$ be a Hilbert space and $P : \mathcal{B}(\mathbb{R}) \to \mathcal{L}(\mathcal{H})$ a projection valued measure. The function $p : \mathbb{R} \to \mathcal{L}(\mathcal{H})$ defined by
	\begin{equation*}
		p(\lambda) := P\intoc{-\infty,\lambda},
	\end{equation*}
	\noindent is called the \bld{resolution of the identity associated to a projection valued measure}.
\end{definition}

\subsection*{The Spectral Theorem}

\begin{theorem}[Spectral Theorem]
	Let $A$ be a self-adjoint operator. Then there exists a unique projection valued measure $P^A$ such that $D(A) = \cbr[0]{\psi \in \mathcal{H} : \int \abs{\lambda}^2 d\mu_\psi^A(\lambda)}$ and
	\begin{equation*}
		A = \int \lambda dp^A(\lambda).
	\end{equation*}
\end{theorem}

\subsection*{The Schr\"odinger Equation}

\begin{theorem}
	Let $\mathcal{H}$ be a Hilbert space and $H : D(H) \to \mathcal{H}$ be self adjoint. Moreover, set $U(t) := \exp(-iHt)$ for $t \in \mathbb{R}$. Then:
	\begin{enumerate}[label=\textup{(}\alph*\textup{)},wide=0pt]
		\item $U(t)$ is a strongly continuous one parameter unitary group.
		\item The limit
			\begin{equation*}
				\lim_{t \to 0} \frac{U(t) - 1}{t}\psi
			\end{equation*}
			\noindent exists if and only if $\psi \in D(H)$. Then
			\begin{equation*}
				\lim_{t \to 0} \frac{U(t) - 1}{t}\psi = -iH\psi.
			\end{equation*}
		\item $U(t)D(H) = D(H)$ and $[U(t),H] = 0$ on $D(H)$.
		\item Let $\psi_0 \in D(H)$. Then $U(t)\psi_0$ uniquely solves the initial value problem
			\begin{equation}
				\ccases{
					i\partial_t \psi(t) = H\psi(t)\\
					\psi(0) = \psi_0,
				}
			\end{equation}
			\noindent called the \bld{Schr\"odinger equation}.
	\end{enumerate}
\end{theorem}
