\section*{Postulates of Quantum Mechanics}
\begin{figure}[h!tb]
	\centering
	\begin{tabular}{c|c}
		\bld{Quantum mechanical system} & separable Hilbert space $\mathcal{H}$\\
		\hline
		\hline
		\bld{State} & $\psi \in \mathcal{H}$, $\norm{\psi} = 1$\\
		\hline
		\hline
		\bld{Observables} & Self-adjoint operators on $\mathcal{H}$\\
		\hline
		\hline
		\bld{Expected Value} of an observable $A$ in the state $\psi$ & $\langle \psi ,A \psi \rangle$\\
		\hline
		\hline
		\bld{Variance} of an observable $A$ in the state $\psi$ & $\Delta A_\psi := \langle \psi, A^2 \psi\rangle - \langle \psi,A\psi\rangle^2$
	\end{tabular}
\end{figure}

\begin{lemma}[Heisenberg Uncertainity Principle]
	Let $A$ and $B$ two self-adjoint operators on a Hilbert space $\mathcal{H}$. Then for any state $\psi$
	\begin{equation*}
		\Delta A_\psi \Delta B_\psi \geq \frac{1}{4} \abs[0]{\langle \psi,[A,B]\psi\rangle}^2.
	\end{equation*}
\end{lemma}

\section*{Unbounded Operators}
\begin{definition}[Linear Operator]
	Let $\mathcal{H}$ be a Hilbert space. A \bld{(linear) operator on $\mathcal{H}$} is simply a linear map $A : D(A) \to \mathcal{H}$, where $D(A)$ is a linear subspace of $\mathcal{H}$.
\end{definition}

\begin{examples}
	~
	\begin{enumerate}[label = \textup{(}\alph*\textup{)},wide = 0pt]
		\item (\bld{Multiplication operator}) Let $\mathcal{H} := L^2(\mathbb{R})$ and consider $\what{x} : D(\what{x}) \to L^2(\mathbb{R})$ defined by $(\what{x}\psi)(x) := x\psi(x)$ (or $(\what{f}\psi)(x) := f(x)\psi(x)$ for any complex valued measurable function $f$).
		\item (\bld{Differential operator}) Let $\mathcal{H} := L^2(\mathbb{R})$ and consider $\nabla : C^\infty(\mathbb{R}) \to L^2(\mathbb{R})$.
	\end{enumerate}
\end{examples}

\begin{definition}[Closed Operator]
	An operator $A$ on $\mathcal{H}$ is said to be \bld{closed}, iff $\Gamma_A$ is closed in $\mathcal{H} \times \mathcal{H}$.
\end{definition}

\begin{definition}[Closable Operator]
	An operator $A$ on $\mathcal{H}$ is said to be \bld{closable}, iff $\wbar{\Gamma}_A$ is a linear graph, i.e. $(0,y) \in \Gamma_A$ implies $y = 0$. The corresponding operator associated to $\wbar{\Gamma}_A$ is denoted by $\wbar{A}$ and called the \bld{closure} of $A$. Clearly $A \subseteq \wbar{A}$.
\end{definition}

\begin{definition}[Adjoint]
	Let $A$ be a densly defined operator on $\mathcal{H}$. Set
	\begin{equation*}
		D(A^*) := \cbr[0]{\psi \in \mathcal{H} : \exists \eta \in \mathcal{H} \text{ s.t. } \forall \varphi \in D(A) \>\langle A \varphi,\psi \rangle = \langle \varphi,\eta \rangle},
	\end{equation*}
	\noindent and $A^*\psi := \eta$. The operator $A^*$ is called the \bld{adjoint} of $A$.
\end{definition}

\begin{theorem}
	Let $A$ be a densly defined operator on $\mathcal{H}$. Then:
	\begin{enumerate}[label = \textup{(}\alph*\textup{)},wide=0pt]
		\item $A^*$ is closed.
		\item $A$ is closable if and only if $D(A^*)$ is dense.
		\item If $A$ is closable, then $(\wbar{A})^* = A^*$.
	\end{enumerate}
\end{theorem}

\begin{definition}[Symmetric Operator]
	A densly defined operator $A$ is said to be \bld{symmetric}, iff $A \subseteq A^*$.
\end{definition}

\begin{definition}[Self-adjoint Operator]
	A densly defined operator is said to be \bld{self-adjoint}, iff $A = A^*$.
\end{definition}

\begin{example}
	Let $A_f$ denote the multiplication operator. Then $A_f^* = A_{\wbar{f}}$.
\end{example}

\begin{definition}[Essentially Self-adjoint Operator]
	A symmetric operator $A$ is said to be \bld{essentially self-adjoint}, iff $\wbar{A}$ is self-adjoint.
\end{definition}

\begin{example}
	Let $\mathcal{H} := L^2[0,2\pi]$ and consider the operator $A$ defined by $A := -i\frac{d}{dx}$ with $D(A) := \cbr[0]{\psi \in C^1[0,2\pi] : \psi(0) = \psi(2\pi)}$.
\end{example}

\begin{theorem}
	Let $A$ be a symmetric operator. Then the following statements are equivalent:
	\begin{enumerate}[label = \textup{(}\alph*\textup{)},wide = 0pt]
		\item $A$ is self-adjoint.
		\item $A$ is closed and $\ker(A^* \pm i) = \{0\}$.
		\item $\im(A \pm i) = \mathcal{H}$.
	\end{enumerate}
\end{theorem}

There is a way of defining uniquely self-adjoint extensions of symmetric non-negative operators (\bld{Friedrichs extension}).

\begin{lemma}[Weyl Lemma]
	Let $A$ be a closed densly defined operator such that there exists a sequence $(\psi_n)_{n \in \omega}$ in $D(A)$ with $\norm[0]{\psi_n} = 1$ for all $n \in \omega$ and $\norm[0]{(A - z)\psi_n} \to 0$ for some $z \in \mathbb{C}$. Then $z \in \sigma(A)$ (the sequence $(\psi_n)_{n \in \omega}$ is called a \bld{Weyl sequence}).
\end{lemma}

\begin{theorem}
	Let $A$ be a symmetric closed operator. Then $A$ is self-adjoint if and only if $\sigma(A) \subseteq \mathbb{R}$.
\end{theorem}

\section*{The Spectral Theorem}

\begin{definition}[Projection Valued Measure]
	Let $\mathcal{H}$ be a Hilbert space. A function 
	\begin{equation*}
		P : \mathcal{B}(\mathbb{R}) \to \mathcal{L}(\mathcal{H})
	\end{equation*}
	\noindent is said to be a \bld{projection valued measure}, iff
	\begin{enumerate}[label = \textup{(}\roman*\textup{)},wide=0pt]
		\item For all $\Omega \in \mathcal{B}(\mathbb{R})$, $P(\Omega)$ is an orthogonal projection, i.e. $P(\Omega)^2 = P(\Omega) = P(\Omega)^*$.
		\item $P(\mathbb{R}) = \id_\mathcal{H}$.
		\item If $(\Omega_n)_{n \in \omega}$ is a sequence of pairwise disjoint elements of $\mathcal{B}(\mathbb{R})$, then
			\begin{equation*}
				P(\Omega)\psi = \sum_{n \in \omega} P(\Omega_n)\psi,
			\end{equation*}
			\noindent for all $\psi \in \mathcal{H}$.
	\end{enumerate}
\end{definition}

\begin{definition}[Resolution of the Identity]
	Let $\mathcal{H}$ be a Hilbert space and $P$ a projection valued measure. The function $p : \mathbb{R} \to \mathcal{L}(\mathcal{H})$ defined by
	\begin{equation*}
		p(\lambda) := P\intoc{-\infty,\lambda},
	\end{equation*}
	\noindent is called the \bld{resolution of the identity associated to a projection valued measure}.
\end{definition}

\subsubsection*{Functional Calculus}
Let $P : \mathcal{B}(\mathbb{R}) \to \mathcal{L}(\mathcal{H})$ be a projection valued measure. Then for any simple function $f := \sum_{k = 1}^n \alpha_k \chi_{\Omega_k}$ we define 
\begin{equation*}
	P(f) := \int_\mathbb{R} f(\lambda) dp(\lambda) := \sum_{k = 1}^n \alpha_k P(\Omega_k).
\end{equation*}
Since the simple functions are dense in the space of \emph{bounded Borel functions} (with respect to $\norm{\cdot}_\infty$) $\mathcal{M}_b$, we can extend above definition to $\mathcal{M}_b$. Actually, this defines a $C^*$-algebra homomorphism.\\
Consider now $f$ just a Borel function. Then we get an operator $P(f) : D(P(f)) \to \mathcal{H}$ where
\begin{equation*}
	D(P(f)) := \cbr[0]{\psi \in \mathcal{H} : f \in L^2(\mathbb{R},d\mu_\psi)},
\end{equation*}
\noindent defined by
\begin{equation*}
	P(f)\psi := \lim_{n \to \infty} P(f_n)\psi,
\end{equation*}
\noindent where $f_n := f \chi_{\abs[0]{f} \leq n}$. We write
\begin{equation*}
	P(f) = \int_\mathbb{R} f(\lambda)dp(\lambda).
\end{equation*}

\subsubsection*{Existence}
Existence is guaranteed by \emph{Herglotz} or \emph{Nevanlinna} functions. 

\begin{theorem}[Spectral Theorem]
	Let $A$ be a self-adjoint operator. Then there exists a unique projection valued measure $P^A$ such that $D(A) = \cbr[0]{\psi \in \mathcal{H} : \int \abs{\lambda}^2 d\mu_\psi^A(\lambda) < \infty}$ and
	\begin{equation*}
		A = \int \lambda dp^A(\lambda).
	\end{equation*}
\end{theorem}

\begin{theorem}
	Let $A$ be a self-adjoint operator with projection valued measure $P^A$. Then
	\begin{equation*}
		\sigma(A) = \cbr[0]{\lambda \in \mathbb{R} : \forall \varepsilon > 0 \> P^A\intoo{\lambda - \varepsilon,\lambda + \varepsilon} \neq 0)}.
	\end{equation*}
\end{theorem}

\begin{definition}[Spectral Basis]
	Let $A$ be a self-adjoint operator. A family $(\psi_\iota)_{\iota \in I}$ in $\mathcal{H}$ is said to be a \bld{spectral basis} of $\mathcal{H}$, iff $\mathcal{H}_{\psi_i} \perp \mathcal{H}_{\psi_j}$ for all $i \neq j$, where
	\begin{equation*}
		H_{\psi_i} := \cbr[0]{f(A)\psi_i \in \mathcal{H} : f \in L^2(\mathbb{R},d\mu_{\psi_i})},
	\end{equation*}
	\noindent and $\mathcal{H} = \bigoplus_{\iota \in I} \mathcal{H}_{\psi_\iota}$.
\end{definition}

Now for any self-adjoint operator there exists a at most countable spectral basis $(\psi_\iota)_{\iota \in I}$ and a unitary operator $U : \mathcal{H} \to \bigoplus_{\iota \in I} L^2(\mathbb{R},d\mu_{\psi_\iota})$ such that $Uf(A)U^* = f$, where $f$ acts as a multiplication operator on each coordinate. Thus \emph{any self-adjoint operator is unitarly equivalent to a multiplication operator}.\\
Moreover, for any Borel measure $\mu$ we have a decomposition
\begin{equation*}
	L^2(\mathbb{R},d\mu) = L^2(\mathbb{R},d\mu_{\mathrm{ac}}) \oplus L^2(\mathbb{R},d\mu_{\mathrm{pp}}) \oplus L^2(\mathbb{R},d\mu_{\mathrm{sc}}).
\end{equation*} 

\section*{Quantum Dynamics}
\subsection*{The Schr\"odinger Equation}

\begin{align*}
	& i\frac{d\psi}{dt} = H\psi \qquad & \mathrm{(\textbf{Time-dependent Sch\"odinger equation})},\\
	& H\psi = E\psi &  \mathrm{(\textbf{Stationary Sch\"odinger equation})}.
\end{align*}

\begin{theorem}
	Let $\mathcal{H}$ be a Hilbert space and $H : D(H) \to \mathcal{H}$ be self adjoint. Moreover, set $U(t) := \exp(-iHt)$ for $t \in \mathbb{R}$. Then:
	\begin{enumerate}[label=\textup{(}\alph*\textup{)},wide=0pt]
		\item $U(t)$ is a strongly continuous one parameter unitary group.
		\item The limit
			\begin{equation*}
				\lim_{t \to 0} \frac{U(t) - 1}{t}\psi
			\end{equation*}
			\noindent exists if and only if $\psi \in D(H)$. Then
			\begin{equation*}
				\lim_{t \to 0} \frac{U(t) - 1}{t}\psi = -iH\psi.
			\end{equation*}
		\item $U(t)D(H) = D(H)$ and $[U(t),H] = 0$ on $D(H)$.
		\item Let $\psi_0 \in D(H)$. Then $U(t)\psi_0$ uniquely solves the initial value problem
			\begin{equation}
				\ccases{
					i\partial_t \psi(t) = H\psi(t)\\
					\psi(0) = \psi_0,
				}
			\end{equation}
			\noindent called the \bld{Schr\"odinger equation}.
	\end{enumerate}
\end{theorem}

Hence every self-adjoint operator $H$ generates a strongly continuous one-parameter unitary group $U(t) = e^{-itH}$. A converse is given by \bld{Stone's theorem}, which states that any weakly continuous one-parameter unitary group gives rise to a self-adjoint operator $H$ such that $U(t) = e^{-itH}$.

\begin{theorem}[Stone]
	Let $U : \mathbb{R} \to \mathcal{L}(\mathcal{H})$ be a weakly continuous one-parameter unitary group. Define $H : D(H) \to \mathcal{H}$	by
	\begin{equation*}
		D(H) := \cbr[1]{\psi \in \mathcal{H} : \lim_{t \to 0} 1/t (U(t)\psi - \psi) \text{ exists}}
	\end{equation*}
	\noindent and
	\begin{equation*}
		H\psi := \lim_{t \to 0} \frac{i}{t} (U(t)\psi - \psi).
	\end{equation*}
	Then $H$ is self-adjoint and $U(t) = e^{-itH}$.
\end{theorem}

\emph{Wiener and RAGE-Theorem}.

\section*{Perturbation Theory}

\begin{theorem}[Kato-Rellich]
	Let $A$ be self-adjoint and $B$ symmetric bounded with respect to $A$ with $A$-bound less than one. Then $A + B$ on $D(A + B) := D(A)$ is self-adjoint.
\end{theorem}

Next we want to investigate if other properties are preserved by perturbations. Clearly, eigenvalues are not preserved. However, we can remove them from the spectrum.

\begin{corollary}
	Let $A$ be self-adjoint and $K$ self-adjoint and (relatively) compact. Then $\sigma_{\mathrm{ess}}(A + K) = \sigma_{\mathrm{ess}}(A)$, where $\sigma_\mathrm{ess}(A) := \cbr[0]{\lambda \in \sigma(A) : \forall \varepsilon > 0 \>\mathrm{rank}P^A\intoo{\lambda-\varepsilon,\lambda + \varepsilon} = \infty}$ 
\end{corollary}

\begin{proof}
	Use Weyl characterization of the essential spectrum via \bld{singular Weyl sequences} (Weyl sequencenses where additionally $\psi_n \rightharpoonup 0$).
\end{proof}

\section*{Time Evolution of explicit Operators}

\begin{examples}
	Let us consider some explicit self-adjoint operators $H$.
	\begin{enumerate}[label = \textup{(}\alph*\textup{)},wide = 0pt]
		\item (\bld{Free Particles}) $\mathcal{H} := L^2(\mathbb{R}^d,dx)$ and $H := -\Delta$. Then
			\begin{equation*}
				\sigma(H) = \sigma_{\mathrm{ac}}(A) = \intco{0,\infty}.
			\end{equation*}
		\item (\bld{Harmonic Oscillator}) $\mathcal{H} := L^2(\mathbb{R}^d,dx)$ and $H := -\Delta + \omega^2x^2$, for some $\omega \in \mathbb{R}$. Then
			\begin{equation*}
				\sigma(H) = \sigma_{\mathrm{pp}}(H) = \omega(2 \mathbb{N} + 1).
			\end{equation*}
		\item (\bld{One dimensional System}) $\mathcal{H} := L^2(\mathbb{R}^d,dx)$ and $H := -\Delta + V(x)$, where
			\begin{align*}
				V(x) := \ccases{
					-b & \abs{x} < a,\\
					0 & \abs{x} \geq a.
				}
			\end{align*}
			\noindent for some $a,b > 0$. Then
			\begin{equation*}
				\sigma_{\mathrm{ess}}(H) = \sigma_{\mathrm{ess}}(-\Delta) = \intco{0,\infty}
			\end{equation*}
			\noindent by Weyl's theorem and we can ask for eigenvalues.
		\item (\bld{Hydrogen Atom}) $\mathcal{H} := L^2(\mathbb{R}^3,dx)$ and $H := -\Delta - \frac{1}{\abs{x}}$ (use the invariance of $H$ under rotations to solve the stationary Sch\"odinger equation). Then
			\begin{equation*}
				\sigma(H) = \cbr[0]{-1/(4n^2) : n \in \mathbb{N}} \cup \intco{0,\infty}.
			\end{equation*}
	\end{enumerate}
\end{examples}

\section*{Stationary States of General Schr\"odinger Operators}
Consider $H := -\Delta + V(x)$, for $V$ locally integrable on $\mathbb{R}^n$. Moreover, consider 
\begin{equation*}
	\varepsilon(\psi) := \langle \psi,H\psi \rangle.
\end{equation*}
We are looking for \emph{existence of minimizers of $\varepsilon$}. First of all, \emph{boundedness from below} is required. Then we get a \emph{variational characterization of the smallest eigenvalue of $H$}. Then \emph{variational characterization of all the negative eigenvalues of $H$}. 

\section*{Semiclassical Approximations}

Obtaining informations on eigenvalues of a Schr\"odinger operator by considering the classical system. We consider here only a Dirichlet-boundary value problem for the Laplacian for bounded open $\Omega \subseteq \mathbb{R}^n$.
