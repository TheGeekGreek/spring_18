\section*{Maximum Principle}
\subsection*{Weak Maximum Principle}

Let $\Omega \subseteq \subseteq \mathbb{R}^n$. In what follows, we consider the second order homogenous differential operator in non-divergence form
\begin{equation*}
	Lu := a^{ij}\frac{\partial^2u}{\partial x^i \partial x^j} + b^i \frac{\partial u}{\partial x^i} + c u
\end{equation*}
\noindent where $a^{ij},b^i,c \in C^0(\wbar{\Omega})$ and $L$ is uniformly elliptic, i.e. there exists $\lambda > 0$ such that
\begin{equation*}
	a^{ij}(x)\xi_i\xi_j \geq \lambda \abs{\xi}^2
\end{equation*}
\noindent holds for all $\xi \in \mathbb{R}^n$ and $x \in \Omega$.

\begin{theorem}[Weak Maximum Principle]
	\label{thm:WMP}
	Let $\Omega \subseteq \subseteq \mathbb{R}^n$ and $u \in C^2(\Omega) \cap C^0(\wbar{\Omega})$ such that $Lu \geq 0$. Then:
	\begin{enumerate}[label = \textup{(}\alph*\textup{)},wide = 0pt]
		\item If $c \leq 0$ in $\Omega$, then $\max_{\wbar{\Omega}} u \leq \max_{\partial \Omega} u_+$.
		\item If $c = 0$ in $\Omega$, then $\max_{\wbar{\Omega}} u = \max_{\partial \Omega} u$.
	\end{enumerate}
\end{theorem}

\begin{proof}
	Consider the perturbation $v_\varepsilon := u + \varepsilon e^{\gamma x_1}$ for $\varepsilon,\gamma > 0$ and use the first and second derivative test.
\end{proof}

\subsection*{Strong Maximum Principle}

\begin{lemma}[Boundary Point Lemma, E. Hopf]
	Let $B := B_\rho(y) \subseteq \mathbb{R}^n$ and $u \in C^2(B) \cap C^0(\wbar{B})$ such that $Lu \geq 0$ in $B$ with $c \leq 0$. Assume for some $x_0 \in \partial B$ that $u(x_0) \geq 0$ and $u(x) < u(x_0)$ for every $x \in B$. Then
	\begin{equation*}
		\limsup_{h \to 0} \frac{u(x_0 + h\eta) - u(x_0)}{h} < 0
	\end{equation*}
	\noindent for $\eta$ the invard pointing unit normal at $x_0$.
\end{lemma}

\begin{proof}
	Without loss of generality one can assume $\rho = 1$ and $y = 0$. Then define $w : \wbar{B} \to \mathbb{R}$ by
	\begin{equation*}
		w(x) := e^{-\alpha \abs{x}^2} - e^{-\alpha}
	\end{equation*}
	\noindent for some $\alpha > 0$ to be determined. We compute
	\begin{equation*}
		Lw \geq e^{-\alpha\abs{x}^2}(4\mu \abs{x}^2\alpha^2 -2 \alpha (\tr A + b^i x_i) + c).
	\end{equation*}
	Thus for some $\alpha$ large enough, we get that $Lw > 0$. Set 
	\begin{equation*}
		v := u - u(x_0) + \varepsilon w
	\end{equation*}
	\noindent for some $\varepsilon > 0$ on the anulus $A := \wbar{B}_1(0) \setminus B_{1/2}(0)$. For $\varepsilon > 0$ sufficiently small, we get that $v \leq 0$ on $\partial A$. Since moreover
	\begin{equation*}
		Lv = Lu - cu(x_0) + \varepsilon Lw > 0
	\end{equation*}
	\noindent the weak maximum principle implies $v \leq 0$ on $A$. Hence $D^+_\eta v \leq 0$, but 
	\begin{equation*}
		D^+_\eta v = D^+_\eta u + \varepsilon D^+_\eta w
	\end{equation*}
	\noindent which yields the statement by observing that $D^+_\eta w > 0$.
\end{proof}


