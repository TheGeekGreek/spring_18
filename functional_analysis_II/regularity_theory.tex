\section*{Regularity Theory}

Goal of this section is to prove the following regularity result.

\begin{theorem}[Global Regularity]
	\label{thm:regularity}
	Let $\Omega \subseteq \subseteq \mathbb{R}^n$ of class $C^{k + 2}$ and $f \in H^k(\Omega)$ for some $k \in \omega$. Moreover, let $u \in H^1_0(\Omega)$ be the unique solution of the homogenous Dirichlet boundary value problem
	\begin{align*}
		\ccases{
			A_0 u = f & \text{in } \Omega,\\
			u = 0 & \text{on } \partial \Omega.
		}
	\end{align*}
	\noindent where $a^{ij} \in C^{k + 1}(\wbar{\Omega})$. Then $u \in H^{k + 2}(\Omega)$ and 
	\begin{equation*}
		\norm{u}_{H^{k + 2}(\Omega)} \leq C\norm{f}_{H^k(\Omega)}.
	\end{equation*}
\end{theorem}

\subsection*{Interior Regularity}

\begin{theorem}
	Let $\Omega \subseteq \subseteq \mathbb{R}^n$ of class $C^1$ and $L$ an elliptic operator in divergence form satisfying $a^{ij} \in C^{k + 1}(\wbar{\Omega})$. If $f \in H^k(\Omega)$, the unique weak solution $u \in H^1_0(\Omega)$ of the homogenous Dirichlet boundary value problem
	\begin{align*}
		\ccases{
			A_0 u = f & \text{in } \Omega,\\
			u = 0 & \text{on } \partial\Omega
		}
	\end{align*}
	\noindent belongs to $H^{k + 2}_{\mathrm{loc}}(\Omega)$ and for all $\Omega' \subseteq \subseteq \Omega$ we have the estimate
	\begin{equation*}
		\norm{u}_{H^{k + 2}(\Omega')} \leq C(\norm{f}_{H^k(\Omega)} + \norm{u}_{H^1(\Omega)}).
	\end{equation*}
\end{theorem}

\begin{proof}
	\begin{enumerate}[label = \emph{Step \arabic*:}, wide = 0pt]
		\item
			~
			$k = 0$. 
			\begin{enumerate}[label = \textup{(}\alph*\textup{)},wide = 0pt]
				\item \emph{A-priori Estimates.} First of all, we are assuming that $u \in H^2_{\text{loc}}(\Omega)$. 
				\begin{enumerate}[label = \textup{(}\roman*\textup{)}]
					\item \emph{$H^1$-Estimate.} Choose a bump function $\varphi \in C^\infty_c(\Omega)$ supported in $\Omega'$. Thus the weak formulation yields by plugging in the test function $\varphi^2 u$
						\begin{equation}
							\label{eq:weak_explicit}
							\int_\Omega \varphi^2 a^{ij}\frac{\partial u}{\partial x^i}\frac{\partial u}{\partial x^j} + 2\int_{\Omega} u\varphi a^{ij}\frac{\partial u}{\partial x^i}\frac{\partial \varphi}{\partial x^j} = \int_\Omega f \varphi^2 u.
						\end{equation}
						Rearanging formula (\ref{eq:weak_explicit}) we compute
						\begin{align*}
							\int_\Omega \varphi^2 a^{ij}\frac{\partial u}{\partial x^i}\frac{\partial u}{\partial x^j} &= \int_\Omega f \varphi^2 u -  2\int_{\Omega} u\varphi a^{ij}\frac{\partial u}{\partial x^i}\frac{\partial \varphi}{\partial x^j}\\
							&\leq \norm{f}_{L^2(\Omega)}^2 + \norm{u}_{L^2(\Omega)}^2 + 2 \Lambda \int_\Omega (-u)\varphi \abs[0]{\nabla u}\abs[0]{\nabla \varphi}\\
							&\leq \norm{f}_{L^2(\Omega)}^2 + \norm{u}_{L^2(\Omega)}^2 + \Lambda \varepsilon \norm[0]{\varphi\nabla u}_{L^2(\Omega)}^2 + \frac{\Lambda}{\varepsilon}\norm[0]{u \nabla \varphi}_{L^2(\Omega)}^2
						\end{align*}
						Noticing that
						\begin{equation*}
							\int_\Omega \varphi^2 a^{ij}\frac{\partial u}{\partial x^i}\frac{\partial u}{\partial x^j} \geq \lambda \norm[0]{\varphi\nabla u}_{L^2(\Omega)}^2
						\end{equation*}
						\noindent yields
						\begin{equation*}
							(\lambda - \Lambda \varepsilon) \norm[0]{\varphi\nabla u}_{L^2(\Omega)}^2 \leq  \norm{f}_{L^2(\Omega)}^2 + \norm{u}_{L^2(\Omega)}^2 + \frac{\Lambda}{\varepsilon}\norm[0]{u \nabla \varphi}_{L^2(\Omega)}^2
						\end{equation*}
						Picking $\varepsilon > 0$ appropriately, yields
						\begin{equation*}
							\norm[0]{\varphi\nabla u}_{L^2(\Omega)}^2 \leq  C(\norm{f}_{L^2(\Omega)}^2 + \norm{u}_{L^2(\Omega)}^2)
						\end{equation*}
						\noindent and thus
						\begin{equation*}
							\norm[0]{\nabla u}_{L^2(\Omega')}^2 \leq C(\norm{f}_{L^2(\Omega)}^2 + \norm{u}_{L^2(\Omega)}^2).
						\end{equation*}
					\item \emph{$H^2$-Estimate.} Let $1 \leq \mu \leq n$. Then $\partial_\mu u$ solves
						\begin{equation*}
							\int_\Omega a^{ij}\partial_i\partial_\mu u\partial_j\varphi = - \int_\Omega f\partial_\mu \varphi - \int_\Omega \partial_\mu a^{ij}\partial_i u\partial_j \varphi
						\end{equation*}
						\noindent for all $\varphi \in C^\infty_c(\Omega)$. Now perform the $H^1$-estimate on $\partial_\mu u$. 
				\end{enumerate}
			\item \emph{Existence: The Nirenberg-Trick.} The trick is to use difference quotients
				\begin{equation*}
					D_hu := \frac{\tau_hu - u}{\abs{h}}
				\end{equation*}
				\noindent for $h \in \mathbb{R}^n$ such that $\abs{h} < \dist(\Omega',\partial \Omega)$. The idea now is to find a PDE solved by $D_h u$ in the weak sense and to to use the characterization of the Sobolev space.
			\end{enumerate}
		\item \emph{Induction Step.}
	\end{enumerate}
\end{proof}

\subsection*{Boundary Regularity}

\begin{proposition}[Minimality Property]
	Let $\Omega \subseteq \subseteq \mathbb{R}^n$. Then $u \in H^1_0(\Omega)$ solves (\ref{eq:HDP}) if and only if the \bld{energy functional} satisfies
	\begin{equation*}
		E(u) := \frac{1}{2}\norm{u}^2_a - \int_\Omega fu = \inf_{v \in H^1_0(\Omega)} E(v).
	\end{equation*}
\end{proposition}

\begin{proof}
	Suppose $u \in H^1_0(\Omega)$ solves (\ref{eq:HDP}) and let $v \in H^1_0(\Omega)$. Then $v = u + \varphi$ for some $\varphi \in H^1_0(\Omega)$ and we compute
	\begin{equation*}
		E(v) = E(u + \varphi) = \frac{1}{2}\norm{u}^2_a + \langle u,\varphi \rangle_a + \frac{1}{2}\norm{\varphi}^2_a - \int_\Omega f(u + \varphi) = E(u) + \frac{1}{2}\norm{\varphi}^2_a \geq E(u).
	\end{equation*}
	Conversly, suppose $u_0 \in H^1_0(\Omega)$ is a minimizer of the energy functional. Thus by elementary calculus
	\begin{equation*}
		\frac{d}{dt}\bigg\vert_{t = 0} E(u_0 + tv) = 0
	\end{equation*}
	\noindent for all $v \in H^1_0(\Omega)$. But
	\begin{equation*}
		\frac{d}{dt}\bigg\vert_{t = 0} E(u_0 + tv) = \langle u_0, v \rangle_a - \int_\Omega f v.
	\end{equation*}
\end{proof}

\subsection*{Eigenfunctions of $-\Delta$}

\begin{theorem}
	Let $\Omega \subseteq \subseteq \mathbb{R}^n$ of class $C^2$. Then there exists a Hilbert-space basis $(\varphi_i)_{i \in \omega}$ of $L^2(\Omega)$ consisting of eigenfunctions of the Laplace opertator, i.e. 
	\begin{align*}
		\ccases{
			-\Delta \varphi_i = \lambda_i \varphi_i & \text{in } \Omega,\\
			\varphi_i = 0 & \text{on } \partial \Omega.
		}
	\end{align*}
	Moreover $0 < \lambda_i \to \infty$ are called \bld{Dirichlet eigenvalues}.
\end{theorem}

\begin{proof}
	Define $K : L^2(\Omega) \to L^2(\Omega)$ by setting $Kf$ to be the unique weak solution of the homogenous Dirichlet boundary value problem
	\begin{align*}
		\ccases{
			-\Delta u = f & \text{in } \Omega,\\
			u = 0 & \text{on } \partial \Omega.
		}
	\end{align*}
	By the global regularity theorem, $u \in H^2(\Omega)$ and thus we can write $K$ as the composition
	\begin{equation*}
		\begin{tikzcd}
			L^2(\Omega) \arrow[r] & H^2(\Omega) \arrow[r,hook] & L^2(\Omega).
		\end{tikzcd}
	\end{equation*}
	Thus $K$ is continuous as a composition of continuous mappings an moreover, since the embedding is compact by the Sobolev theorem, so is $K$.
\end{proof}
