\input{header.tex}

\setcounter{section}{1}

\title{Functional Analysis II Summary}
\author{Yannis B\"{a}hni}
\address[Yannis B\"{a}hni]{University of Zurich, R\"{a}mistrasse 71, 8006 Zurich}
\email[Yannis B\"{a}hni]{\href{mailto:yannis.baehni@uzh.ch}{\nolinkurl{yannis.baehni@uzh.ch}}}

\begin{document}

\begin{abstract}
	This is a rough summary of the course \emph{Functional Analysis II} held at \emph{ETH Zurich} by \emph{Prof. Dr. Alessandro Carlotto} in spring $2018$. The main focus of this summary is to give a neat preparation for the oral exam.
\end{abstract}

\maketitle

\tableofcontents

\section*{Introduction}

\begin{theorem}
	Let $\Omega \subseteq \mathbb{R}^n$ be open and $1 \leq p < \infty$. Then $C^\infty_c(\Omega)$ is dense in $L^p(\Omega)$.
\end{theorem}

\begin{proposition}
	If $\mu(X) < \infty$ and $0 < p < q \leq \infty$. Then $L^q(\mu) \subseteq L^p(\mu)$.
\end{proposition}

\begin{proposition}[Integration by Parts]
	Let $(M,g)$ be a compact Riemannian manifold with boundary. Then
	\begin{equation*}
		\int_M \langle \grad f,X \rangle_g dV_g = \int_{\partial M}f \langle X,N \rangle dV_{\wtilde{g}} - \int_M (f \div X) dV_g
	\end{equation*}
	\noindent for $f \in C^\infty(M)$ and $X \in \mathfrak{X}(M)$. Moreover, \bld{Green's identities} hold:
	\begin{equation*}
		\int_M u\Delta v dV_g = \int_M \langle \grad u,\grad v\rangle_g dV_g - \int_{\partial M} uNv dV_{\wtilde{g}}
	\end{equation*}
	\noindent and
	\begin{equation*}
		\int_M (u\Delta v - v \Delta u)dV_g = \int_{\partial M}(vNu - u Nv)dV_{\wtilde{g}}
	\end{equation*}
	\noindent for $u,v \in C^\infty(M)$.
\end{proposition}

Suppose $u \in \mathcal{A}$ is a minimizer of $E_p$ and $\varphi \in C^2(\wbar{\Omega})$ with $\varphi\vert_{\partial\Omega} = 0$. We compute
\begin{align*}
	\frac{d}{dt}E_p(u + t\varphi) &= \frac{d}{dt}\int_\Omega \abs[0]{\nabla u + t \nabla \varphi}^p\\
	&= \frac{d}{dt} \int_\Omega \langle \nabla u + t\nabla \varphi,\nabla u + t\nabla \varphi\rangle^{p/2}\\
	&= p\int_\Omega \abs[0]{\nabla u + t \nabla \varphi}^{p - 2}\langle \nabla \varphi,\nabla u + t\nabla \varphi\rangle.
\end{align*}
In particular
\begin{equation*}
	\frac{d}{dt}\bigg\vert_{t = 0} E_p(u + t\varphi) = p\int_\Omega \abs[0]{\nabla u}^{p - 2} \langle \nabla \varphi,\nabla u\rangle = - \int_\Omega \div(\abs[0]{\nabla u}^{p - 2}\nabla u)\varphi.
\end{equation*}

\section*{Sobolev Space Theory}
\subsection*{The Spaces $W^{k,p}(\Omega)$}
In what follows, let $n \in \omega$, $n \geq 1$, and $1 \leq p \leq \infty$.

\begin{definition}[Distributional and Weak Derivative]
	Let $\Omega \subseteq \mathbb{R}^n$ open and $u \in L^1_{\mathrm{loc}}(\Omega)$. For any multiindex $\alpha$, the \bld{distributional derivative of order $\alpha$ of $u$}, written $D^\alpha u$, is defined to be the mapping $D^\alpha u : C^\infty_c(\Omega) \to \mathbb{R}$ defined by
	\begin{equation*}
		\varphi \mapsto (-1)^{\abs{\alpha}}\int_\Omega u D^\alpha \varphi dx.
	\end{equation*}
	Moreover, a function $D^\alpha u \in L^p(\Omega)$ is called \bld{weak derivative of order $\alpha$ of $u$ with exponent $p$}, iff
	\begin{equation*}
		\forall \varphi \in C^\infty_c(\Omega): \> \int_\Omega D^\alpha u \varphi dx =  (-1)^{\abs{\alpha}} \int_\Omega u D^\alpha \varphi dx.
	\end{equation*}
\end{definition}

\begin{theorem}[Fundamental Lemma of Variational Calculus]
	\label{thm:flvc}
	Let $\Omega \subseteq \mathbb{R}^n$ open and $f \in L^1_{\mathrm{loc}}(\Omega)$. If
	\begin{equation*}
		\forall \varphi \in C^\infty_c(\Omega) : \> \int_\Omega f \varphi dx = 0,
	\end{equation*}
	\noindent then $f = 0$ a.e.
\end{theorem}

\begin{remark}
	Let $\Omega \subseteq \mathbb{R}^n$ open. Then $L^p(\Omega) \subseteq L^1_{\mathrm{loc}}(\Omega)$.
\end{remark}

\begin{remark}
	From the fundamental lemma of variational calculus \ref{thm:flvc} it follows that \emph{weak derivatives, if they exist, are unique}.
\end{remark}

\begin{examples}
	~
	\begin{enumerate}[label = \textup{(}\alph*\textup{)},wide = 0pt]
		\item Consider $\Omega := \intoo{-1,1}$ and $u := \abs{x}$. Then $u' = \chi_{\intco{0,1}} -\chi_{\intoo{-1,0}}$.
		\item Consider $\Omega := \mathbb{R}$ and $u := \chi_{\intoo{0,\infty}}$. Then the weak derivative $u'$ does not exist. Indeed, the \emph{Dirac distribution} is not representable as one may see by considering the smooth family $\varphi_\varepsilon : \mathbb{R} \to \mathbb{R}$ for $\varepsilon > 0$ defined by
			\begin{equation*}
				\varphi_\varepsilon(x) := \begin{cases}
					e^{\varepsilon^2/(x^2 - \varepsilon^2)} & \abs{x} < \varepsilon,\\
					0 & \abs{x} \geq \varepsilon.
				\end{cases}
			\end{equation*}
	\end{enumerate}
\end{examples}

\begin{definition}[Sobolev Space]
	Let $\Omega \subseteq \mathbb{R}^n$ open. For any $k \in \omega$, the \bld{Sobolev space of index $(k,p)$}, written $W^{k,p}(\Omega)$, is defined to be the space
	\begin{equation*}
		W^{k,p}(\Omega) := \cbr[0]{f \in L^p(\Omega) : D^\alpha u \in L^p(\Omega) \text{ exists for all } \abs{\alpha} \leq k},
	\end{equation*}
	\noindent with norm
	\begin{equation*}
		\norm{-} _{W^{k,p}(\Omega)} := \sum_{\abs{\alpha} \leq k} \norm[0]{D^\alpha -}_{L^p(\Omega)}.
	\end{equation*}
	Moreover, define
	\begin{equation*}
		W^{k,p}_0(\Omega) := \overline{C^\infty_c(\Omega)}^{\norm{-}_{W^{k,p}(\Omega)}},
	\end{equation*}
	\noindent and $H^k(\Omega) := W^{k,2}(\Omega)$ as well as $H_0^k(\Omega) := W^{k,2}_0(\Omega)$.
\end{definition}

\begin{theorem}
	Let $\Omega \subseteq \mathbb{R}^n$ open. Then $W^{k,p}(\Omega)$ is
	\begin{enumerate}[label = \textup{(}\alph*\textup{)},wide = 0pt]
		\item a Banach space for all $1 \leq p \leq \infty$.
		\item separable for all $1 \leq p < \infty$.
		\item reflexive for all $1 < p < \infty$.
	\end{enumerate}
\end{theorem}

\begin{proof}
	~
	\begin{enumerate}[label = \textup{(}\alph*\textup{)},wide = 0pt]
		\item This follows from the fact that $L^p(\Omega)$ is a Banach space for all $1 \leq p \leq \infty$. Let $(f_i)_{i \in \omega}$ be a Cauchy sequence in $W^{k,p}$. By definition of the $W^{k,p}$-norm, $(D^\alpha f_i)_{i \in \omega}$ is a Cauchy sequence in $L^p$. Thus we get $D^\alpha f_i \to f_\alpha$ in $L^p$, in particular, $f_i \to f$ in $L^p$. Using H\"older's inequality we compute
			\begin{equation*}
				\int_\Omega f_\alpha \varphi dx = \lim_{i \to \infty} \int_\Omega D^\alpha f_i \varphi dx = (-1)^{\abs{\alpha}} \lim_{i \to \infty}\int_\Omega f_i D^\alpha \varphi dx = (-1)^{\abs{\alpha}}\int_\Omega f D^\alpha \varphi dx
			\end{equation*}
			\noindent for $\varphi \in C^\infty_c(\Omega)$.
		\item For simplicity, we consider $k = 1$ only. Consider $\iota : W^{1,p} \hookrightarrow (L^p)^{n + 1}$ defined in the obvious way. Then $\iota$ is an isometry and the statement follows. 
		\item Same argument as in part (b).
	\end{enumerate}
\end{proof}

\subsection*{Elliptic Operators}

\begin{lemma}[Poincar\'e Inequality]
	\label{lem:PI}
	Let $\Omega \subseteq \mathbb{R}^n$ open and bounded. Then for any $u \in C^\infty_c(\Omega)$ we have that
	\begin{equation*}
		\norm{u}_{L^2} \leq C\norm[0]{\nabla u}_{L^2}.
	\end{equation*}
\end{lemma}

\begin{proof}
	Let $n = 1$. Since $\Omega$ is bounded, we get that $\Omega \subseteq \intcc{a,b}$ and we may extend $u$ on $\intcc{a,b} =: I$ to be zero. Hence an application of Jensen's inequality (or Cauchy-Schwarz) yields
	\begin{equation*}
		\abs{u(x)}^2 = \abs{u(x) - u(a)}^2 = \abs[3]{\int_a^x u'(t)dt}^2 \leq (x - a) \int_a^x \abs[0]{u'(t)}^2dt \leq (b - a) \norm[0]{u'}_{L^2(I)}^2.
	\end{equation*}
	Thus
	\begin{equation*}
		\norm[0]{u}_{L^2(\Omega)}^2 \leq \norm{u}_{L^2(I)}^2 \leq (b - a)^2 \norm[0]{u'}^2_{L^2(I)} = (b - a)^2 \norm[0]{u'}_{L^2(\Omega)}^2
	\end{equation*}
	\noindent where the last equality follows due to the fact that $u$ and thus $u'$ is compactly supported in $\Omega$. If $n > 1$, we have $\Omega \subseteq \intcc{a,b} \times \mathbb{R}^{n-1}$ and thus the claim follows by reduction to the previous case.
\end{proof}

\begin{theorem}[Riesz Representation Theorem]
	\label{thm:RRT}
	Let $H$ be a real Hilbert space. Then the mapping $J : H \to H^*$ defined by $J(x) := \langle x,-\rangle$ is an isometric isomorphism.
\end{theorem}

\begin{theorem}
	Let $\Omega \subseteq\subseteq \mathbb{R}^n$ and consider the elliptic operator
	\begin{equation*}
		L := \frac{\partial}{\partial x^i}\del[3]{a^{ij}\frac{\partial}{\partial x^j}},
	\end{equation*}
	\noindent for $a^{ij} \in L^\infty(\Omega)$ symmetric. Then: Given $f \in L^2(\Omega)$, the homogenous Dirichlet problem
	\begin{align}
		\label{eq:HDP}
		\ccases{
			-Lu = f & \text{in } \Omega,\\
			u = 0 & \text{on } \partial\Omega
		}
	\end{align}
	\noindent admits a unique weak solution $u \in H^1_0(\Omega)$.
\end{theorem}

\begin{proof}
	~
	\begin{enumerate}[label = \textit{Step \arabic*:},wide=0pt]
		\item \textit{Derivation of Weak Formulation.} Suppose $u \in C^2(\wbar{\Omega})$ is a solution of (\ref{eq:HDP}). Let $\varphi \in C^\infty_c(\Omega)$. Then integration by parts (see \cite[436]{lee:smooth_manifolds:2013}) yields
			\begin{equation*}
				-\int_\Omega L(u) \varphi = - \sum_{j = 1}^n \int_\Omega \mathrm{div}(X_j)\varphi = \int_\Omega a^{ij} \frac{\partial u}{\partial x^j}\frac{\partial \varphi}{\partial x^i} = \int_\Omega a^{ij} \frac{\partial u}{\partial x^i}\frac{\partial \varphi}{\partial x^j},
			\end{equation*}
			\noindent where $X_j := \del[2]{a^{ij}\frac{\partial}{\partial x^j}}_i$. Thus we get the weak formulation:
			\begin{equation}
				\label{eq:HDPweak}
				\forall \varphi \in C^\infty_c(\Omega):\>\int_\Omega a^{ij} \frac{\partial u}{\partial x^i}\frac{\partial \varphi}{\partial x^j} = \int_\Omega f \varphi.
			\end{equation}
		\item \textit{Existence and Uniqueness of Weak Solutions.} Since $L$ is uniformly elliptic, there exists $\lambda > 0$ such that 
			\begin{equation*}
				a^{ij}(x) \xi_i \xi_j \geq \lambda \abs{\xi}^2
			\end{equation*}
			\noindent holds for any $x \in \Omega$ and $\xi \in \mathbb{R}^n$. Moreover, since $a^{ij} \in L^\infty(\Omega)$, we get that $L$ is uniformly bounded, i.e. there exists $\Lambda > 0$ such that
			\begin{equation*}
				a^{ij}(x) \xi_i \xi_j \leq \Lambda \abs{\xi}^2
			\end{equation*}
			\noindent holds for any $x \in \Omega$ and $\xi \in \mathbb{R}^n$. Now define a bilinear form $\langle \cdot,\cdot \rangle_a : H^1_0(\Omega) \times H^1_0(\Omega) \to \mathbb{R}$ by
			\begin{equation}
				\label{eq:inner_prod}
				\langle u,v \rangle_a := \int_\Omega a^{ij} \frac{\partial u}{\partial x^i}\frac{\partial v}{\partial x^j}
			\end{equation}
			Then it is easy to see, that $\langle \cdot,\cdot\rangle_a$ is symmetric. Also, $\langle \cdot,\cdot \rangle_a$ is positive definite since
			\begin{equation*}
				\langle u,u\rangle_a = \int_\Omega a^{ij} \frac{\partial u}{\partial x^i}\frac{\partial u}{\partial x^j} \geq \lambda \int_\Omega\abs{\nabla u}^2 \geq C \lambda \int_\Omega \abs{u}^2
			\end{equation*}
			\noindent using ellipticity and Poincar\'e's inequality. Moreover by Poincar\'e's inequality we have that
			\begin{equation*}
				\lambda \norm{u}^2_{H^1_0(\Omega)} \leq \norm{u}_a^2 \leq \Lambda \norm{u}_{H^1_0(\Omega)}^2
			\end{equation*}
			\noindent for the induced norm $\norm{\cdot}_a$. Hence the induced norm is equivalent to the standard norm on $H^1_0(\Omega)$ and thus $(H^1_0(\Omega),\norm{\cdot}_a)$ is a Hilbert space. Thus an application of Riesz representation theorem \ref{thm:RRT} yields the existence of a unique $u \in H^1_0(\Omega)$, such that
			\begin{equation*}
				\langle u,\varphi \rangle_a = l(\varphi) := \int_\Omega f \varphi
			\end{equation*}
			\noindent holds for all $\varphi \in H^1_0(\Omega)$, since $l \in (H^1_0(\Omega))^*$.
	\end{enumerate}
\end{proof}

\subsection*{Sobolev Spaces on an Interval}
In what follows, let $-\infty \leq a < b \leq \infty$ and $I := \intoo{a,b}$.

\begin{lemma}[Du Bois-Reymond]
	\label{lem:BoisReymond}
	Let $f \in L^1_{\mathrm{loc}}(I)$ such that
	\begin{equation*}
		\forall \varphi \in C^\infty_c(I): \> \int_I f\varphi' dx = 0.
	\end{equation*}
	Then $f$ is almost everywehere constant.
\end{lemma}

\begin{proof}
	Let $v := w - c_0 \psi$ for $w,\psi \in C^\infty_c(I)$ such that $\int_I \psi = 1$ and $\int_I v = 0$. This implies $c_0 = \int_I w$. By the fundamental theorem of calculus, the function $\varphi : I \to \mathbb{R}$ defined by
	\begin{equation*}
		\varphi(x) := \int_I v(t) dt
	\end{equation*}
	\noindent belongs to $C^\infty_c(I)$ with $\varphi' = v$. Thus we compute
	\begin{align*}
		0 = \int_I f \varphi' = \int_I f v = \int_I fw - c_0 \int_I f \psi = \int_I fw - \int_I w \int_I f \psi = \int_I (f - c)w,
	\end{align*}
	\noindent where $c := \int_I f \psi$. Since $w$ was arbitrary, we conclude by the fundamental lemma of variational calculus \ref{thm:flvc}.
\end{proof}

\begin{lemma}
	\label{lem:abscontint}
	Let $f \in L^1_{\mathrm{loc}}(I)$ and $x_0 \in I$. Then $u : I \to \mathbb{R}$ defined by
	\begin{equation*}
		u(x) := \int_{x_0}^x f(t)dt
	\end{equation*}
	\noindent is absolutely continuous and belongs to $W^{1,1}_{\mathrm{loc}}(I)$ with $u' = f$ a.e.
\end{lemma}

\begin{proof}
	Absolute continuity follows from real analysis. Let $\varphi \in C^\infty_c(I)$. Then Fubini yields
	\begin{align*}
		\int_I u\varphi' &= \int_a^{x_0}\int_{x_0}^x f(t)\varphi'(x)dtdx + \int_{x_0}^b \int_{x_0}^x f(t)\varphi'(x)dtdx\\
		&= -\int_a^{x_0}\int_x^{x_0} f(t)\varphi'(x)dtdx + \int_{x_0}^b \int_{x_0}^x f(t)\varphi'(x)dtdx\\
		&= -\int_a^{x_0}\int_a^t f(t)\varphi'(x)dxdt + \int_{x_0}^b \int_t^b f(t)\varphi'(x)dxdt\\
		&= -\int_a^{x_0} f(t)\varphi(t)dt - \int_{x_0}^b f(t)\varphi(t)dt\\
		&= -\int_I f \varphi.
	\end{align*}
\end{proof}

\begin{theorem}
	\label{thm:W1p}
	Let $u \in W^{1,p}(I)$. Then there exists an absolutely continuous representant $\wtilde{u}$ of $u$ on $\wbar{I}$, such that
	\begin{equation*}
		\wtilde{u}(x) = \wtilde{u}(x_0) + \int_{x_0}^x u'(t)dt
	\end{equation*}
	\noindent holds for all $x,x_0 \in I$.
\end{theorem}

\begin{proof}
	By lemma \ref{lem:abscontint}, the function $v(x) := \int_{x_0}^x u'(t)dt$ is in $W^{1,1}_{\mathrm{loc}}(I)$ with weak derivative $u'$. Moreover, for any $\varphi \in C^\infty_c(I)$ we compute
	\begin{equation*}
		\int_I (u - v)\varphi' = \int_I u\varphi' - \int_I v\varphi' = -\int_I u'\varphi + \int_I u'\varphi = 0.
	\end{equation*}
	Thus lemma \ref{lem:BoisReymond} yields $u = c + v$, for some $c \in \mathbb{R}$. But $c = u(x_0)$ and we conclude by setting
	\begin{equation*}
		\wtilde{u}(x) := u(x_0) + \int_{x_0}^x u'(t) dt.
	\end{equation*}
\end{proof}

\begin{theorem}[Characterization of $W^{1,p}(I)$]
	Let $1 < p \leq \infty$ and $u \in L^p(I)$. Then the following statements are equivalent:
	\begin{enumerate}[label = \textup{(}\alph*\textup{)},wide = 0pt]
		\item $u \in W^{1,p}(I)$.
		\item There exists $C \geq 0$ such that
			\begin{equation*}
				\forall \varphi \in C^\infty_c(I): \> \abs[3]{\int_I u\varphi'} \leq C\norm{\varphi}_{L^q}.
			\end{equation*}
		\item There exists $C \geq 0$ such that for all $I' \subseteq \subseteq I$ and $\abs{h} < \dist(I',\partial I)$ holds
			\begin{equation*}
				\norm{\tau_hu - u}_{L^p(I')} \leq C\abs{h},
			\end{equation*}
			\noindent where $\tau_hu(x) := u(x + h)$.
	\end{enumerate}
\end{theorem}

\begin{proof}
	The implication $(a)\Rightarrow(b)$ follows immediately from H\"older's inequality. To prove $(b)\Rightarrow(a)$, we observe that $l : C^\infty_c(I) \to \mathbb{R}$ defined by
	\begin{equation*}
		L(\varphi) := \int_I u \varphi'
	\end{equation*}
	\noindent is continuous. Since $C^\infty_c(I)$ is dense in $L^q(I)$, we get that $l \in (L^q(I))^*$. Hence we find $g \in L^p$, such that $\int_I g\varphi = l(\varphi)$ and so $u' = -g$.\\
	Next we show $(a)\Rightarrow(c)$. By theorem \ref{thm:W1p}, we find an absolutely continuous representant $\wtilde{u}$ of $u$. Thus
	\begin{equation*}
		\wtilde{u}(x + h) - \wtilde{u}(x) = h\int_0^1 u'(x + th) dt
	\end{equation*}
	Hence Jensen's inequality yields
	\begin{equation*}
		\norm[0]{\tau_hu - u}_{L^p(I')} \leq \abs{h} \int_0^1 \norm[0]{u'(\cdot + th)}_{L^p(I')}dt \leq \abs{h} \norm[0]{u'}_{L^p(I)}.
	\end{equation*}
	Lastly, we prove $(c)\Rightarrow(b)$. Let $\varphi \in C^\infty_c(I)$. Then we may find $I' \subseteq \subseteq I$ such that $\supp \varphi\subseteq I'$. Hence we compute
	\begin{align*}
		\abs[3]{\int_I u\varphi'} &= \lim_{h \to 0}\frac{1}{\abs{h}} \abs[3]{\int_I u(x) \del[1]{\varphi(x + h) - \varphi(x)} dx}\\
		&= \lim_{h \to 0}\frac{1}{\abs{h}} \abs[3]{\int_I \del[1]{u(x - h) - u(x)} \varphi(x)dx}\\
		&= \lim_{h \to 0}\frac{1}{\abs{h}} \abs[3]{\int_I \del[0]{\tau_{-h}u - u}\varphi}\\
		&\leq \lim_{h \to 0}\frac{1}{\abs{h}}\norm[0]{\tau_{-h}u - u}_{L^p(I')}\norm{\varphi}_{L^q(I)}\\
		&\leq C\norm{\varphi}_{L^q(I)}.
	\end{align*}
\end{proof}

\begin{theorem}[Extension Theorem]
	\label{thm:extension}
	There exists a continuous linear operator 
	\begin{equation*}
		E : W^{1,p}(I) \to W^{1,p}(\mathbb{R})
	\end{equation*}
	\noindent such that:
	\begin{enumerate}[label = \textup{(}\roman*\textup{)},wide = 0pt]
		\item $Eu\vert_I = u$.
		\item $\norm{Eu}_{L^p(\mathbb{R})} \leq C \norm{u}_{L^p(I)}$.
		\item $\norm[0]{(Eu)'}_{L^p(\mathbb{R})} \leq C\norm{u}_{W^{1,p}(I)}$.
	\end{enumerate}
\end{theorem}

\begin{proof}
	First we consider the case $I = \intoo{0,\infty}$. We extend $u$ by continuity to $0$ and then we extend $u$ by means of even symmetry. If $I$ is bounded we can without loss of generality assume that $I = \intoo{0,1}$. Now use a cut-off function.
\end{proof}

\begin{theorem}[Approximation Theorem]
	Let $1 \leq p < \infty$ and $u \in W^{1,p}(I)$. Then there exists a sequence $(u_i)_{i \in \omega}$ in $C^\infty_c(\mathbb{R})$ such that
	\begin{equation*}
		\norm[0]{u_i\vert_I - u}_{W^{1,p}(I)} \to 0.
	\end{equation*}
\end{theorem}

\begin{proof}
	The main idea of the proof is to use convolutions. Moreover, it is enough to consider the case $I = \mathbb{R}$ only, due to the extension theorem \ref{thm:extension}. 	
\end{proof}

\begin{theorem}[Sobolev Embedding]
	\label{thm:Sobolev_embedding}
	There is a continuous embedding 
	\begin{equation*}
			W^{1,p}(I) \hookrightarrow L^\infty(I).
	\end{equation*}
\end{theorem}

\begin{proof}
	Without loss of generality, consider $\abs{I} \leq 1$. By theorem \ref{thm:W1p} we get that
	\begin{equation*}
		\norm{u}_{L^\infty} = \sup_{x \in I} \abs[0]{u(x)} \leq \abs{u(y)} + \sup_{x \in I} \abs[3]{\int_y^x u'(t) dt} \leq \abs{u(y)} + \norm[0]{u'}_{L^1}, 
	\end{equation*}
	\noindent for any $y \in I$. Hence
	\begin{equation*}
		\norm{u}_{L^\infty} \leq \inf_{y \in I} \abs{u(y)} + \norm[0]{u'}_{L^1} \leq \frac{1}{\abs{I}}\int_I \abs{u(y)} + \norm[0]{u'}_{L^1} \leq C \norm{u}_{W^{1,1}} \leq C \norm{u}_{W^{1,p}}.
	\end{equation*}
\end{proof}

\begin{corollary}
	Let $I$ be unbounded and $u \in W^{1,p}(I)$ for $1 \leq p < \infty$. Then $u \to 0$ as $\abs{x} \to \infty$.
\end{corollary}

\subsection*{Dirichlet and Neumann Boundary Problems on $I$}
In what follows, let us consider $-\infty < a < b < \infty$ and $I := \intoo{a,b}$.

\begin{proposition}
	Let $f \in C^0(\wbar{I})$. Then the weak solution $u$ of the homogenous Dirichlet problem
	\begin{align*}
		\ccases{
			-u'' = f & \text{in }I,\\
			u(a) = 0 = u(b).
		}
	\end{align*}
	\noindent is a classical solution, i.e. $u \in C^2(\wbar{I})$.
\end{proposition}

\begin{proof}
	
\end{proof}

\begin{proposition}
	Let $f \in C^0(\wbar{I})$. Then the weak solution $u$ of the homogenous Neumann problem
	\begin{align*}
		\ccases{
			-u'' + u = f & \text{in }I,\\
			u'(a) = 0 = u'(b).
		}
	\end{align*}
	\noindent is a classical solution, i.e. $u \in C^2(\wbar{I})$.
\end{proposition}

\begin{proof}
	
\end{proof}

\subsection*{Sobolev Spaces on a Domain}

\begin{example}[Vanishing $W^{1,p}$-Capacity]
	For $n \in \omega$, $n > 1$, and $1 \leq p \leq n$, the set $\cbr{0}$ has vanishing $W^{1,p}$-capacity.
\end{example}

\begin{theorem}[Meyers-Serrin]
	Let $\Omega \subseteq \mathbb{R}^n$ be open. Then $C^\infty(\Omega) \cap W^{1,p}(\Omega)$ is dense in $W^{1,p}(\Omega)$ for every $1 \leq p < \infty$.
\end{theorem}

\begin{proof}
	Convolutions and a partition of unity argument.
\end{proof}

\begin{theorem}[Characterization of $W^{1,p}(\Omega)$]
	Let $1 < p \leq \infty$ and $u \in L^p(\Omega)$. Then the following statements are equivalent:
	\begin{enumerate}[label = \textup{(}\alph*\textup{)},wide = 0pt]
		\item $u \in W^{1,p}(\Omega)$.
		\item There exists $C \geq 0$ such that
			\begin{equation*}
				\forall \abs{\alpha} \leq 1\forall \varphi \in C^\infty_c(\Omega): \> \abs[3]{\int_I uD^\alpha \varphi} \leq C\norm{\varphi}_{L^q}.
			\end{equation*}
		\item There exists $C \geq 0$ such that for all $\Omega' \subseteq \subseteq \Omega$ and $\abs{h} < \dist(I',\partial I)$ holds
			\begin{equation*}
				\norm{\tau_hu - u}_{L^p(\Omega')} \leq C\abs{h},
			\end{equation*}
			\noindent where $\tau_hu(x) := u(x + h)$.
	\end{enumerate}
\end{theorem}

\begin{proof}
	The proof $(c)\Rightarrow(b)\Rightarrow(a)$ is almost the same as the one given in the characterization theorem for $\Omega$ an interval. For proving $(a)\Rightarrow(c)$, use Meyers-Serrin.
\end{proof}

\begin{corollary}
	Let $u \in L^\infty(\Omega)$. Then $u \in W^{1,\infty}(\Omega)$ if and only if $u$ admits a locally Lipschitz continuous representant. Moreover, if $\Omega$ is convex, then $u \in W^{1,\infty}(\Omega)$ if and only if $u$ admits a Lipschitz continuous representant. 
\end{corollary}

\subsection*{Extension and Trace Operator}

We start off with \emph{local theory}. In what follows, define
\begin{equation*}
	Q := \cbr[0]{(x',x_n) \in \mathbb{R}^{n - 1} \times \mathbb{R} : \abs[0]{x'} < 1, \abs{x_n} < 1}.
\end{equation*}
\noindent Moreover
\begin{equation*}
	Q_+ := \cbr{(x',x_n) \in Q : x_n > 0} \qquad \text{and} \qquad Q_0 := \cbr{(x',x_n) \in Q : x_n = 0}.
\end{equation*}

\begin{lemma}
	Let $u \in W^{1,p}(Q_+)$. Set
	\begin{align*}
		u^*(x',x_n) := \ccases{
			u(x',x_n) & x_n > 0,\\
			u(x',-x_n) & x_n < 0.
		}
	\end{align*}
	Then $u^* \in W^{1,p}(Q)$ and $\norm[0]{u^*}_{W^{1,p}(Q)} \leq C \norm{u}_{W^{1,p}(Q_+)}$.
\end{lemma}

Now to the \emph{global theory}.

\begin{theorem}[Extension]
	Let $\Omega \subseteq\subseteq \mathbb{R}^n$ of class $C^1$. Then there exists a continuous linear operator
	\begin{equation*}
		E : W^{1,p}(\Omega) \to W^{1,p}(\mathbb{R}^n)
	\end{equation*}
	\noindent such that:
	\begin{enumerate}[label = \textup{(}\roman*\textup{)},wide = 0pt]
		\item $Eu\vert_\Omega = u$.
		\item $\norm{Eu}_{L^p(\mathbb{R}^n)} \leq C \norm{u}_{L^p(\Omega)}$.
		\item $\norm[0]{Eu}_{W^{1,p}(\mathbb{R}^n)} \leq C\norm{u}_{W^{1,p}(\Omega)}$.
	\end{enumerate}
\end{theorem}

\begin{corollary}
	Let $\Omega \subseteq \subseteq \mathbb{R}^n$ of class $C^1$ and $1 \leq p < \infty$. Then $C^\infty(\wbar{\Omega})$ is dense in $W^{1,p}(\Omega)$.
\end{corollary}

Again, we tackle first the \emph{local theory}.

\begin{lemma}
	Let $u \in W^{1,p}(Q_+)$. Then $u\vert_{Q_0} \in L^p(Q_0)$ is well defined and the induced trace operator $W^{1,p}(Q_+) \to L^p(Q_0)$ is linear and continuous.
\end{lemma}

\begin{proof}
	We consider the case \underline{$1 \leq p < \infty$}. The main idea is to show this for $u \in C^\infty(Q)$, then for $u \in W^{1,p}(Q)$ and then finally for $u \in W^{1,p}(Q_+)$ by extension.\\
	Consider now \underline{$p = \infty$}. Since $Q_+$ is convex, $u \in W^{1,\infty}(Q_+)$ admits a Lipschitz continuous representant and the result follows by extending via continuity.
\end{proof}

\begin{theorem}[Characterization of $H^1(\Omega)$]
	Let $\Omega \subseteq \subseteq \mathbb{R}^n$ of class $C^1$. Then
	\begin{equation*}
		H^1(\Omega) = H^1_0(\Omega) \oplus \cbr[0]{u \in H^1(\Omega) : \Delta u = 0}.
	\end{equation*}
\end{theorem}

\begin{corollary}[Characterization of $H^1_0(\Omega)$]
	Let $\subseteq \subseteq \mathbb{R}^n$. Then
	\begin{equation*}
		H^1_0(\Omega) = \cbr[0]{u \in H^1(\Omega) : u\vert_{\partial \Omega} = 0}.
	\end{equation*}
\end{corollary}

\subsection*{Sobolev Embeddings}
\subsubsection*{$p < n$}

\begin{theorem}[Sobolev-Gagliardo-Nirenberg]
	Let $1 \leq p < n$ and let $p^* := \frac{np}{n - p}$. Then $W^{1,p}(\mathbb{R}^n) \hookrightarrow L^{p^*}(\mathbb{R}^n)$ with
	\begin{equation*}
		\norm{u}_{L^{p^*}} \leq C \norm[0]{\nabla u}_{L^p}.
	\end{equation*}
\end{theorem}

\begin{theorem}[Sobolev-Gagliardo-Nirenberg Compactness]
	Let $\Omega \subseteq \subseteq \mathbb{R}^n$ and $1 \leq p < n$. Then $W^{1,p}(\Omega) \hookrightarrow L^q(\Omega)$ for $1 \leq q \leq p^*$ and the embedding is compact if $q < p^*$.
\end{theorem}

\subsubsection*{$p > n$}

\begin{theorem}
	Let $p > n$. Then $W^{1,p}(\mathbb{R}^n) \hookrightarrow C^{0,\alpha}(\mathbb{R}^n)$ with $\alpha := 1 - \frac{n}{p}$ and
	\begin{equation*}
		\norm[0]{u}_{C^{0,\alpha}(\mathbb{R}^n)} \leq \norm{u}_{W^{1,p}(\Omega)}.
	\end{equation*}
\end{theorem}

\begin{remark}
	For $p = \infty$, the statement is trivially true, since any function in $W^{1,\infty}(\mathbb{R}^n)$ is Lipschitz continuous since $\mathbb{R}^n$ is convex, and thus belongs to $C^{0,1}(\mathbb{R}^n)$.
\end{remark}

The proof uses the notion of so-called \emph{Morrey-Campanato spaces}. 

\begin{theorem}
	Let $\Omega \subseteq \subseteq \mathbb{R}^n$ of type $A$ for some $A > 0$ and $1 \leq p < \infty$, $\lambda > n$, $\alpha := \frac{\lambda - n}{p}$. Then
	\begin{equation*}
		\mathcal{L}^{p,\lambda}(\Omega) \cong C^{0,\alpha}(\wbar{\Omega}).
	\end{equation*}
\end{theorem}

\begin{proof}
	The inclusion $\mathcal{L}^{p,\lambda}(\Omega) \cong C^{0,\alpha}(\wbar{\Omega})$ follows from the Campanato-theorem and does also hold for general $\Omega \subseteq \mathbb{R}^n$ open.
\end{proof}

\begin{theorem}[Poincar\'e-Wirtinger Inequality]
	Let $u \in W^{1,p}(\mathbb{R}^n)$, $1 \leq p < \infty$. Then for all $x_0 \in \mathbb{R}^n$ and $r > 0$ we have that
	\begin{equation*}
		\norm[0]{u - u_{x_0,r}}_{L^p(B_r(x_0))}^p \leq Cr^p \norm[0]{\nabla u}^p_{L^p(B_r(x_0))}.
	\end{equation*}
\end{theorem}

\begin{proof}
	
\end{proof}<++>

Now the proof of the Sobolev embedding theorem for $p > n$ is immediaty by considering
\begin{equation*}
	\begin{tikzcd}
		W^{1,p}(\mathbb{R}^n) \arrow[r,"\text{P.W.}",hook] & \mathcal{L}^{p,p}(\mathbb{R}^n) \arrow[r,"\text{Campanato}",hook] & C^{0,\alpha}(\mathbb{R}^n)
	\end{tikzcd}
\end{equation*}
\noindent and observing that $\mathbb{R}^n$ is of type $\frac{\pi^{n/2}}{\Gamma(n/2 + 1)} > 0$.

\printbibliography
\end{document}
