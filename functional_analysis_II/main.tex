%%%%%%%%%%%%%%%%%%%%%%%%%%%%%%%%%%%%%%%%%%%%%%%%%%%%%%%%%%%%%%%%%%%%%%%%%%
%Author:																 %
%-------																 %
%Yannis Baehni at University of Zurich									 %
%baehni.yannis@uzh.ch													 %
%																		 %
%Version log:															 %
%------------															 %
%06/02/16 . Basic structure												 %
%04/08/16 . Layout changes including section, contents, abstract.		 %
%%%%%%%%%%%%%%%%%%%%%%%%%%%%%%%%%%%%%%%%%%%%%%%%%%%%%%%%%%%%%%%%%%%%%%%%%%

%Page Setup
\documentclass[
	12pt, 
	oneside, 
	a4paper,
	reqno,
	final
]{amsart}

\usepackage[
	left = 3cm, 
	right = 3cm, 
	top = 3cm, 
	bottom = 3cm
]{geometry}

\newcommand\hmmax{0}
\newcommand\bmmax{0}

%Headers and footers
\usepackage{fancyhdr}
	\pagestyle{fancy}
	%Clear fields
	\fancyhf{}
	%Header right
	\fancyhead[R]{
		\footnotesize
		Yannis B\"{a}hni\\
		\href{mailto:yannis.baehni@uzh.ch}{yannis.baehni@uzh.ch}
	}
	%Header left
	\fancyhead[L]{
		\footnotesize
		MAT579:	Lie groups and Lie algebras\\
		Spring 2018
	}
	%Page numbering in footer
	\fancyfoot[C]{\thepage}
	%Separation line header and footer
	\renewcommand{\headrulewidth}{0.4pt}
	%\renewcommand{\footrulewidth}{0.4pt}
	
	\setlength{\headheight}{19pt} 

%Title
\usepackage[foot]{amsaddr}
\usepackage{newtxtext}
\usepackage[subscriptcorrection,nofontinfo,mtpcal,mtphrb]{mtpro2}
\usepackage{mathtools}
\usepackage{bm}
\usepackage{xspace}
\makeatletter
\def\@textbottom{\vskip \z@ \@plus 1pt}
\let\@texttop\relax
\usepackage{etoolbox}
\patchcmd{\abstract}{\scshape\abstractname}{\textbf{\abstractname}}{}{}

%Section, subsection and subsubsection font
%------------------------------------------
	\renewcommand{\@secnumfont}{\bfseries}
	\renewcommand\section{\@startsection{section}{1}%
  	\z@{.7\linespacing\@plus\linespacing}{.5\linespacing}%
  	{\normalfont\boldmath\bfseries\centering}}
	\renewcommand\subsection{\@startsection{subsection}{2}%
    	\z@{.5\linespacing\@plus.7\linespacing}{-.5em}%
    	{\normalfont\bfseries}}%
	\renewcommand\subsubsection{\@startsection{subsubsection}{3}%
    	\z@{.5\linespacing\@plus.7\linespacing}{-.5em}%
    	{\normalfont\bfseries}}%
%Formatting title of TOC
\renewcommand{\contentsnamefont}{\bfseries}
%Table of Contents
\setcounter{tocdepth}{3}

% Add bold to \section titles in ToC and remove . after numbers
\renewcommand{\tocsection}[3]{%
	\indentlabel{\@ifnotempty{#2}{\bfseries\ignorespaces#1 #2\quad}}\boldmath\bfseries#3}
\renewcommand{\tocappendix}[3]{%
  	\indentlabel{\@ifnotempty{#2}{\bfseries\ignorespaces#1 #2: }}\bfseries#3}
% Remove . after numbers in \subsection
\renewcommand{\tocsubsection}[3]{%
  \indentlabel{\@ifnotempty{#2}{\ignorespaces#1 #2\quad}}#3}
%\let\tocsubsubsection\tocsubsection% Update for \subsubsection
%...

\newcommand\@dotsep{4.5}
\def\@tocline#1#2#3#4#5#6#7{\relax
  \ifnum #1>\c@tocdepth % then omit
  \else
    \par \addpenalty\@secpenalty\addvspace{#2}%
    \begingroup \hyphenpenalty\@M
    \@ifempty{#4}{%
      \@tempdima\csname r@tocindent\number#1\endcsname\relax
    }{%
      \@tempdima#4\relax
    }%
    \parindent\z@ \leftskip#3\relax \advance\leftskip\@tempdima\relax
    \rightskip\@pnumwidth plus1em \parfillskip-\@pnumwidth
    #5\leavevmode\hskip-\@tempdima{#6}\nobreak
    \leaders\hbox{$\m@th\mkern \@dotsep mu\hbox{.}\mkern \@dotsep mu$}\hfill
    \nobreak
    \hbox to\@pnumwidth{\@tocpagenum{\ifnum#1=1\bfseries\fi#7}}\par% <-- \bfseries for \section page
    \nobreak
    \endgroup
  \fi}
\AtBeginDocument{%
\expandafter\renewcommand\csname r@tocindent0\endcsname{0pt}
}
\def\l@subsection{\@tocline{2}{0pt}{2.5pc}{5pc}{}}
\def\l@subsubsection{\@tocline{2}{0pt}{4.5pc}{5pc}{}}
\makeatother

\advance\footskip0.4cm
\textheight=54pc    %a4paper
\textheight=50.5pc %letterpaper
\advance\textheight-0.4cm
\calclayout
%Font settings
%\usepackage{anyfontsize}
%Footnote settings
\usepackage{footmisc}
%	\renewcommand*{\thefootnote}{\fnsymbol{footnote}}
\usepackage{commath}
%Further math environments
%Further math fonts (loads amsfonts implicitely)
%Redefinition of \text
%\usepackage{amstext}
\usepackage{upref}
%Graphics
%\usepackage{graphicx}
%\usepackage{caption}
%\usepackage{subcaption}
%Frames
\usepackage{mdframed}
\allowdisplaybreaks
%\usepackage{interval}
\newcommand{\toup}{%
  \mathrel{\nonscript\mkern-1.2mu\mkern1.2mu{\uparrow}}%
}
\newcommand{\todown}{%
  \mathrel{\nonscript\mkern-1.2mu\mkern1.2mu{\downarrow}}%
}
\AtBeginDocument{\renewcommand*\d{\mathop{}\!\mathrm{d}}}
\renewcommand{\Re}{\operatorname{Re}}
\renewcommand{\Im}{\operatorname{Im}}
\DeclareMathOperator\Log{Log}
\DeclareMathOperator\Arg{Arg}
\DeclareMathOperator\id{id}
\DeclareMathOperator\sech{sech}
\DeclareMathOperator\Aut{Aut}
\DeclareMathOperator\h{h}
\DeclareMathOperator\sgn{sgn}
\DeclareMathOperator\arctanh{arctanh}
\DeclareMathOperator*\esssup{ess.sup}
\DeclareMathOperator\ob{ob}
\DeclareMathOperator\coker{coker}
\DeclareMathOperator\im{im}
\DeclareMathOperator\Ch{Ch}
\DeclareMathOperator\Ext{Ext}
\DeclareMathOperator\Hom{Hom}
\DeclareMathOperator\tr{tr}
%\usepackage{hhline}
%\usepackage{booktabs} 
%\usepackage{array}
%\usepackage{xfrac} 
%\everymath{\displaystyle}
%Enumerate
\usepackage{tikz-cd}
\usepackage{enumitem} 
%\renewcommand{\labelitemi}{$\bullet$}
%\renewcommand{\labelitemii}{$\ast$}
%\renewcommand{\labelitemiii}{$\cdot$}
%\renewcommand{\labelitemiv}{$\circ$}
%Colors
%\usepackage{color}
%\usepackage[cmtip, all]{xy}
%Main style theorem environment
\newtheoremstyle{main} 		             	 		%Stylename
  	{}	                                     		%Space above
  	{}	                                    		%Space below
  	{\itshape}			                     		%Body font
  	{}        	                             		%Indent
  	{\boldmath\bfseries}   	                         		%Head font
  	{.}            	                        		%Head punctuation
  	{ }           	                         		%Head space 
  	{\thmname{#1}\thmnumber{ #2}\thmnote{ (#3)}}	%Head specification
\theoremstyle{main}
\newtheorem{definition}{Definition}[section]
\newtheorem{proposition}{Proposition}[section]
\newtheorem{corollary}{Corollary}[section]
\newtheorem{theorem}{Theorem}[section]
\newtheorem{lemma}{Lemma}[section]
\newtheoremstyle{nonit} 		             	 		%Stylename
  	{}	                                     		%Space above
  	{}	                                    		%Space below
  	{}			                     		%Body font
  	{}        	                             		%Indent
	{\boldmath\bfseries}   	                         		%Head font
  	{.}            	                        		%Head punctuation
  	{ }           	                         		%Head space 
  	{\thmname{#1}\thmnumber{ #2}\thmnote{ (#3)}}	%Head specification
\theoremstyle{nonit}
\newtheorem{remark}{Remark}[section]
\newtheorem{example}{Example}[section]
\newtheorem{examples}{Examples}[section]
\newtheoremstyle{ex} 		             	 		%Stylename
  	{}	                                     		%Space above
  	{}	                                    		%Space below
  	{\small}			                     		%Body font
  	{}        	                             		%Indent
  	{\bfseries\boldmath}   	                         		%Head font
  	{.}            	                        		%Head punctuation
  	{ }           	                         		%Head space 
  	{\thmname{#1}\thmnumber{ #2}\thmnote{ (#3)}}	%Head specification
\theoremstyle{ex}
\newtheorem{exercise}[theorem]{Exercise}
%German non-ASCII-Characters
%Graphics-Tool
%\usepackage{tikz}
%\usepackage{tikzscale}
%\usepackage{bbm}
%\usepackage{bera}
%Listing-Setup
%Bibliographie
\usepackage[backend=bibtex, style=alphabetic]{biblatex}
%\usepackage[babel, german = swiss]{csquotes}
\bibliography{bibliography}
%PDF-Linking
%\usepackage[hyphens]{url}
\usepackage[bookmarksopen=true,bookmarksnumbered=true]{hyperref}
%\PassOptionsToPackage{hyphens}{url}\usepackage{hyperref}
\urlstyle{rm}
\hypersetup{
  colorlinks   = true, %Colours links instead of ugly boxes
  urlcolor     = blue, %Colour for external hyperlinks
  linkcolor    = blue, %Colour of internal links
  citecolor    = blue %Colour of citations
}
\newcommand{\bld}[1]{\boldmath\textit{\textbf{#1}}\unboldmath}


\setcounter{section}{1}

\title{Functional Analysis II Summary}
\author{Yannis B\"{a}hni}
\address[Yannis B\"{a}hni]{University of Zurich, R\"{a}mistrasse 71, 8006 Zurich}
\email[Yannis B\"{a}hni]{\href{mailto:yannis.baehni@uzh.ch}{\nolinkurl{yannis.baehni@uzh.ch}}}

\begin{document}

\begin{abstract}

\end{abstract}

\maketitle

\tableofcontents

\section*{Elliptic Operators in Divergence Form}

\begin{lemma}[Poincar\'e Inequality]
	\label{lem:PI}
\end{lemma}

\begin{theorem}[Riesz Representation Theorem]
	\label{thm:RRT}

\end{theorem}

\begin{theorem}
	Let $\Omega \subseteq\subseteq \mathbb{R}^n$, $k \in \omega$ and consider the elliptic operator
	\begin{equation*}
		L := \sum_{i,j = 1}^n \frac{\partial}{\partial x^i}\del[3]{a_{ij}\frac{\partial}{\partial x^j}},
	\end{equation*}
	\noindent for $a_{ij} \in C^{k + 1}(\wbar{\Omega})$ symmetric. Then:
	\begin{enumerate}[label = \textup{(}\alph*\textup{)},wide=0pt]
		\item Given $f \in L^2(\Omega)$, the homogenous Dirichlet problem
			\begin{align}
				\label{eq:HDP}
				\ccases{
					-L(u) = f & \text{in } \Omega,\\
					u = 0 & \text{on } \partial\Omega
				}
			\end{align}
			\noindent admits a unique weak solution $u \in H^1_0(\Omega)$.
		\item If $f \in H^k(\Omega)$ for some $k \in \omega$, then we have $u \in H^{k + 2}_{\mathrm{loc}}(\Omega)$ for the unique weak solution of part (a) and moreover, for any $\Omega' \subseteq \subseteq \Omega$ we have the estimate
			\begin{equation*}
				\norm{u}_{H^{k + 2}(\Omega')} \leq C\del[1]{\norm{f}_{H^k(\Omega)} + \norm{u}_{H^1(\Omega)}}.
			\end{equation*}
	\end{enumerate}
\end{theorem}

\begin{proof}
	~
	\begin{enumerate}[label = \textit{Step \arabic*:},wide=0pt]
		\item \textit{Derivation of Weak Formulation.} Suppose $u \in C^2(\wbar{\Omega})$ is a solution of (\ref{eq:HDP}). Let $\varphi \in C^\infty_c(\Omega)$. Then integration by parts (see \cite[436]{lee:smooth_manifolds:2013}) yields
			\begin{equation*}
				-\int_\Omega L(u) \varphi = - \sum_{j = 1}^n \int_\Omega \mathrm{div}(X_j)\varphi = \sum_{i,j = 1}^n \int_\Omega a_{ij} \frac{\partial u}{\partial x^j}\frac{\partial \varphi}{\partial x^i} = \sum_{i,j = 1}^n \int_\Omega a_{ij} \frac{\partial u}{\partial x^i}\frac{\partial \varphi}{\partial x^j},
			\end{equation*}
			\noindent where $X_j := \del[2]{a_{ij}\frac{\partial}{\partial x^j}}_i$. Thus we get the weak formulation:
			\begin{equation}
				\label{eq:HDPweak}
				\sum_{i,j = 1}^n \int_\Omega a_{ij} \frac{\partial u}{\partial x^i}\frac{\partial \varphi}{\partial x^j} = \int_\Omega f \varphi \qquad \forall \varphi \in C^\infty_c(\Omega).
			\end{equation}
		\item \textit{Existence and Uniqueness of Weak Solutions.} Since $L$ is uniformly elliptic, there exists $\lambda > 0$ such that 
			\begin{equation*}
				\sum_{i,j = 1}^n a_{ij}(x) \xi_i \xi_j \geq \lambda \abs{\xi}^2
			\end{equation*}
			\noindent holds for any $x \in \Omega$ and $\xi \in \mathbb{R}^n$. Moreover, since $a_{ij} \in C^0(\wbar{\Omega})$, we get that $L$ is uniformly bounded, i.e. there exists $\Lambda > 0$ such that
			\begin{equation*}
				\sum_{i,j = 1}^n a_{ij}(x) \xi_i \xi_j \leq \Lambda \abs{\xi}^2
			\end{equation*}
			\noindent holds for any $x \in \Omega$ and $\xi \in \mathbb{R}^n$. Now define a bilinear form $\langle \cdot,\cdot \rangle_a : H^1_0(\Omega) \times H^1_0(\Omega) \to \mathbb{R}$ by
			\begin{equation}
				\label{eq:inner_prod}
				\langle u,v \rangle_a := \sum_{i,j = 1}^n \int_\Omega a_{ij} \frac{\partial u}{\partial x^i}\frac{\partial v}{\partial x^j}
			\end{equation}
			Then it is easy to see, that $\langle \cdot,\cdot\rangle_a$ is symmetric. Also, $\langle \cdot,\cdot \rangle_a$ is positive definite since
			\begin{equation*}
				\langle u,u\rangle_a = \sum_{i,j = 1}^n \int_\Omega a_{ij} \frac{\partial u}{\partial x^i}\frac{\partial u}{\partial x^j} \geq \lambda \int_\Omega\abs{\nabla u}^2 \geq C \lambda \int_\Omega \abs{u}^2
			\end{equation*}
			\noindent using ellipticity and Poincar\'e's inequality. Moreover by Poincar\'e's inequality we have that
			\begin{equation*}
				C \lambda \norm{u}^2_{H^1_0(\Omega)} \leq \norm{u}_a \leq \Lambda \norm{u}_{H^1_0(\Omega)}^2
			\end{equation*}
			\noindent for the induced norm $\norm{\cdot}_a$. Hence the induced norm is equivalent to the standard norm on $H^1_0(\Omega)$ and thus $(H^1_0(\Omega),\norm{\cdot}_a)$ is a Hilbert space. Thus an application of Riesz representation theorem \ref{thm:RRT} yields the existence of a unique $u \in H^1_0(\Omega)$, such that
			\begin{equation*}
				\langle u,\varphi \rangle_a = l(\varphi) := \int_\Omega f \varphi
			\end{equation*}
			\noindent holds for all $\varphi \in H^1_0(\Omega)$, since $l \in (H^1_0(\Omega))^*$ This proves part (a).
		\item \textit{$H^1$-Estimate.} The main idea in proving part (b) is an induction on $k \in \omega$. So let us assume that $k = 0$. Let $u \in H^1_0(\Omega)$ denote the unique solution of part (a).
			\begin{lemma}
				\label{lem:H^1_estimate}
				\begin{equation*}
					\norm{\nabla u}_{L^2(\Omega')}
				\end{equation*}
			\end{lemma}
	\end{enumerate}
\end{proof}

\printbibliography
\end{document}
