\input{header.tex}

\setcounter{section}{1}

\title{Functional Analysis II Summary}
\author{Yannis B\"{a}hni}
\address[Yannis B\"{a}hni]{University of Zurich, R\"{a}mistrasse 71, 8006 Zurich}
\email[Yannis B\"{a}hni]{\href{mailto:yannis.baehni@uzh.ch}{\nolinkurl{yannis.baehni@uzh.ch}}}

\begin{document}

\begin{abstract}
	This is a rough summary of the course \emph{Functional Analysis II} held at \emph{ETH Zurich} by \emph{Prof. Dr. Alessandro Carlotto} in spring $2018$. The main focus of this summary is to give a neat preparation for the oral exam.
\end{abstract}

\maketitle

\tableofcontents

\section*{Introduction}
This serves as a summary of useful facts from \emph{measure theory} which are used throughout the text.
\begin{theorem}[Transformation Formula]
	Let $n \in \omega$, $n > 0$, $U,V \subseteq \mathbb{R}^n$ open and $\varphi : U \to V$ a $C^1$-diffeomorphism. A function $f : V \to \wbar{\mathbb{R}}$ is in $\mathcal{L}^1(V)$ if and only if $(f \circ \varphi) \abs[0]{\det (D\varphi)}$ is in $\mathcal{L}^1(U)$. Then
	\begin{equation*}
		\int_V f = \int_U (f \circ \varphi) \abs[0]{\det (D\varphi)}.
	\end{equation*}
\end{theorem}

\begin{theorem}
	Let $\Omega \subseteq \mathbb{R}^n$ be open and $1 \leq p < \infty$. Then $C^\infty_c(\Omega)$ is dense in $L^p(\Omega)$.
\end{theorem}

\begin{proposition}
	\label{prop:inclusion}
	If $\abs[0]{\Omega} < \infty$ and $0 < p < q \leq \infty$. Then $L^q(\Omega) \subseteq L^p(\Omega)$.
\end{proposition}

\begin{proposition}[Jensen's Inequality]
	\label{prop:Jensen}
	Let $\Omega \subseteq \mathbb{R}^n$ bounded and $\varphi : \mathbb{R} \to \mathbb{R}$ convex. Then
	\begin{equation*}
		\varphi\del[3]{\frac{1}{\abs{\Omega}}\int_\Omega f} \leq \frac{1}{\abs{\Omega}} \int_\Omega \varphi \circ f
	\end{equation*}
	\noindent for any $f \in L^1(\Omega)$.
\end{proposition}

\begin{proposition}[Dual of $L^p(\Omega)$]
	Let $\Omega \subseteq \mathbb{R}^n$ and $1 \leq p < \infty$. Then the mapping $T : L^q(\Omega) \to (L^p(\Omega))^*$ defined by
	\begin{equation*}
		T(f)(g) := \int_\Omega fg
	\end{equation*}
	\noindent is an isometric isomorphism.
\end{proposition}

\begin{proposition}[Integration by Parts]
	\label{prop:IP}
	Let $(M,g)$ be a compact Riemannian manifold with boundary. Then
	\begin{equation*}
		\int_M \langle \grad f,X \rangle_g dV_g = \int_{\partial M}f \langle X,N \rangle dV_{\wtilde{g}} - \int_M (f \div X) dV_g
	\end{equation*}
	\noindent for $f \in C^\infty(M)$ and $X \in \mathfrak{X}(M)$. Moreover, \bld{Green's identities} hold:
	\begin{equation*}
		\int_M u\Delta v dV_g = \int_M \langle \grad u,\grad v\rangle_g dV_g - \int_{\partial M} uNv dV_{\wtilde{g}}
	\end{equation*}
	\noindent and
	\begin{equation*}
		\int_M (u\Delta v - v \Delta u)dV_g = \int_{\partial M}(vNu - u Nv)dV_{\wtilde{g}}
	\end{equation*}
	\noindent for $u,v \in C^\infty(M)$.
\end{proposition}

\section*{Sobolev Space Theory}
\subsection*{The Spaces $W^{k,p}(\Omega)$}
In what follows, let $n \in \omega$, $n \geq 1$, and $1 \leq p \leq \infty$.

\begin{definition}[Distributional and Weak Derivative]
	Let $\Omega \subseteq \mathbb{R}^n$ open and $u \in L^1_{\mathrm{loc}}(\Omega)$. For any multiindex $\alpha$, the \bld{distributional derivative of order $\alpha$ of $u$}, written $D^\alpha u$, is defined to be the mapping $D^\alpha u : C^\infty_c(\Omega) \to \mathbb{R}$ defined by
	\begin{equation*}
		\varphi \mapsto (-1)^{\abs{\alpha}}\int_\Omega u D^\alpha \varphi dx.
	\end{equation*}
	Moreover, a function $D^\alpha u \in L^p(\Omega)$ is called \bld{weak derivative of order $\alpha$ of $u$ with exponent $p$}, iff
	\begin{equation*}
		\forall \varphi \in C^\infty_c(\Omega): \> \int_\Omega D^\alpha u \varphi dx =  (-1)^{\abs{\alpha}} \int_\Omega u D^\alpha \varphi dx.
	\end{equation*}
\end{definition}

\begin{theorem}[Fundamental Lemma of Variational Calculus]
	\label{thm:flvc}
	Let $\Omega \subseteq \mathbb{R}^n$ open and $f \in L^1_{\mathrm{loc}}(\Omega)$. If
	\begin{equation*}
		\forall \varphi \in C^\infty_c(\Omega) : \> \int_\Omega f \varphi dx = 0,
	\end{equation*}
	\noindent then $f = 0$ a.e.
\end{theorem}

\begin{remark}
	Let $\Omega \subseteq \mathbb{R}^n$ open. Then $L^p(\Omega) \subseteq L^1_{\mathrm{loc}}(\Omega)$.
\end{remark}

\begin{remark}
	From the fundamental lemma of variational calculus \ref{thm:flvc} it follows that \emph{weak derivatives, if they exist, are unique}.
\end{remark}

\begin{examples}[Weak Derivatives]
	~
	\begin{enumerate}[label = \textup{(}\alph*\textup{)},wide = 0pt]
		\item Suppose $u$ is classically differentiable. Then $u$ is weakly differentiable using integration by parts \ref{prop:IP}.
		\item Consider $\Omega := \intoo{-1,1}$ and $u := \abs{x}$. Then $u' = \chi_{\intco{0,1}} -\chi_{\intoo{-1,0}}$.
		\item Consider $\Omega := \mathbb{R}$ and $u := \chi_{\intoo{0,\infty}}$. Then the weak derivative $u'$ does not exist. Indeed, the \emph{Dirac distribution} is not representable as one may see by considering the smooth family $\varphi_\varepsilon : \mathbb{R} \to \mathbb{R}$ for $\varepsilon > 0$ defined by
			\begin{equation*}
				\varphi_\varepsilon(x) := \begin{cases}
					e^{\varepsilon^2/(x^2 - \varepsilon^2)} & \abs{x} < \varepsilon,\\
					0 & \abs{x} \geq \varepsilon.
				\end{cases}
			\end{equation*}
		\item Let $\Omega := \intoo{0,1}$ and consider the \emph{Cantor function} $u : \Omega \to \Omega$. Then $u' = 0$ classically a.e. but the distributional derivative of $u$ does not vanish.
		\item Let $f \in L^p(\Omega)$. Then the computation performed in the proof of lemma \ref{lem:abscontint} shows, that the function $u : I \to \mathbb{R}$ defined by
			\begin{equation*}
				u(x) := \int_{x_0}^x f(t)dt
			\end{equation*}
			\noindent for $x_0 \in I$, admits the weak derivative $f$.
	\end{enumerate}
\end{examples}

\begin{definition}[Sobolev Space]
	Let $\Omega \subseteq \mathbb{R}^n$ open. For any $k \in \omega$, the \bld{Sobolev space of index $(k,p)$}, written $W^{k,p}(\Omega)$, is defined to be the space
	\begin{equation*}
		W^{k,p}(\Omega) := \cbr[0]{f \in L^p(\Omega) : D^\alpha u \in L^p(\Omega) \text{ exists for all } \abs{\alpha} \leq k},
	\end{equation*}
	\noindent with norm
	\begin{equation*}
		\norm{-} _{W^{k,p}(\Omega)} := \sum_{\abs{\alpha} \leq k} \norm[0]{D^\alpha -}_{L^p(\Omega)}.
	\end{equation*}
	Moreover, define
	\begin{equation*}
		W^{k,p}_0(\Omega) := \overline{C^\infty_c(\Omega)}^{\norm{-}_{W^{k,p}(\Omega)}},
	\end{equation*}
	\noindent and $H^k(\Omega) := W^{k,2}(\Omega)$ as well as $H_0^k(\Omega) := W^{k,2}_0(\Omega)$.
\end{definition}

\begin{examples}[Sobolev Functions]
	A main tool in constructing Sobolev functions for $n \geq 2$ is using that the origin in $\mathbb{R}^n$ has vanishing $W^{1,p}$-capacity for $1 \leq p \leq n$.
	\begin{enumerate}[label = \textup{(}\alph*\textup{)},wide=0pt]
		\item Let $\Omega := \mathbb{R}$, $A$ Lebesgue-measurable and $u := \chi_A$. Then $u \notin W^{1,p}(\Omega)$, since by theorem \ref{thm:W1p} $u$ must admit a continuous representant, which it obviously does not.
		\item Let $\Omega := B_1(0) \subseteq \mathbb{R}^n$ for $n \geq 2$. Then $u : \Omega \to \wbar{\mathbb{R}}$ defined by $u(x) := \log\abs{x}$ belongs to $L^p(\Omega)$ for any $1 \leq p < \infty$ and moreover, $u \in W^{1,p}(\Omega)$ for any $p < n$.
		\item Let $\Omega := B_{1/e}(0) \subseteq \mathbb{R}^n$ for $n \geq 2$. Then $u : \Omega \to \wbar{\mathbb{R}}$ defined by $u(x) := \log\log\frac{1}{\abs{x}}$ belongs to $W^{1,n}(\Omega)$.
		\item Let $\Omega := B_{1/2}(0) \subseteq \mathbb{R}^n$. For $\alpha \in \mathbb{R}$ define $u_\alpha : \Omega \to \wbar{\mathbb{R}}$ by $u_\alpha(x) := \abs[0]{\log\abs{x}}^\alpha$. Then $u_\alpha \in H^1(\Omega)$ for $n = 1$ if and only if $\alpha = 0$, for $n = 2$ if and only if $\alpha \in \intoo[0]{-\infty,1/2}$ and for $n \geq 3$ if and only if $\alpha \in \mathbb{R}$.
	\end{enumerate}
\end{examples}

\begin{remark}
	Using proposition \ref{prop:inclusion}, we immediately get
	\begin{equation*}
		W^{1,q}(\Omega) \hookrightarrow W^{1,p}(\Omega)
	\end{equation*}
	\noindent for all $1 \leq p \leq q \leq \infty$ whenever $\Omega \subseteq \subseteq \mathbb{R}^n$.
\end{remark}

\begin{theorem}
	Let $\Omega \subseteq \mathbb{R}^n$ open. Then $W^{k,p}(\Omega)$ is
	\begin{enumerate}[label = \textup{(}\alph*\textup{)},wide = 0pt]
		\item a Banach space for all $1 \leq p \leq \infty$.
		\item separable for all $1 \leq p < \infty$.
		\item reflexive for all $1 < p < \infty$.
	\end{enumerate}
\end{theorem}

\begin{proof}
	The proof basically boils down to using the correponding properties of the Lebesgue spaces $L^p(\Omega)$.
	\begin{enumerate}[label = \textup{(}\alph*\textup{)},wide = 0pt]
		\item This follows from the fact that $L^p(\Omega)$ is a Banach space for all $1 \leq p \leq \infty$. Let $(f_i)_{i \in \omega}$ be a Cauchy sequence in $W^{k,p}$. By definition of the $W^{k,p}$-norm, $(D^\alpha f_i)_{i \in \omega}$ is a Cauchy sequence in $L^p$. Thus we get $D^\alpha f_i \to f_\alpha$ in $L^p$, in particular, $f_i \to f$ in $L^p$. Using H\"older's inequality we compute
			\begin{equation*}
				\int_\Omega f_\alpha \varphi dx = \lim_{i \to \infty} \int_\Omega D^\alpha f_i \varphi dx = (-1)^{\abs{\alpha}} \lim_{i \to \infty}\int_\Omega f_i D^\alpha \varphi dx = (-1)^{\abs{\alpha}}\int_\Omega f D^\alpha \varphi dx
			\end{equation*}
			\noindent for $\varphi \in C^\infty_c(\Omega)$.
		\item For simplicity, we consider $k = 1$ only. Consider $\iota : W^{1,p} \hookrightarrow (L^p)^{n + 1}$ defined in the obvious way. Then $\iota$ is an isometry and the statement follows. 
		\item Same argument as in part (b).
	\end{enumerate}
\end{proof}

\subsection*{Elliptic Operators in Divergence Form}

\begin{lemma}[Poincar\'e Inequality]
	\label{lem:PI}
	Let $\Omega \subseteq\subseteq \mathbb{R}^n$ and $1 \leq p < \infty$. Then for any $u \in C^\infty_c(\Omega)$ we have that
	\begin{equation*}
		\norm{u}_{L^p} \leq C\norm[0]{\nabla u}_{L^p}.
	\end{equation*}
\end{lemma}

\begin{proof}
	Let $n = 1$. Since $\Omega$ is bounded, we get that $\Omega \subseteq \intoo{a,b}$ and we may extend $u$ on $\intcc{a,b} =: I$ to be zero. Hence an application of Jensen's inequality \ref{prop:Jensen} yields
	\begin{equation*}
		\abs{u(x)}^p = \abs[3]{\int_a^x u'(t)dt}^p \leq (x - a)^{p - 1} \int_a^x \abs[0]{u'(t)}^pdt \leq (b - a)^{p-1} \norm[0]{u'}_{L^p(I)}^p.
	\end{equation*}
	Thus
	\begin{equation*}
		\norm[0]{u}_{L^p(\Omega)}^p = \norm{u}_{L^p(I)}^p \leq (b - a)^p \norm[0]{u'}^p_{L^p(I)} = (b - a)^p \norm[0]{u'}_{L^p(\Omega)}^p
	\end{equation*}
	\noindent where the last equality follows due to the fact that $u$ and thus $u'$ is compactly supported in $\Omega$. If $n > 1$, we have $\Omega \subseteq \intoo{a,b} \times \mathbb{R}^{n-1}$. Hence for fixed $y \in \mathbb{R}^{n - 1}$, above computation yields
	\begin{equation*}
		\abs[0]{u(x,y)}^p \leq (b - a)^{p - 1}\norm[0]{\partial_xu(-,y)}^p_{L^p(I)}
	\end{equation*}
	\noindent for any $x \in I$. Hence
	\begin{align*}
		\norm{u}_{L^p(\Omega)}^p &= \norm{u}_{L^p(\intoo{a,b} \times \mathbb{R}^{n - 1})}\\
		&\leq (b - a)^p \int_{\mathbb{R}^{n - 1}} \norm[0]{\partial_xu(-,y)}^p_{L^p(I)}dy\\
		&\leq (b - a)^p \norm[0]{\nabla u}^p_{L^p(\intoo{a,b} \times \mathbb{R}^{n - 1})}\\
		&= (b - a)^p \norm[0]{\nabla u}_{L^p(\Omega)}^p.
	\end{align*}
\end{proof}

\begin{theorem}[Riesz Representation Theorem]
	\label{thm:RRT}
	Let $H$ be a real Hilbert space. Then the mapping $J : H \to H^*$ defined by $J(x) := \langle x,-\rangle$ is an isometric isomorphism.
\end{theorem}

\begin{theorem}
	Let $\Omega \subseteq\subseteq \mathbb{R}^n$ and consider the elliptic operator
	\begin{equation*}
		A_0 := -\frac{\partial}{\partial x^i}\del[3]{a^{ij}\frac{\partial}{\partial x^j}},
	\end{equation*}
	\noindent for $a^{ij} \in L^\infty(\Omega)$ symmetric. Then: Given $f \in L^2(\Omega)$, the homogenous Dirichlet problem
	\begin{align}
		\label{eq:HDP}
		\ccases{
			A_0u = f & \text{in } \Omega,\\
			u = 0 & \text{on } \partial\Omega
		}
	\end{align}
	\noindent admits a unique weak solution $u \in H^1_0(\Omega)$.
\end{theorem}

\begin{proof}
	The proof is divided into two steps.
	\begin{enumerate}[label = \textit{Step \arabic*:},wide=0pt]
		\item \textit{Derivation of Weak Formulation.} Suppose $u \in C^2(\wbar{\Omega})$ is a solution of (\ref{eq:HDP}). Let $\varphi \in C^\infty_c(\Omega)$. Then integration by parts (see \cite[436]{lee:smooth_manifolds:2013}) yields
			\begin{equation*}
				-\int_\Omega A_0u \varphi = - \sum_{j = 1}^n \int_\Omega \mathrm{div}(X_j)\varphi = \int_\Omega a^{ij} \frac{\partial u}{\partial x^j}\frac{\partial \varphi}{\partial x^i} = \int_\Omega a^{ij} \frac{\partial u}{\partial x^i}\frac{\partial \varphi}{\partial x^j},
			\end{equation*}
			\noindent where $X_j := \del[2]{a^{ij}\frac{\partial}{\partial x^j}}_i$. Thus we get the weak formulation:
			\begin{equation}
				\label{eq:HDPweak}
				\forall \varphi \in C^\infty_c(\Omega):\>\int_\Omega a^{ij} \frac{\partial u}{\partial x^i}\frac{\partial \varphi}{\partial x^j} = \int_\Omega f \varphi.
			\end{equation}
		\item \textit{Existence and Uniqueness of Weak Solutions.} Since $A_0$ is uniformly elliptic, there exists $\lambda > 0$ such that 
			\begin{equation*}
				\xi^t(a^{ij}(x))\xi = a^{ij}(x) \xi_i \xi_j \geq \lambda \abs{\xi}^2
			\end{equation*}
			\noindent holds for any $x \in \Omega$ and $\xi \in \mathbb{R}^n$. Moreover, since $a^{ij} \in L^\infty(\Omega)$, we get that $A_0$ is uniformly bounded, i.e. there exists $\Lambda > 0$ such that
			\begin{equation*}
				a^{ij}(x) \xi_i \eta_j \leq \Lambda \abs{\xi}\abs{\eta}
			\end{equation*}
			\noindent for
			\begin{equation*}
				\Lambda = \sum_{i,j = 1}^n \norm[0]{a^{ij}}_{L^\infty(\Omega)}
			\end{equation*}
			\noindent holds for almost all $x \in \Omega$ and $\xi,\eta \in \mathbb{R}^n$. Now define a bilinear form 
			\begin{equation*}
				\langle \cdot,\cdot \rangle_a : H^1_0(\Omega) \times H^1_0(\Omega) \to \mathbb{R}
			\end{equation*}
			\noindent by
			\begin{equation}
				\label{eq:inner_prod}
				\langle u,v \rangle_a := \int_\Omega a^{ij} \frac{\partial u}{\partial x^i}\frac{\partial v}{\partial x^j}
			\end{equation}
			Then it is easy to see, that $\langle \cdot,\cdot\rangle_a$ is symmetric. Also, $\langle \cdot,\cdot \rangle_a$ is positive definite since
			\begin{equation*}
				\langle u,u\rangle_a = \int_\Omega a^{ij} \frac{\partial u}{\partial x^i}\frac{\partial u}{\partial x^j} \geq \lambda \int_\Omega\abs{\nabla u}^2 \geq C \lambda \int_\Omega \abs{u}^2
			\end{equation*}
			\noindent using ellipticity and Poincar\'e's inequality. Moreover by Poincar\'e's inequality we have that
			\begin{equation*}
				\lambda \norm{u}^2_{H^1_0(\Omega)} \leq \norm{u}_a^2 \leq \Lambda \norm{u}_{H^1_0(\Omega)}^2
			\end{equation*}
			\noindent for the induced norm $\norm{\cdot}_a$. Hence the induced norm is equivalent to the standard norm on $H^1_0(\Omega)$ and thus $(H^1_0(\Omega),\norm{\cdot}_a)$ is a Hilbert space. Thus an application of Riesz representation theorem \ref{thm:RRT} yields the existence of a unique $u \in H^1_0(\Omega)$, such that
			\begin{equation*}
				\langle u,\varphi \rangle_a = l(\varphi) := \int_\Omega f \varphi
			\end{equation*}
			\noindent holds for all $\varphi \in H^1_0(\Omega)$, since $l \in (H^1_0(\Omega))^*$ by
			\begin{equation*}
				\abs{l(\varphi)} \leq \norm{f}_{L^2}\norm{\varphi}_{L^2} \leq \norm{f}_{L^2}\norm{\varphi}_{H^1}.
			\end{equation*}
	\end{enumerate}
\end{proof}

\begin{examples}[Elliptic Operators in Divergence Form]
	~
	\begin{enumerate}[label = \textup{(}\alph*\textup{)},wide = 0pt]
		\item Set $a^{ij}(x) := \delta^{ij}$ for all $x \in \Omega$. Then $A_0 = - \Delta$. Moroever $A_0$ is uniformly elliptic, since $\delta^{ij}\xi_i\xi_j = \abs{\xi}^2$ for all $\xi \in \mathbb{R}^n$.
		\item For $\Omega \subseteq \mathbb{R}^2$ consider
			\begin{equation*}
				(a^{ij}(x,y)) := \begin{pmatrix}
					2 & xy/\abs{xy}\\
					xy/\abs{xy} & 2
				\end{pmatrix}.
			\end{equation*}
			Then $A_0$ is elliptic. Indeed, $a^{ij}$ admits the eigenvalues $1$ and $3$, thus by the \emph{Min-Max theorem} we get that
			\begin{equation*}
				1 \leq R_{A(x,y)}(z) \leq 3
			\end{equation*}
			\noindent for all $(x,y) \in \Omega$ and where $R_A(z)$ denotes the \emph{Rayleigh-Ritz quotient} defined by
			\begin{equation*}
				R_A(z) := \frac{\langle Az,z\rangle}{\norm{z}^2}
			\end{equation*}
			\noindent for $z \in \mathbb{C}^2$.
		\item A non-example would be $a^{ij}(x) := 0$.
		\item Another non-example is given by
			\begin{equation*}
				(a^{ij}(x,y)) := \begin{pmatrix}
					x^2 + y^2 & x + y\\
					x + y & 1
				\end{pmatrix}
			\end{equation*}
			\noindent for any $\Omega \subseteq \mathbb{R}^2$ containing the origin. Indeed, we get $\det(a^{ij}(0,0)) = 0$.
	\end{enumerate}
\end{examples}

\subsection*{Sobolev Spaces on an Interval}
In what follows, let $-\infty \leq a < b \leq \infty$ and $I := \intoo{a,b}$.

\begin{lemma}[Du Bois-Reymond]
	\label{lem:BoisReymond}
	Let $f \in L^1_{\mathrm{loc}}(I)$ such that
	\begin{equation*}
		\forall \varphi \in C^\infty_c(I): \> \int_I f\varphi' dx = 0.
	\end{equation*}
	Then $f$ is almost everywhere constant.
\end{lemma}

\begin{proof}
	Let $v := w - c_0 \psi$ for $w,\psi \in C^\infty_c(I)$ such that $\int_I \psi = 1$ and $\int_I v = 0$. This implies $c_0 = \int_I w$. By the fundamental theorem of calculus, the function $\varphi : I \to \mathbb{R}$ defined by
	\begin{equation*}
		\varphi(x) := \int_I v(t) dt
	\end{equation*}
	\noindent belongs to $C^\infty_c(I)$ with $\varphi' = v$. Thus we compute
	\begin{align*}
		0 = \int_I f \varphi' = \int_I f v = \int_I fw - c_0 \int_I f \psi = \int_I fw - \int_I w \int_I f \psi = \int_I (f - c)w,
	\end{align*}
	\noindent where $c := \int_I f \psi$. Since $w$ was arbitrary, we conclude by the fundamental lemma of variational calculus \ref{thm:flvc}.
\end{proof}

\begin{lemma}
	\label{lem:abscontint}
	Let $f \in L^1_{\mathrm{loc}}(I)$ and $x_0 \in I$. Then $u : I \to \mathbb{R}$ defined by
	\begin{equation*}
		u(x) := \int_{x_0}^x f(t)dt
	\end{equation*}
	\noindent is absolutely continuous and belongs to $W^{1,1}_{\mathrm{loc}}(I)$ with $u' = f$ a.e.
\end{lemma}

\begin{proof}
	Absolute continuity follows from real analysis. Let $\varphi \in C^\infty_c(I)$. Then Fubini yields
	\begin{align*}
		\int_I u\varphi' &= \int_a^{x_0}\int_{x_0}^x f(t)\varphi'(x)dtdx + \int_{x_0}^b \int_{x_0}^x f(t)\varphi'(x)dtdx\\
		&= -\int_a^{x_0}\int_x^{x_0} f(t)\varphi'(x)dtdx + \int_{x_0}^b \int_{x_0}^x f(t)\varphi'(x)dtdx\\
		&= -\int_a^{x_0}\int_a^t f(t)\varphi'(x)dxdt + \int_{x_0}^b \int_t^b f(t)\varphi'(x)dxdt\\
		&= -\int_a^{x_0} f(t)\varphi(t)dt - \int_{x_0}^b f(t)\varphi(t)dt\\
		&= -\int_I f \varphi.
	\end{align*}
\end{proof}

\begin{theorem}
	\label{thm:W1p}
	Let $u \in W^{1,p}(I)$. Then there exists an absolutely continuous representant $\wtilde{u}$ of $u$ on $\wbar{I}$, such that
	\begin{equation*}
		\wtilde{u}(x) = \wtilde{u}(x_0) + \int_{x_0}^x u'(t)dt
	\end{equation*}
	\noindent holds for all $x,x_0 \in I$. In particular, $\wtilde{u}$ is classically differentiable a.e. and $\wtilde{u}' = u'$.
\end{theorem}

\begin{proof}
	By lemma \ref{lem:abscontint}, the function $v(x) := \int_{x_0}^x u'(t)dt$ is in $W^{1,1}_{\mathrm{loc}}(I)$ with weak derivative $u'$. Moreover, for any $\varphi \in C^\infty_c(I)$ we compute
	\begin{equation*}
		\int_I (u - v)\varphi' = \int_I u\varphi' - \int_I v\varphi' = -\int_I u'\varphi + \int_I u'\varphi = 0.
	\end{equation*}
	Thus lemma \ref{lem:BoisReymond} yields $u = c + v$, for some $c \in \mathbb{R}$. Set
	\begin{equation*}
		\wtilde{u}(x) := c + \int_{x_0}^x u'(t) dt.
	\end{equation*}
	Then $\wtilde{u}(x_0) = c$ and thus the statement follows.
\end{proof}

\begin{theorem}[Characterization of $W^{1,p}(I)$]
	Let $1 < p \leq \infty$ and $u \in L^p(I)$. Then the following statements are equivalent:
	\begin{enumerate}[label = \textup{(}\alph*\textup{)},wide = 0pt]
		\item $u \in W^{1,p}(I)$.
		\item There exists $C \geq 0$ such that
			\begin{equation*}
				\forall \varphi \in C^\infty_c(I): \> \abs[3]{\int_I u\varphi'} \leq C\norm{\varphi}_{L^q}.
			\end{equation*}
		\item There exists $C \geq 0$ such that for all $I' \subseteq \subseteq I$ and $\abs{h} < \dist(I',\partial I)$ holds
			\begin{equation*}
				\norm{\tau_hu - u}_{L^p(I')} \leq C\abs{h},
			\end{equation*}
			\noindent where $\tau_hu(x) := u(x + h)$.
	\end{enumerate}
\end{theorem}

\begin{proof}
	The implication $(a)\Rightarrow(b)$ follows immediately from H\"older's inequality. To prove $(b)\Rightarrow(a)$, we observe that $l : C^\infty_c(I) \to \mathbb{R}$ defined by
	\begin{equation*}
		L(\varphi) := \int_I u \varphi'
	\end{equation*}
	\noindent is continuous. Since $C^\infty_c(I)$ is dense in $L^q(I)$, we get that $l \in (L^q(I))^*$. Hence we find $g \in L^p$, such that $\int_I g\varphi = l(\varphi)$ and so $u' = -g$.\\
	Next we show $(a)\Rightarrow(c)$. By theorem \ref{thm:W1p}, we find an absolutely continuous representant $\wtilde{u}$ of $u$. Thus
	\begin{equation*}
		\wtilde{u}(x + h) - \wtilde{u}(x) = h\int_0^1 u'(x + th) dt
	\end{equation*}
	Hence Jensen's inequality yields
	\begin{equation*}
		\norm[0]{\tau_hu - u}_{L^p(I')} \leq \abs{h} \int_0^1 \norm[0]{u'(\cdot + th)}_{L^p(I')}dt \leq \abs{h} \norm[0]{u'}_{L^p(I)}.
	\end{equation*}
	Lastly, we prove $(c)\Rightarrow(b)$. Let $\varphi \in C^\infty_c(I)$. Then we may find $I' \subseteq \subseteq I$ such that $\supp \varphi\subseteq I'$. Hence we compute
	\begin{align*}
		\abs[3]{\int_I u\varphi'} &= \lim_{h \to 0}\frac{1}{\abs{h}} \abs[3]{\int_I u(x) \del[1]{\varphi(x + h) - \varphi(x)} dx}\\
		&= \lim_{h \to 0}\frac{1}{\abs{h}} \abs[3]{\int_I \del[1]{u(x - h) - u(x)} \varphi(x)dx}\\
		&= \lim_{h \to 0}\frac{1}{\abs{h}} \abs[3]{\int_I \del[0]{\tau_{-h}u - u}\varphi}\\
		&\leq \lim_{h \to 0}\frac{1}{\abs{h}}\norm[0]{\tau_{-h}u - u}_{L^p(I')}\norm{\varphi}_{L^q(I)}\\
		&\leq C\norm{\varphi}_{L^q(I)}.
	\end{align*}
\end{proof}

\begin{theorem}[Extension Theorem]
	\label{thm:extension}
	There exists a continuous linear operator 
	\begin{equation*}
		E : W^{1,p}(I) \to W^{1,p}(\mathbb{R})
	\end{equation*}
	\noindent such that:
	\begin{enumerate}[label = \textup{(}\roman*\textup{)},wide = 0pt]
		\item $Eu\vert_I = u$.
		\item $\norm{Eu}_{L^p(\mathbb{R})} \leq C \norm{u}_{L^p(I)}$.
		\item $\norm[0]{(Eu)'}_{L^p(\mathbb{R})} \leq C\norm{u}_{W^{1,p}(I)}$.
	\end{enumerate}
\end{theorem}

\begin{proof}
	First we consider the case $I = \intoo{0,\infty}$. We extend $u$ by continuity to $0$ and then we extend $u$ by means of \emph{even symmetry}. If $I$ is bounded we can without loss of generality assume that $I = \intoo{0,1}$. Now use a cut-off function.
\end{proof}

\begin{theorem}[Approximation Theorem]
	Let $1 \leq p < \infty$ and $u \in W^{1,p}(I)$. Then there exists a sequence $(u_i)_{i \in \omega}$ in $C^\infty_c(\mathbb{R})$ such that
	\begin{equation*}
		\norm[0]{u_i\vert_I - u}_{W^{1,p}(I)} \to 0.
	\end{equation*}
\end{theorem}

\begin{proof}
	The main idea of the proof is to use convolutions. Moreover, it is enough to consider the case $I = \mathbb{R}$ only, due to the extension theorem \ref{thm:extension}. 	
\end{proof}

\begin{theorem}[Sobolev Embedding]
	\label{thm:Sobolev_embedding}
	There is a continuous embedding 
	\begin{equation*}
			W^{1,p}(I) \hookrightarrow L^\infty(I).
	\end{equation*}
\end{theorem}

\begin{proof}
	First consider $I$ bounded. By theorem \ref{thm:W1p} we get that
	\begin{equation*}
		\norm{u}_{L^\infty} = \sup_{x \in I} \abs[0]{u(x)} \leq \abs{u(y)} + \sup_{x \in I} \abs[3]{\int_y^x u'(t) dt} \leq \abs{u(y)} + \norm[0]{u'}_{L^1}, 
	\end{equation*}
	\noindent for any $y \in I$. Hence
	\begin{equation*}
		\norm{u}_{L^\infty} \leq \inf_{y \in I} \abs{u(y)} + \norm[0]{u'}_{L^1} \leq \frac{1}{\abs{I}}\int_I \abs{u(y)} + \norm[0]{u'}_{L^1} \leq C \norm{u}_{W^{1,1}} \leq C \norm{u}_{W^{1,p}}.
	\end{equation*}
	Assume now that $I$ is unbounded. Then we find $I' \subseteq \subseteq I$ such that 
	\begin{equation*}
		\norm{u}_{L^\infty(I')} \geq \frac{1}{2}\norm{u}_{L^\infty(I)}
	\end{equation*}
	\noindent and thus the claim follows by the previous computation. Indeed, note that by theorem \ref{thm:W1p}, we have that 
	\begin{equation*}
		\abs{u(x)} \leq \abs{u(y)} + \norm[0]{u'}_{L^1(I)}
	\end{equation*}
	\noindent for all $x \in I$ and fixed $y \in I$, and thus $u \in L^\infty(I)$. Moreover, there exists $x_0 \in I$ such that $\abs{u(x_0)} > \frac{1}{2}\norm{u}_{L^\infty(I)}$, if not, this would contradict the definition of the supremum norm. Since $u$ is continuous by theorem \ref{thm:W1p}, we find $\delta > 0$ such that
	\begin{equation*}
		\abs[0]{u(x) - u(x_0)} \leq \abs{u(x_0)} - \frac{1}{2}\norm{u}_{L^\infty(I)}
	\end{equation*}
	\noindent for all $x \in I$ such that $\abs{x - x_0} < \delta$. Hence the reversed triangle inequality yields
	\begin{equation*}
		\frac{1}{2}\norm{u}_{L^\infty(I)} - \abs{u(x_0)} \leq \abs[0]{u(x)} - \abs{u(x_0)} \leq \abs{u(x_0)} - \frac{1}{2}\norm{u}_{L^\infty(I)}
	\end{equation*}
	\noindent and so
	\begin{equation*}
		\frac{1}{2}\norm{u}_{L^\infty(I)} \leq \abs[0]{u(x)}
	\end{equation*}
	\noindent for all $x \in I \cap \intoo{x_0 - \delta,x_0 + \delta} =: I'$.
\end{proof}

\begin{corollary}
	Let $I$ be unbounded and $u \in W^{1,p}(I)$ for $1 \leq p < \infty$. Then $u \to 0$ as $\abs{x} \to \infty$.
\end{corollary}

\subsection*{Dirichlet and Neumann Boundary Problems on an Interval}
In what follows, let us consider $-\infty < a < b < \infty$ and $I := \intoo{a,b}$.

\begin{proposition}
	Let $f \in C^0(\wbar{I})$. Then the weak solution $u$ of the homogenous Dirichlet problem
	\begin{align*}
		\ccases{
			-u'' = f & \text{in }I,\\
			u(a) = 0 = u(b),
		}
	\end{align*}
	\noindent is a classical solution, i.e. $u \in C^2(\wbar{I})$.
\end{proposition}

\begin{proposition}
	Let $f \in C^0(\wbar{I})$. Then the weak solution $u$ of the homogenous Neumann problem
	\begin{align*}
		\ccases{
			-u'' + u = f & \text{in }I,\\
			u'(a) = 0 = u'(b),
		}
	\end{align*}
	\noindent is a classical solution, i.e. $u \in C^2(\wbar{I})$.
\end{proposition}

\subsection*{Sobolev Spaces on a Domain}

\begin{theorem}[Meyers-Serrin]
	\label{thm:Meyers-Serrin}
	Let $\Omega \subseteq \mathbb{R}^n$ be open. Then $C^\infty(\Omega) \cap W^{1,p}(\Omega)$ is dense in $W^{1,p}(\Omega)$ for every $1 \leq p < \infty$.
\end{theorem}

\begin{proof}
	Convolutions and a partition of unity argument.
\end{proof}

\begin{proposition}[Product Rule]
	Let $u,v \in W^{1,p}(\Omega) \cap L^\infty(\Omega)$. Then $uv \in W^{1,p}(\Omega) \cap L^\infty(\Omega)$ and
	\begin{equation*}
		\partial_\alpha(uv) = (\partial_\alpha u) v + u (\partial_\alpha v).
	\end{equation*}
\end{proposition}

\begin{proof}
	First consider the case \underline{$p < \infty$.} Then 
	\begin{equation*}
		\norm{uv}_{L^p} \leq \norm{u}_{L^\infty}\norm{v}_{L^p}
	\end{equation*} 
	\noindent and
	\begin{equation*}
		\norm[0]{(\partial_\alpha u) v + u (\partial_\alpha)v}_{L^p} \leq \norm[0]{\partial_\alpha u}_{L^p} \norm{v}_{L^\infty} + \norm{u}_{L^\infty}\norm[0]{\partial_\alpha v}_{L^p}.
	\end{equation*}
	Meyers-Serrin \ref{thm:Meyers-Serrin} yields the existence of sequences $u_k$ and $v_k$ in $C^\infty(\Omega) \cap W^{1,p}(\Omega)$ such that $u_k \to u$ and $v_k \to v$ in $W^{1,p}(\Omega)$. For any $\varphi \in C^\infty_c(\Omega)$, we compute
	\begin{align*}
		\int_\Omega uv \partial_\alpha\varphi &= \lim_{k \to \infty} \int_\Omega u_kv_k \partial_\alpha \varphi\\
		&= - \lim_{k \to \infty} \int_\Omega ((\partial_\alpha u_k)v_k + u_k(\partial_\alpha v_k))\varphi\\
		&= - \int_\Omega ((\partial_\alpha u)v + u(\partial_\alpha v))\varphi.
	\end{align*}
	Now consider the case \underline{$p = \infty$}. We have $uv \in L^\infty(\Omega)$ as well as $(\partial_\alpha u_k)v_k + u_k(\partial_\alpha v_k)\in L^\infty(\Omega)$. Let $\varphi \in C^\infty_c(\Omega)$. Hence we find $\Omega' \subseteq \subseteq \Omega$ with $\supp \varphi \subseteq \Omega'$. But then the above calculation holds on $\Omega'$.
\end{proof}

\begin{theorem}[Characterization of $W^{1,p}(\Omega)$]
	Let $1 < p \leq \infty$ and $u \in L^p(\Omega)$. Then the following statements are equivalent:
	\begin{enumerate}[label = \textup{(}\alph*\textup{)},wide = 0pt]
		\item $u \in W^{1,p}(\Omega)$.
		\item There exists $C \geq 0$ such that
			\begin{equation*}
				\forall \abs{\alpha} \leq 1\forall \varphi \in C^\infty_c(\Omega): \> \abs[3]{\int_I uD^\alpha \varphi} \leq C\norm{\varphi}_{L^q}.
			\end{equation*}
		\item There exists $C \geq 0$ such that for all $\Omega' \subseteq \subseteq \Omega$ and $\abs{h} < \dist(I',\partial I)$ holds
			\begin{equation*}
				\norm{\tau_hu - u}_{L^p(\Omega')} \leq C\abs{h},
			\end{equation*}
			\noindent where $\tau_hu(x) := u(x + h)$.
	\end{enumerate}
\end{theorem}

\begin{proof}
	The proof $(c)\Rightarrow(b)\Rightarrow(a)$ is almost the same as the one given in the characterization theorem for $\Omega$ an interval. For proving $(a)\Rightarrow(c)$, use Meyers-Serrin.
\end{proof}

\begin{corollary}
	Let $u \in L^\infty(\Omega)$. Then $u \in W^{1,\infty}(\Omega)$ if and only if $u$ admits a locally Lipschitz continuous representant. Moreover, if $\Omega$ is convex, then $u \in W^{1,\infty}(\Omega)$ if and only if $u$ admits a Lipschitz continuous representant. 
\end{corollary}

\subsection*{Extension and Trace Operator}

We start off with \emph{local theory}. In what follows, define
\begin{equation*}
	Q := \cbr[0]{(x',x_n) \in \mathbb{R}^{n - 1} \times \mathbb{R} : \abs[0]{x'} < 1, \abs{x_n} < 1}.
\end{equation*}
\noindent Moreover
\begin{equation*}
	Q_+ := \cbr{(x',x_n) \in Q : x_n > 0} \qquad \text{and} \qquad Q_0 := \cbr{(x',x_n) \in Q : x_n = 0}.
\end{equation*}

\begin{lemma}
	Let $u \in W^{1,p}(Q_+)$. Set
	\begin{align*}
		u^*(x',x_n) := \ccases{
			u(x',x_n) & x_n > 0,\\
			u(x',-x_n) & x_n < 0.
		}
	\end{align*}
	Then $u^* \in W^{1,p}(Q)$ and $\norm[0]{u^*}_{W^{1,p}(Q)} \leq C \norm{u}_{W^{1,p}(Q_+)}$.
\end{lemma}

Now to the \emph{global theory}.

\begin{theorem}[Extension]
	Let $\Omega \subseteq\subseteq \mathbb{R}^n$ of class $C^1$. Then there exists a continuous linear operator
	\begin{equation*}
		E : W^{1,p}(\Omega) \to W^{1,p}(\mathbb{R}^n)
	\end{equation*}
	\noindent such that:
	\begin{enumerate}[label = \textup{(}\roman*\textup{)},wide = 0pt]
		\item $Eu\vert_\Omega = u$.
		\item $\norm{Eu}_{L^p(\mathbb{R}^n)} \leq C \norm{u}_{L^p(\Omega)}$.
		\item $\norm[0]{Eu}_{W^{1,p}(\mathbb{R}^n)} \leq C\norm{u}_{W^{1,p}(\Omega)}$.
	\end{enumerate}
\end{theorem}

\begin{corollary}
	Let $\Omega \subseteq \subseteq \mathbb{R}^n$ of class $C^1$ and $1 \leq p < \infty$. Then $C^\infty(\wbar{\Omega})$ is dense in $W^{1,p}(\Omega)$.
\end{corollary}

Again, we tackle first the \emph{local theory}.

\begin{lemma}
	Let $u \in W^{1,p}(Q_+)$. Then $u\vert_{Q_0} \in L^p(Q_0)$ is well defined and the induced trace operator $W^{1,p}(Q_+) \to L^p(Q_0)$ is linear and continuous.
\end{lemma}

\begin{proof}
	We consider the case \underline{$1 \leq p < \infty$}. The main idea is to show this for $u \in C^\infty(Q)$, then for $u \in W^{1,p}(Q)$ and then finally for $u \in W^{1,p}(Q_+)$ by extension.\\
	Consider now \underline{$p = \infty$}. Since $Q_+$ is convex, $u \in W^{1,\infty}(Q_+)$ admits a Lipschitz continuous representant and the result follows by extending via continuity.
\end{proof}

\begin{theorem}[Characterization of $H^1(\Omega)$]
	\label{thm:characterization_H^1}
	Let $\Omega \subseteq \subseteq \mathbb{R}^n$. Then
	\begin{equation*}
		H^1(\Omega) = H^1_0(\Omega) \oplus \cbr[0]{u \in H^1(\Omega) : \Delta u = 0}.
	\end{equation*}
\end{theorem}

\begin{proof}
	Let $u \in H^1(\Omega)$ and let $u_0 \in H^1_0(\Omega)$ denote the unique solution of
	\begin{equation*}
		\forall \varphi \in C^\infty_c(\Omega): \> \int_\Omega \nabla u_0 \nabla \varphi = \int_\Omega \nabla u \nabla \varphi.
	\end{equation*}
	Set $u_1 := u - u_0$. Then for any $\varphi \in C^\infty_c(\Omega)$ we compute
	\begin{equation*}
		-\int_\Omega u_1 \Delta \varphi = \int_\Omega \nabla u_1 \nabla \varphi = \int_\Omega \nabla u \nabla \varphi - \int_\Omega \nabla u_0 \nabla \varphi = 0.
	\end{equation*}
	Thus $u = u_0 + u_1$ is of the desired form. Moreover, we have
	\begin{equation*}
		\norm[0]{\nabla u}_{L^2(\Omega)}^2 = \norm[0]{\nabla u_0}^2_{L^2(\Omega)} + \norm[0]{\nabla u_1}^2_{L^2(\Omega)} + 2 \int_\Omega \nabla u_0 \nabla u_1.
	\end{equation*}
	Since $u_0 \in H^1_0(\Omega)$, we find a sequence $\varphi_k$ in $C^\infty_c(\Omega)$ such that $\varphi_k \to u$ in $H^1(\Omega)$. But then
	\begin{equation*}
		\int_\Omega \nabla u_0 \nabla u_1 = \lim_{k \to \infty} \int_\Omega \nabla \varphi_k \nabla u_1 = 0.
	\end{equation*}
	Hence
	\begin{equation*}
		\norm[0]{\nabla u}_{L^2(\Omega)}^2 = \norm[0]{\nabla u_0}^2_{L^2(\Omega)} + \norm[0]{\nabla u_1}^2_{L^2(\Omega)}
	\end{equation*}
	\noindent which implies that the decomposition is direct. Indeed, suppose $u \in H^1_0(\Omega)$ such that $\Delta u = 0$. Then $u = u/2 + u/2$ which yields $u = 0$ by the above computation.
\end{proof}

\begin{corollary}[Characterization of $H^1_0(\Omega)$]
	Let $\Omega \subseteq \subseteq \mathbb{R}^n$ of class $C^1$. Then
	\begin{equation*}
		H^1_0(\Omega) = \cbr[0]{u \in H^1(\Omega) : u\vert_{\partial \Omega} = 0}.
	\end{equation*}
\end{corollary}

\begin{proof}
	Suppose $u \in H^1_0(\Omega)$. Then $\varphi_k \to u$ in $H^1(\Omega)$ for some sequence $\varphi_k$ in $C^\infty_c(\Omega)$. Hence
	\begin{equation*}
		u\vert_{\partial \Omega} = \lim_{k \to \infty} \varphi_k\vert_{\partial\Omega} = 0
	\end{equation*}
	\noindent by continuity of the trace operator. Conversly, suppose $u \in H^1(\Omega)$ with $u\vert_{\partial \Omega} = 0$. Using the characterization of $H^1(\Omega)$ \ref{thm:characterization_H^1}, we get a unique decomposition $u = u_0 + u_1$ for $u_0 \in H^1_0(\Omega)$ and $u_1 \in H^1(\Omega)$ with $\Delta u_1 = 0$. Moreover, observe that
	\begin{equation*}
		0 = u\vert_{\partial \Omega} = u_0\vert_{\partial\Omega} + u_1\vert_{\partial \Omega} = u_1\vert_{\partial \Omega}.
	\end{equation*}
	Since harmonic extensions are unique, we conclude $u_1 = 0$.
\end{proof}

\subsection*{Sobolev Embeddings}

\begin{theorem}[Sobolev Embedding Theorem]
	\label{thm:Sobolev_embedding_theorem}
	Let $\Omega \subseteq \subseteq \mathbb{R}^n$ of class $C^1$ and $k \in \omega$, $k \geq 1$. Then:
	\begin{enumerate}[label = \textup{(}\alph*\textup{)},wide = 0pt]
		\item If $kp < n$, then $W^{k,p}(\Omega) \hookrightarrow L^q(\Omega)$ for all $1 \leq q \leq p^* := \frac{np}{n - pk}$ and the embedding is compact for $q < p^*$.
		\item If $kp = n$, then $W^{k,p}(\Omega) \hookrightarrow L^q(\Omega)$ for all $1 \leq q < \infty$ and those embeddings are compact.
		\item If $kp > n$ and $k - \frac{n}{p} \notin \omega$, then $W^{k,p}(\Omega) \hookrightarrow C^{l,\alpha}(\Omega)$ for $l := \sbr[1]{k - \frac{n}{p}}$ and $0 \leq \alpha \leq \alpha^* := k - l - \frac{n}{p}$ and those embeddings are compact for $\alpha < \alpha^*$.
		\item If $kp > n$ and $k - \frac{n}{p} = l + 1 \in \omega$, then $W^{k,p}(\Omega) \hookrightarrow C^{l,\alpha}(\Omega)$ for $0 \leq \alpha < 1$ and those embeddings are compact.
	\end{enumerate}
\end{theorem}

\begin{corollary}
	Let $\Omega \subseteq \subseteq \mathbb{R}^n$ of class $C^1$ and $u \in H^1(\Omega)$. Moreover, assume that $u \in H^k(\Omega)$ for some $k > \frac{n}{2} + 2$. Then $u \in C^2(\Omega)$.
\end{corollary}

\subsubsection*{$p < n$}

\begin{theorem}[Sobolev-Gagliardo-Nirenberg]
	Let $1 \leq p < n$ and let $p^* := \frac{np}{n - p}$. Then $W^{1,p}(\mathbb{R}^n) \hookrightarrow L^{p^*}(\mathbb{R}^n)$ with
	\begin{equation*}
		\norm{u}_{L^{p^*}} \leq C \norm[0]{\nabla u}_{L^p}.
	\end{equation*}
\end{theorem}

\begin{theorem}[Sobolev-Gagliardo-Nirenberg Compactness]
	Let $\Omega \subseteq \subseteq \mathbb{R}^n$ of class $C^1$ and $1 \leq p < n$. Then $W^{1,p}(\Omega) \hookrightarrow L^q(\Omega)$ for $1 \leq q \leq p^*$ and the embedding is compact if $q < p^*$.
\end{theorem}

\subsubsection*{$p = n$}

\begin{theorem}
	It holds that $W^{1,n}(\mathbb{R}^n) \hookrightarrow L^p(\mathbb{R}^n)$ for $n \leq p < \infty$. Moreover, if $\Omega \subseteq \subseteq \mathbb{R}^n$ is of class $C^1$, then $W^{1,n}(\Omega) \hookrightarrow L^p(\Omega)$ compactly for any $1 \leq p < \infty$.
\end{theorem}

\subsubsection*{$p > n$}

\begin{theorem}
	Let $p > n$. Then $W^{1,p}(\mathbb{R}^n) \hookrightarrow C^{0,\alpha}(\mathbb{R}^n)$ with $\alpha := 1 - \frac{n}{p}$ and
	\begin{equation*}
		\norm[0]{u}_{C^{0,\alpha}(\mathbb{R}^n)} \leq \norm{u}_{W^{1,p}(\Omega)}.
	\end{equation*}
	In particular, we have that $W^{1,p}(\mathbb{R}^n) \hookrightarrow L^\infty(\mathbb{R}^n)$.
\end{theorem}

\begin{remark}
	For $p = \infty$, the statement is trivially true, since any function in $W^{1,\infty}(\mathbb{R}^n)$ is Lipschitz continuous since $\mathbb{R}^n$ is convex, and thus belongs to $C^{0,1}(\mathbb{R}^n)$.
\end{remark}

The proof uses the notion of so-called \emph{Campanato spaces}. 

\begin{theorem}
	Let $\Omega \subseteq \subseteq \mathbb{R}^n$ of type $A$ for some $A > 0$ and $1 \leq p < \infty$, $\lambda > n$, $\alpha := \frac{\lambda - n}{p}$. Then
	\begin{equation*}
		\mathcal{L}^{p,\lambda}(\Omega) \cong C^{0,\alpha}(\wbar{\Omega}).
	\end{equation*}
\end{theorem}

\begin{proof}
	The inclusion $\mathcal{L}^{p,\lambda}(\Omega) \cong C^{0,\alpha}(\wbar{\Omega})$ follows from the Campanato-theorem and does also hold for general $\Omega \subseteq \mathbb{R}^n$ open.
\end{proof}

\begin{lemma}
	Let $u \in W^{1,p}(\mathbb{R}^n)$, $1 \leq p < \infty$. Then for all $x_0 \in \mathbb{R}^n$ and $r > 0$ we have that
	\begin{equation*}
		\norm[0]{u - u_{x_0,r}}_{L^p(B_r(x_0))}^p \leq Cr^p \norm[0]{\nabla u}^p_{L^p(B_r(x_0))}.
	\end{equation*}
\end{lemma}

\begin{proof}
	This is an application of the Poincar\'e-Wirtinger inequality \ref{thm:PW} since without loss of generality, we may assume $x_0 = 0$ and $r = 1$.
\end{proof}

Now the proof of the Sobolev embedding theorem for $p > n$ is immediaty by considering
\begin{equation*}
	\begin{tikzcd}
		W^{1,p}(\mathbb{R}^n) \arrow[r,"\text{P.W.}",hook] & \mathcal{L}^{p,p}(\mathbb{R}^n) \arrow[r,"\text{Campanato}",hook] & C^{0,\alpha}(\mathbb{R}^n)
	\end{tikzcd}
\end{equation*}
\noindent and observing that $\mathbb{R}^n$ is of type $\frac{\pi^{n/2}}{\Gamma(n/2 + 1)} > 0$.

\begin{theorem}[Poincar\'e-Wirtinger Inequality]
	\label{thm:PW}
	Let $\Omega \subseteq \subseteq \mathbb{R}^n$ connected and of class $C^1$ and $1 \leq p < \infty$. Then there exists $C \geq 0$ such that 
	\begin{equation*}
		\norm[0]{u - \wbar{u}}_{L^p(\Omega)} \leq C \norm[0]{\nabla u}_{L^p(\Omega)}
	\end{equation*}
	\noindent holds for all $u \in W^{1,p}(\Omega)$.
\end{theorem}

\begin{proof}
	Towards a contradiction, assume that for any $C \geq 0$ there exists $u \in W^{1,p}(\Omega)$ such that 
	\begin{equation*}
		\norm[0]{u - \wbar{u}}_{L^p(\Omega)} > C \norm[0]{\nabla u}_{L^p(\Omega)}.
	\end{equation*}
	In particular, there exists a sequence $u_k$ in $W^{1,p}(\Omega)$, such that 
	\begin{equation*}
		\norm[0]{u_k - \wbar{u_k}}_{L^p(\Omega)} > k \norm[0]{\nabla u_k}_{L^p(\Omega)}
	\end{equation*}
	\noindent holds for each $k \in \omega$, $k \geq 1$. Defining $v_k := u_k - \wbar{u_k}$ and normalizing, i.e. setting $w_k := v_k/\norm{v_k}_{L^p(\Omega)}$ (this is valid since $\norm{v_k}_{L^p(\Omega)} > 0$), yields a sequence $w_k$ in $W^{1,p}(\Omega)$ such that 
	\begin{equation*}
		\wbar{w_k} = 0, \qquad \norm{w_k}_{L^p(\Omega)} = 1 \qquad \text{and} \qquad \norm[0]{\nabla w_k}_{L^p(\Omega)} \to 0
	\end{equation*}
	\noindent for any $k \in \omega$, $k \geq 1$. Using the Sobolev embedding theorem \ref{thm:Sobolev_embedding_theorem}, we get
	\begin{equation*}
		W^{1,p}(\Omega) \hookrightarrow W^{1,n}(\Omega) \hookrightarrow L^p(\Omega) \qquad \text{and} \qquad W^{1,p}(\Omega) \hookrightarrow L^p(\Omega)
	\end{equation*}
	\noindent if $p \geq n$ and $p < n$, respectively. Moreover, those are compact embeddings. Thus since $w_k$ is bounded in $W^{1,p}(\Omega)$, we have that $w_{k_i} \to w$ in $L^p(\Omega)$ for a subsequence $w_{k_i}$ of $w_k$. Moreover, $\nabla w = 0$. Indeed, for any $\varphi \in C^\infty_c(\Omega)$ we compute
	\begin{equation*}
		\int_\Omega w\nabla \varphi = \lim_{i \to \infty}\int_\Omega w_{k_i}\nabla\varphi = -\lim_{i \to \infty}\int_\Omega \nabla w_{k_i}\varphi = 0.
	\end{equation*}
	By the constancy lemma we therefore conclude that $w = c \in \mathbb{R}$ a.e. But
	\begin{equation*}
		\wbar{w} = \frac{1}{\abs{\Omega}} \int_\Omega w = \frac{1}{\abs{\Omega}} \lim_{i \to \infty} \int_\Omega w_{k_i} = 0
	\end{equation*}
	\noindent implies $w = 0$ a.e. contradicting
	\begin{equation*}
		\norm{w}_{L^p(\Omega)} = \lim_{i \to \infty} \norm[0]{w_{k_i}}_{L^p(\Omega)} = 1.
	\end{equation*}
\end{proof}

\section*{Regularity Theory}

Goal of this section is to prove the following regularity result.

\begin{theorem}[Global Regularity]
	\label{thm:regularity}
	Let $\Omega \subseteq \subseteq \mathbb{R}^n$ of class $C^{k + 2}$ and $f \in H^k(\Omega)$ for some $k \in \omega$. Moreover, let $u \in H^1_0(\Omega)$ be the unique solution of the homogenous Dirichlet boundary value problem
	\begin{align*}
		\ccases{
			A_0 u = f & \text{in } \Omega,\\
			u = 0 & \text{on } \partial \Omega.
		}
	\end{align*}
	\noindent where $a^{ij} \in C^{k + 1}(\wbar{\Omega})$. Then $u \in H^{k + 2}(\Omega)$ and 
	\begin{equation*}
		\norm{u}_{H^{k + 2}(\Omega)} \leq C\norm{f}_{H^k(\Omega)}.
	\end{equation*}
\end{theorem}

\begin{theorem}[Higher Sobolev Estimates]
	Let $\Omega \subseteq \subseteq \mathbb{R}^n$ of class $C^1$ and $k \in \omega$, $k \geq 1$. Then:
	\begin{enumerate}[label = \textup{(}\alph*\textup{)},wide = 0pt]
		\item If $kp < n$, then $W^{k,p}(\Omega) \hookrightarrow L^q(\Omega)$ for all $1 \leq q \leq p^* := \frac{np}{n - pk}$ and the embedding is compact for $q < p^*$.
		\item If $kp = n$, then $W^{k,p}(\Omega) \hookrightarrow L^q(\Omega)$ for all $1 \leq q < \infty$ and those embeddings are compact.
		\item If $kp > n$ and $k - \frac{n}{p} \notin \omega$, then $W^{k,p}(\Omega) \hookrightarrow C^{l,\alpha}(\Omega)$ for $l := \sbr[1]{k - \frac{n}{p}}$ and $0 \leq \alpha \leq \alpha^* := k - l - \frac{n}{p}$ and those embeddings are compact for $\alpha < \alpha^*$.
		\item If $kp > n$ and $k - \frac{n}{p} = l + 1 \in \omega$, then $W^{k,p}(\Omega) \hookrightarrow C^{l,\alpha}(\Omega)$ for $0 \leq \alpha < 1$ and those embeddings are compact.
	\end{enumerate}
\end{theorem}

\begin{corollary}
	Let $\Omega \subseteq \subseteq \mathbb{R}^n$ of class $C^1$ and $u \in H^1(\Omega)$. Moreover, assume that $u \in H^k(\Omega)$ for some $k > \frac{n}{2} + 2$. Then $u \in C^2(\Omega)$.
\end{corollary}

\subsection*{Interior Regularity}

\begin{theorem}
	Let $\Omega \subseteq \subseteq \mathbb{R}^n$ of class $C^1$ and $L$ an elliptic operator in divergence form satisfying $a^{ij} \in C^{k + 1}(\wbar{\Omega})$. If $f \in H^k(\Omega)$, the unique weak solution $u \in H^1_0(\Omega)$ of the homogenous Dirichlet boundary value problem
	\begin{align*}
		\ccases{
			A_0 u = f & \text{in } \Omega,\\
			u = 0 & \text{on } \partial\Omega
		}
	\end{align*}
	\noindent belongs to $H^{k + 2}_{\mathrm{loc}}(\Omega)$ and for all $\Omega' \subseteq \subseteq \Omega$ we have the estimate
	\begin{equation*}
		\norm{u}_{H^{k + 2}(\Omega')} \leq C(\norm{f}_{H^k(\Omega)} + \norm{u}_{H^1(\Omega)}).
	\end{equation*}
\end{theorem}

\begin{proof}
	\begin{enumerate}[label = \emph{Step \arabic*:}, wide = 0pt]
		\item
			~
			$k = 0$. 
			\begin{enumerate}[label = \textup{(}\alph*\textup{)},wide = 0pt]
				\item \emph{A-priori Estimates.} First of all, we are assuming that $u \in H^2_{\text{loc}}(\Omega)$. 
				\begin{enumerate}[label = \textup{(}\roman*\textup{)}]
					\item \emph{$H^1$-Estimate.} Choose a bump function $\varphi \in C^\infty_c(\Omega)$ supported in $\Omega'$. Thus the weak formulation yields by plugging in the test function $\varphi^2 u$
						\begin{equation}
							\label{eq:weak_explicit}
							\int_\Omega \varphi^2 a^{ij}\frac{\partial u}{\partial x^i}\frac{\partial u}{\partial x^j} + 2\int_{\Omega} u\varphi a^{ij}\frac{\partial u}{\partial x^i}\frac{\partial \varphi}{\partial x^j} = \int_\Omega f \varphi^2 u.
						\end{equation}
						Rearanging formula (\ref{eq:weak_explicit}) we compute
						\begin{align*}
							\int_\Omega \varphi^2 a^{ij}\frac{\partial u}{\partial x^i}\frac{\partial u}{\partial x^j} &= \int_\Omega f \varphi^2 u -  2\int_{\Omega} u\varphi a^{ij}\frac{\partial u}{\partial x^i}\frac{\partial \varphi}{\partial x^j}\\
							&\leq \norm{f}_{L^2(\Omega)}^2 + \norm{u}_{L^2(\Omega)}^2 + 2 \Lambda \int_\Omega (-u)\varphi \abs[0]{\nabla u}\abs[0]{\nabla \varphi}\\
							&\leq \norm{f}_{L^2(\Omega)}^2 + \norm{u}_{L^2(\Omega)}^2 + \Lambda \varepsilon \norm[0]{\varphi\nabla u}_{L^2(\Omega)}^2 + \frac{\Lambda}{\varepsilon}\norm[0]{u \nabla \varphi}_{L^2(\Omega)}^2
						\end{align*}
						Noticing that
						\begin{equation*}
							\int_\Omega \varphi^2 a^{ij}\frac{\partial u}{\partial x^i}\frac{\partial u}{\partial x^j} \geq \lambda \norm[0]{\varphi\nabla u}_{L^2(\Omega)}^2
						\end{equation*}
						\noindent yields
						\begin{equation*}
							(\lambda - \Lambda \varepsilon) \norm[0]{\varphi\nabla u}_{L^2(\Omega)}^2 \leq  \norm{f}_{L^2(\Omega)}^2 + \norm{u}_{L^2(\Omega)}^2 + \frac{\Lambda}{\varepsilon}\norm[0]{u \nabla \varphi}_{L^2(\Omega)}^2
						\end{equation*}
						Picking $\varepsilon > 0$ appropriately, yields
						\begin{equation*}
							\norm[0]{\varphi\nabla u}_{L^2(\Omega)}^2 \leq  C(\norm{f}_{L^2(\Omega)}^2 + \norm{u}_{L^2(\Omega)}^2)
						\end{equation*}
						\noindent and thus
						\begin{equation*}
							\norm[0]{\nabla u}_{L^2(\Omega')}^2 \leq C(\norm{f}_{L^2(\Omega)}^2 + \norm{u}_{L^2(\Omega)}^2).
						\end{equation*}
					\item \emph{$H^2$-Estimate.} Let $1 \leq \mu \leq n$. Then $\partial_\mu u$ solves
						\begin{equation*}
							\int_\Omega a^{ij}\partial_i\partial_\mu u\partial_j\varphi = - \int_\Omega f\partial_\mu \varphi - \int_\Omega \partial_\mu a^{ij}\partial_i u\partial_j \varphi
						\end{equation*}
						\noindent for all $\varphi \in C^\infty_c(\Omega)$. Now perform the $H^1$-estimate on $\partial_\mu u$. 
				\end{enumerate}
			\item \emph{Existence: The Nirenberg-Trick.} The trick is to use difference quotients
				\begin{equation*}
					D_hu := \frac{\tau_hu - u}{\abs{h}}
				\end{equation*}
				\noindent for $h \in \mathbb{R}^n$ such that $\abs{h} < \dist(\Omega',\partial \Omega)$. The idea now is to find a PDE solved by $D_h u$ in the weak sense and to to use the characterization of the Sobolev space.
			\end{enumerate}
		\item \emph{Induction Step.}
	\end{enumerate}
\end{proof}

\subsection*{Boundary Regularity}

\begin{proposition}[Minimality Property]
	Let $\Omega \subseteq \subseteq \mathbb{R}^n$. Then $u \in H^1_0(\Omega)$ solves (\ref{eq:HDP}) if and only if the \bld{energy functional} satisfies
	\begin{equation*}
		E(u) := \frac{1}{2}\norm{u}^2_a - \int_\Omega fu = \inf_{v \in H^1_0(\Omega)} E(v).
	\end{equation*}
\end{proposition}

\begin{proof}
	Suppose $u \in H^1_0(\Omega)$ solves (\ref{eq:HDP}) and let $v \in H^1_0(\Omega)$. Then $v = u + \varphi$ for some $\varphi \in H^1_0(\Omega)$ and we compute
	\begin{equation*}
		E(v) = E(u + \varphi) = \frac{1}{2}\norm{u}^2_a + \langle u,\varphi \rangle_a + \frac{1}{2}\norm{\varphi}^2_a - \int_\Omega f(u + \varphi) = E(u) + \frac{1}{2}\norm{\varphi}^2_a \geq E(u).
	\end{equation*}
	Conversly, suppose $u_0 \in H^1_0(\Omega)$ is a minimizer of the energy functional. Thus by elementary calculus
	\begin{equation*}
		\frac{d}{dt}\bigg\vert_{t = 0} E(u_0 + tv) = 0
	\end{equation*}
	\noindent for all $v \in H^1_0(\Omega)$. But
	\begin{equation*}
		\frac{d}{dt}\bigg\vert_{t = 0} E(u_0 + tv) = \langle u_0, v \rangle_a - \int_\Omega f v.
	\end{equation*}
\end{proof}

\subsection*{Eigenfunctions of $-\Delta$}

\begin{theorem}
	Let $\Omega \subseteq \subseteq \mathbb{R}^n$ of class $C^2$. Then there exists a Hilbert-space basis $(\varphi_i)_{i \in \omega}$ of $L^2(\Omega)$ consisting of eigenfunctions of the Laplace opertator, i.e. 
	\begin{align*}
		\ccases{
			-\Delta \varphi_i = \lambda_i \varphi_i & \text{in } \Omega,\\
			\varphi_i = 0 & \text{on } \partial \Omega.
		}
	\end{align*}
	Moreover $0 < \lambda_i \to \infty$ are called \bld{Dirichlet eigenvalues}.
\end{theorem}

\begin{proof}
	Define $K : L^2(\Omega) \to L^2(\Omega)$ by setting $Kf$ to be the unique weak solution of the homogenous Dirichlet boundary value problem
	\begin{align*}
		\ccases{
			-\Delta u = f & \text{in } \Omega,\\
			u = 0 & \text{on } \partial \Omega.
		}
	\end{align*}
	By the global regularity theorem, $u \in H^2(\Omega)$ and thus we can write $K$ as the composition
	\begin{equation*}
		\begin{tikzcd}
			L^2(\Omega) \arrow[r] & H^2(\Omega) \arrow[r,hook] & L^2(\Omega).
		\end{tikzcd}
	\end{equation*}
	Thus $K$ is continuous as a composition of continuous mappings an moreover, since the embedding is compact by the Sobolev theorem, so is $K$.
\end{proof}

\section*{Schauder Theory}
\subsection*{Schauder Estimates}

\begin{theorem}[Global $C^{2,\alpha}$-Estimate]
	\label{thm:gSe}
	Let $\Omega \subseteq \subseteq \mathbb{R}^n$ of class $C^{2,\alpha}$, $0 < \alpha < 1$. Moreover, let $a^{ij} \in C^{1,\alpha}(\Omega)$ symmetric, uniformly elliptic and uniformly bounded, $c \in C^\alpha(\Omega)$, $u_0 \in C^{2,\alpha}(\wbar{\Omega})$, $f = (f^1,\dots,f^n) \in C^{1,\alpha}(\Omega)$ and $h \in C^{\alpha}(\Omega)$. Then any solution $u \in C^{2,\alpha}(\Omega)$ of the Dirichlet boundary value problem
	\begin{align*}
		\ccases{
			A_0u + cu = -\frac{\partial}{\partial x^i} f^i + h & \text{in } \Omega,\\
			u = u_0 & \text{on } \Omega
		}
	\end{align*}
	\noindent satisfies
	\begin{equation*}
		\norm{u}_{C^{2,\alpha}} \leq C(\norm{u}_{H^1} + \norm{f}_{C^{1,\alpha}} + \norm{h}_{C^\alpha} + \norm{u_0}_{C^{2,\alpha}})
	\end{equation*}
	\noindent where $C$ does not depend on $u$.
\end{theorem}

\subsection*{Existence Theorems}

\begin{proposition}[Method of Continuity]
	\label{pro:moc}
	Let $(X,\norm{\cdot}_X)$ and $(Y,\norm{\cdot}_Y)$ be Banach spaces. Given $A_0,A_1 \in \mathcal{L}(X,Y)$ define $A_t := (1 - t)A_0 + tA_1$, $t \in \intcc{0,1}$. Suppose that
	\begin{equation*}
		\exists C > 0\forall t \in \intcc{0,1}\forall x \in X:\> \norm{x}_X \leq \norm{A_t x}_Y.
	\end{equation*}
	\noindent Then $A_0$ is surjective if and only if $A_1$ is surjective.
\end{proposition}

Using the method of continuity \ref{pro:moc}, one can show existence results of solutions of Dircihlet boundary value problems. Define
\begin{equation*}
	A_0 := - \frac{\partial}{\partial x^i}\del[3]{a^{ij}\frac{\partial}{\partial x^j}}
\end{equation*}
\noindent for $a^{ij} \in C^{1,\alpha}$ symmetric, uniformly elliptic and uniformly bounded. Consider the problem
\begin{align*}
	\ccases{
		A_0u + cu = -\frac{\partial}{\partial x^j}f^i + h & \text{in } \Omega,\\
		u = u_0 & \text{on } \partial \Omega
	}
\end{align*}
\noindent for $c \in C^\alpha$, $f = (f^1,\dots,f^n) \in C^{1,\alpha}$ and $h \in C^\alpha$. If $c \geq 0$, one can show existence and uniqueness of $C^{2,\alpha}$ solutions. First of all, suppose that solutions of 
\begin{align*}
	\ccases{
		A_0u = h & \text{in } \Omega,\\
		u = 0 & \text{on } \partial \Omega
	}
\end{align*}
\noindent do exist. Let us define
\begin{equation*}
	X := \cbr[0]{u \in C^{2,\alpha} : u\vert_{\partial \Omega} = 0} \qquad \text{and} \qquad Y := C^\alpha.
\end{equation*}
Then $X$ and $Y$ are Banach spaces, since $X$ is a closed subset of a Banach space. Define now $A_1 := A_0 + c$. Then it is easy to show that $A_0$ and $A_1$ are continuous. Thus to apply the continuity method, we have to show the existence of a constant $C > 0$, such that for all $t \in \intcc{0,1}$ and $u \in X$
\begin{equation*}
	\norm{x}_{C^{2,\alpha}} \leq \norm{A_t x}_{C^\alpha}
\end{equation*}
\noindent holds. But this looks like the $C^{2,\alpha}$-estimate \ref{thm:gSe}. Indeed, since $u \in C^{2,\alpha}$ solves $A_t u = A_tu$, we get
\begin{equation*}
	\norm{u}_{C^{2,\alpha}} \leq C(\norm{u}_{H^1} + \norm[0]{A_tu}_{C^\alpha}).
\end{equation*}
Using ellipticity, integration by parts (justified since any fucntion in $X$ vanishes on the boundary $\partial \Omega$) and $c \geq 0$, we compute
\begin{align*}
	\lambda\norm{u}_{H^1}^2 &= \lambda\int_{\Omega}\abs[0]{\nabla u}^2\\
	&\leq \int_{\Omega} a^{ij} \frac{\partial u}{\partial x^i}\frac{\partial u}{\partial x^j}\\
	&= \int_\Omega (A_0u)u\\
	&= \int_\Omega (A_0u)u + ctu^2 - ctu^2\\
	&= \int_\Omega (A_tu)u - ctu^2\\
	&\leq \int_\Omega (A_tu)u\\
	&\leq \norm{A_tu}_{L^2}\norm{u}_{L^2}\\
	&\leq C\norm{A_tu}_{C^\alpha}\norm{u}_{H^1}.
\end{align*}


\section*{Maximum Principle}
\subsection*{Weak Maximum Principle}

Let $\Omega \subseteq \subseteq \mathbb{R}^n$. In what follows, we consider the second order homogenous differential operator in non-divergence form
\begin{equation*}
	Lu := a^{ij}\frac{\partial^2u}{\partial x^i \partial x^j} + b^i \frac{\partial u}{\partial x^i} + c u
\end{equation*}
\noindent where $a^{ij},b^i,c \in C^0(\wbar{\Omega})$ and $L$ is uniformly elliptic, i.e. there exists $\lambda > 0$ such that
\begin{equation*}
	a^{ij}(x)\xi_i\xi_j \geq \lambda \abs{\xi}^2
\end{equation*}
\noindent holds for all $\xi \in \mathbb{R}^n$ and $x \in \Omega$.

\begin{theorem}[Weak Maximum Principle]
	\label{thm:WMP}
	Let $\Omega \subseteq \subseteq \mathbb{R}^n$ and $u \in C^2(\Omega) \cap C^0(\wbar{\Omega})$ such that $Lu \geq 0$. Then:
	\begin{enumerate}[label = \textup{(}\alph*\textup{)},wide = 0pt]
		\item If $c \leq 0$ in $\Omega$, then $\max_{\wbar{\Omega}} u \leq \max_{\partial \Omega} u_+$.
		\item If $c = 0$ in $\Omega$, then $\max_{\wbar{\Omega}} u = \max_{\partial \Omega} u$.
	\end{enumerate}
\end{theorem}

\begin{proof}
	Consider the perturbation $u_\varepsilon := u + \varepsilon e^{\gamma x_1}$ for $\varepsilon,\gamma > 0$ and use the first and second derivative test.
\end{proof}

\subsection*{Strong Maximum Principle}

\begin{theorem}[Strong Maximum Principle, E. Hopf]
	\label{thm:SMP}
	Let $\Omega \subseteq \subseteq \mathbb{R}^n$ connected and $u \in C^2(\Omega)$ such that $Lu \geq 0$. Then:
	\begin{enumerate}[label = \textup{(}\alph*\textup{)},wide=0pt]
		\item If $c \leq 0$ in $\Omega$, then 
			\begin{equation*}
				\del[0]{\exists x_0 \in \Omega:\> \sup_\Omega u = u(x_0) \geq 0} \rightarrow u = u(x_0).
			\end{equation*}
		\item If $c = 0$ in $\Omega$, then
			\begin{equation*}
				\del[1]{\exists x_0 \in \Omega:\> \sup_\Omega u = u(x_0)} \rightarrow u = u(x_0).
			\end{equation*}
		\item 
			\begin{equation*}
				\del[1]{\exists x_0 \in \Omega:\> \sup_\Omega u = u(x_0) = 0} \rightarrow u = 0.
			\end{equation*}
	\end{enumerate}
\end{theorem}

\begin{lemma}[Boundary Point Lemma, E. Hopf]
	Let $B := B_\rho(y) \subseteq \mathbb{R}^n$ and $u \in C^2(B) \cap C^0(\wbar{B})$ such that $Lu \geq 0$ in $B$ with $c \leq 0$. Assume for some $x_0 \in \partial B$ that $u(x_0) \geq 0$ and $u(x) < u(x_0)$ for every $x \in B$. Then
	\begin{equation*}
		D_\eta^+(x_0) := \limsup_{h \to 0} \frac{u(x_0 + h\eta) - u(x_0)}{h} < 0
	\end{equation*}
	\noindent for $\eta$ the invard pointing unit normal at $x_0$. Moreover, if $c = 0$, then we do not require $u(x_0) \geq 0$ and if $u(x_0) = 0$ we can neglect the sign of $c$.
\end{lemma}

\begin{proof}
	Without loss of generality one can assume $\rho = 1$ and $y = 0$. Then define $w : \wbar{B} \to \mathbb{R}$ by
	\begin{equation*}
		w(x) := e^{-\alpha \abs{x}^2} - e^{-\alpha}
	\end{equation*}
	\noindent for some $\alpha > 0$ to be determined. We compute
	\begin{equation*}
		Lw \geq e^{-\alpha\abs{x}^2}(4\mu \abs{x}^2\alpha^2 -2 \alpha (\tr A + b^i x_i) + c).
	\end{equation*}
	Thus for some $\alpha$ large enough, we get that $Lw > 0$. Set 
	\begin{equation*}
		v := u - u(x_0) + \varepsilon w
	\end{equation*}
	\noindent for some $\varepsilon > 0$ on the anulus $A := \wbar{B}_1(0) \setminus B_{1/2}(0)$. For $\varepsilon > 0$ sufficiently small, we get that $v \leq 0$ on $\partial A$. Since moreover
	\begin{equation*}
		Lv = Lu - cu(x_0) + \varepsilon Lw > 0
	\end{equation*}
	\noindent the weak maximum principle implies $v \leq 0$ on $A$. Hence $D^+_\eta v \leq 0$, but 
	\begin{equation*}
		D^+_\eta v = D^+_\eta u + \varepsilon D^+_\eta w
	\end{equation*}
	\noindent which yields the statement by observing that $D^+_\eta w > 0$.
\end{proof}




\printbibliography

\end{document}
