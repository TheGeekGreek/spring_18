\input{header.tex}

\setcounter{section}{1}

\title{Functional Analysis II Summary}
\author{Yannis B\"{a}hni}
\address[Yannis B\"{a}hni]{University of Zurich, R\"{a}mistrasse 71, 8006 Zurich}
\email[Yannis B\"{a}hni]{\href{mailto:yannis.baehni@uzh.ch}{\nolinkurl{yannis.baehni@uzh.ch}}}

\begin{document}

\begin{abstract}
	This is a rough summary of the course \emph{Functional Analysis II} held at \emph{ETH Zurich} by \emph{Prof. Dr. Alessandro Carlotto} in spring $2018$. The main focus of this summary is to give a neat preparation for the oral exam.
\end{abstract}

\maketitle

\tableofcontents

\section*{Sobolev Space Theory}
\subsection*{The Spaces $W^{k,p}(\Omega)$}
In what follows, let $n \in \omega$, $n \geq 1$, and $1 \leq p \leq \infty$.

\begin{definition}[Distributional and Weak Derivative]
	Let $\Omega \subseteq \mathbb{R}^n$ open and $u \in L^1_{\mathrm{loc}}(\Omega)$. For any multiindex $\alpha$, the \bld{distributional derivative of order $\alpha$ of $u$}, written $D^\alpha u$, is defined to be the mapping $D^\alpha u : C^\infty_c(\Omega) \to \mathbb{R}$ defined by
	\begin{equation*}
		\varphi \mapsto (-1)^{\abs{\alpha}}\int_\Omega u D^\alpha \varphi dx.
	\end{equation*}
	Moreover, a function $D^\alpha u \in L^p(\Omega)$ is called \bld{weak derivative of order $\alpha$ of $u$ with exponent $p$}, iff
	\begin{equation*}
		\forall \varphi \in C^\infty_c(\Omega): \> \int_\Omega D^\alpha u \varphi dx =  (-1)^{\abs{\alpha}} \int_\Omega u D^\alpha \varphi dx.
	\end{equation*}
\end{definition}

\begin{theorem}[Fundamental Lemma of Variational Calculus]
	\label{thm:flvc}
	Let $\Omega \subseteq \mathbb{R}^n$ open and $f \in L^1_{\mathrm{loc}}(\Omega)$. If
	\begin{equation*}
		\forall \varphi \in C^\infty_c(\Omega) : \> \int_\Omega f \varphi dx = 0,
	\end{equation*}
	\noindent then $f = 0$ a.e.
\end{theorem}

\begin{remark}
	Let $\Omega \subseteq \mathbb{R}^n$ open. Then $L^p(\Omega) \subseteq L^1_{\mathrm{loc}}(\Omega)$.
\end{remark}

\begin{remark}
	From the fundamental lemma of variational calculus \ref{thm:flvc} it follows that \emph{weak derivatives, if they exist, are unique}.
\end{remark}

\begin{examples}
	~
	\begin{enumerate}[label = \textup{(}\alph*\textup{)},wide = 0pt]
		\item Consider $\Omega := \intoo{-1,1}$ and $u := \abs{x}$. Then $u' = \chi_{\intco{0,1}} -\chi_{\intoo{-1,0}}$.
		\item Consider $\Omega := \mathbb{R}$ and $u := \chi_{\intoo{0,\infty}}$. Then the weak derivative $u'$ does not exist. Indeed, the \emph{Dirac distribution} is not representable as one may see by considering the smooth family $\varphi_\varepsilon : \mathbb{R} \to \mathbb{R}$ for $\varepsilon > 0$ defined by
			\begin{equation*}
				\varphi_\varepsilon(x) := \begin{cases}
					e^{\varepsilon^2/(x^2 - \varepsilon^2)} & \abs{x} < \varepsilon,\\
					0 & \abs{x} \geq \varepsilon.
				\end{cases}
			\end{equation*}
	\end{enumerate}
\end{examples}

\begin{definition}[Sobolev Space]
	Let $\Omega \subseteq \mathbb{R}^n$ open. For any $k \in \omega$, the \bld{Sobolev space of index $(k,p)$}, written $W^{k,p}(\Omega)$, is defined to be the space
	\begin{equation*}
		W^{k,p}(\Omega) := \cbr[0]{f \in L^p(\Omega) : D^\alpha u \in L^p(\Omega) \text{ exists for all } \abs{\alpha} \leq k},
	\end{equation*}
	\noindent with norm
	\begin{equation*}
		\norm{-} _{W^{k,p}(\Omega)} := \sum_{\abs{\alpha} \leq k} \norm[0]{D^\alpha -}_{L^p(\Omega)}.
	\end{equation*}
	Moreover, define
	\begin{equation*}
		W^{k,p}_0(\Omega) := \overline{C^\infty_c(\Omega)}^{\norm{-}_{W^{k,p}(\Omega)}},
	\end{equation*}
	\noindent and $H^k(\Omega) := W^{k,2}(\Omega)$ as well as $H_0^k(\Omega) := W^{k,2}_0(\Omega)$.
\end{definition}

\begin{theorem}
	Let $\Omega \subseteq \mathbb{R}^n$ open. Then $W^{k,p}(\Omega)$ is
	\begin{enumerate}[label = \textup{(}\alph*\textup{)},wide = 0pt]
		\item a Banach space for all $1 \leq p \leq \infty$.
		\item separable for all $1 \leq p < \infty$.
		\item reflexive for all $1 < p < \infty$.
	\end{enumerate}
\end{theorem}

\begin{proof}
	~
	\begin{enumerate}[label = \textup{(}\alph*\textup{)},wide = 0pt]
		\item This follows from the fact that $L^p(\Omega)$ is a Banach space for all $1 \leq p \leq \infty$. Let $(f_i)_{i \in \omega}$ be a Cauchy sequence in $W^{k,p}$. By definition of the $W^{k,p}$-norm, $(D^\alpha f_i)_{i \in \omega}$ is a Cauchy sequence in $L^p$. Thus we get $D^\alpha f_i \to f_\alpha$ in $L^p$, in particular, $f_i \to f$ in $L^p$. Using H\"older's inequality we compute
			\begin{equation*}
				\int_\Omega f_\alpha \varphi dx = \lim_{i \to \infty} \int_\Omega D^\alpha f_i \varphi dx = (-1)^{\abs{\alpha}} \lim_{i \to \infty}\int_\Omega f_i D^\alpha \varphi dx = (-1)^{\abs{\alpha}}\int_\Omega f D^\alpha \varphi dx
			\end{equation*}
			\noindent for $\varphi \in C^\infty_c(\Omega)$.
		\item For simplicity, we consider $k = 1$ only. Consider $\iota : W^{1,p} \hookrightarrow (L^p)^{n + 1}$ defined in the obvious way. Then $\iota$ is an isometry and the statement follows. 
		\item Same argument as in part (b).
	\end{enumerate}
\end{proof}

\subsection*{Sobolev Spaces on an Interval}

\section*{Elliptic Operators in Divergence Form}

\begin{lemma}[Poincar\'e Inequality]
	\label{lem:PI}
\end{lemma}

\begin{theorem}[Riesz Representation Theorem]
	\label{thm:RRT}
	Let $H$ be a real Hilbert space. Then the mapping $J : H \to H^*$ defined by $J(x) := \langle x,-\rangle$ is an isometric isomorphism.
\end{theorem}

\begin{theorem}
	Let $\Omega \subseteq\subseteq \mathbb{R}^n$, $k \in \omega$ and consider the elliptic operator
	\begin{equation*}
		L := \sum_{i,j = 1}^n \frac{\partial}{\partial x^i}\del[3]{a_{ij}\frac{\partial}{\partial x^j}},
	\end{equation*}
	\noindent for $a_{ij} \in C^{k + 1}(\wbar{\Omega})$ symmetric. Then:
	\begin{enumerate}[label = \textup{(}\alph*\textup{)},wide=0pt]
		\item Given $f \in L^2(\Omega)$, the homogenous Dirichlet problem
			\begin{align}
				\label{eq:HDP}
				\ccases{
					-L(u) = f & \text{in } \Omega,\\
					u = 0 & \text{on } \partial\Omega
				}
			\end{align}
			\noindent admits a unique weak solution $u \in H^1_0(\Omega)$.
		\item If $f \in H^k(\Omega)$ for some $k \in \omega$, then we have $u \in H^{k + 2}_{\mathrm{loc}}(\Omega)$ for the unique weak solution of part (a) and moreover, for any $\Omega' \subseteq \subseteq \Omega$ we have the estimate
			\begin{equation*}
				\norm{u}_{H^{k + 2}(\Omega')} \leq C\del[1]{\norm{f}_{H^k(\Omega)} + \norm{u}_{H^1(\Omega)}}.
			\end{equation*}
	\end{enumerate}
\end{theorem}

\begin{proof}
	~
	\begin{enumerate}[label = \textit{Step \arabic*:},wide=0pt]
		\item \textit{Derivation of Weak Formulation.} Suppose $u \in C^2(\wbar{\Omega})$ is a solution of (\ref{eq:HDP}). Let $\varphi \in C^\infty_c(\Omega)$. Then integration by parts (see \cite[436]{lee:smooth_manifolds:2013}) yields
			\begin{equation*}
				-\int_\Omega L(u) \varphi = - \sum_{j = 1}^n \int_\Omega \mathrm{div}(X_j)\varphi = \sum_{i,j = 1}^n \int_\Omega a_{ij} \frac{\partial u}{\partial x^j}\frac{\partial \varphi}{\partial x^i} = \sum_{i,j = 1}^n \int_\Omega a_{ij} \frac{\partial u}{\partial x^i}\frac{\partial \varphi}{\partial x^j},
			\end{equation*}
			\noindent where $X_j := \del[2]{a_{ij}\frac{\partial}{\partial x^j}}_i$. Thus we get the weak formulation:
			\begin{equation}
				\label{eq:HDPweak}
				\sum_{i,j = 1}^n \int_\Omega a_{ij} \frac{\partial u}{\partial x^i}\frac{\partial \varphi}{\partial x^j} = \int_\Omega f \varphi \qquad \forall \varphi \in C^\infty_c(\Omega).
			\end{equation}
		\item \textit{Existence and Uniqueness of Weak Solutions.} Since $L$ is uniformly elliptic, there exists $\lambda > 0$ such that 
			\begin{equation*}
				\sum_{i,j = 1}^n a_{ij}(x) \xi_i \xi_j \geq \lambda \abs{\xi}^2
			\end{equation*}
			\noindent holds for any $x \in \Omega$ and $\xi \in \mathbb{R}^n$. Moreover, since $a_{ij} \in C^0(\wbar{\Omega})$, we get that $L$ is uniformly bounded, i.e. there exists $\Lambda > 0$ such that
			\begin{equation*}
				\sum_{i,j = 1}^n a_{ij}(x) \xi_i \xi_j \leq \Lambda \abs{\xi}^2
			\end{equation*}
			\noindent holds for any $x \in \Omega$ and $\xi \in \mathbb{R}^n$. Now define a bilinear form $\langle \cdot,\cdot \rangle_a : H^1_0(\Omega) \times H^1_0(\Omega) \to \mathbb{R}$ by
			\begin{equation}
				\label{eq:inner_prod}
				\langle u,v \rangle_a := \sum_{i,j = 1}^n \int_\Omega a_{ij} \frac{\partial u}{\partial x^i}\frac{\partial v}{\partial x^j}
			\end{equation}
			Then it is easy to see, that $\langle \cdot,\cdot\rangle_a$ is symmetric. Also, $\langle \cdot,\cdot \rangle_a$ is positive definite since
			\begin{equation*}
				\langle u,u\rangle_a = \sum_{i,j = 1}^n \int_\Omega a_{ij} \frac{\partial u}{\partial x^i}\frac{\partial u}{\partial x^j} \geq \lambda \int_\Omega\abs{\nabla u}^2 \geq C \lambda \int_\Omega \abs{u}^2
			\end{equation*}
			\noindent using ellipticity and Poincar\'e's inequality. Moreover by Poincar\'e's inequality we have that
			\begin{equation*}
				C \lambda \norm{u}^2_{H^1_0(\Omega)} \leq \norm{u}_a \leq \Lambda \norm{u}_{H^1_0(\Omega)}^2
			\end{equation*}
			\noindent for the induced norm $\norm{\cdot}_a$. Hence the induced norm is equivalent to the standard norm on $H^1_0(\Omega)$ and thus $(H^1_0(\Omega),\norm{\cdot}_a)$ is a Hilbert space. Thus an application of Riesz representation theorem \ref{thm:RRT} yields the existence of a unique $u \in H^1_0(\Omega)$, such that
			\begin{equation*}
				\langle u,\varphi \rangle_a = l(\varphi) := \int_\Omega f \varphi
			\end{equation*}
			\noindent holds for all $\varphi \in H^1_0(\Omega)$, since $l \in (H^1_0(\Omega))^*$ This proves part (a).
		\item \textit{$H^1$-Estimate.} The main idea in proving part (b) is an induction on $k \in \omega$. So let us assume that $k = 0$. Let $u \in H^1_0(\Omega)$ denote the unique solution of part (a).
			\begin{lemma}
				\label{lem:H^1_estimate}
				\begin{equation*}
					\norm{\nabla u}_{L^2(\Omega')}
				\end{equation*}
			\end{lemma}
	\end{enumerate}
\end{proof}

\printbibliography
\end{document}
