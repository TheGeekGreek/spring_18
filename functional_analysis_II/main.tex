\input{header.tex}

\setcounter{section}{1}

\title{Functional Analysis II Summary}
\author{Yannis B\"{a}hni}
\address[Yannis B\"{a}hni]{University of Zurich, R\"{a}mistrasse 71, 8006 Zurich}
\email[Yannis B\"{a}hni]{\href{mailto:yannis.baehni@uzh.ch}{\nolinkurl{yannis.baehni@uzh.ch}}}

\begin{document}

\begin{abstract}

\end{abstract}

\maketitle

\tableofcontents

\section*{Elliptic Operators in Divergence Form}

\begin{lemma}[Poincar\'e Inequality]
	\label{lem:PI}
\end{lemma}

\begin{theorem}[Riesz Representation Theorem]
	\label{thm:RRT}

\end{theorem}

\begin{theorem}
	Let $\Omega \subseteq\subseteq \mathbb{R}^n$, $k \in \omega$ and consider the elliptic operator
	\begin{equation*}
		L := \sum_{i,j = 1}^n \frac{\partial}{\partial x^i}\del[3]{a_{ij}\frac{\partial}{\partial x^j}},
	\end{equation*}
	\noindent for $a_{ij} \in C^{k + 1}(\wbar{\Omega})$ symmetric. Then:
	\begin{enumerate}[label = \textup{(}\alph*\textup{)},wide=0pt]
		\item Given $f \in L^2(\Omega)$, the homogenous Dirichlet problem
			\begin{align}
				\label{eq:HDP}
				\ccases{
					-L(u) = f & \text{in } \Omega,\\
					u = 0 & \text{on } \partial\Omega
				}
			\end{align}
			\noindent admits a unique weak solution $u \in H^1_0(\Omega)$.
		\item If $f \in H^k(\Omega)$ for some $k \in \omega$, then we have $u \in H^{k + 2}_{\mathrm{loc}}(\Omega)$ for the unique weak solution of part (a) and moreover, for any $\Omega' \subseteq \subseteq \Omega$ we have the estimate
			\begin{equation*}
				\norm{u}_{H^{k + 2}(\Omega')} \leq C\del[1]{\norm{f}_{H^k(\Omega)} + \norm{u}_{H^1(\Omega)}}.
			\end{equation*}
	\end{enumerate}
\end{theorem}

\begin{proof}
	~
	\begin{enumerate}[label = \textit{Step \arabic*:},wide=0pt]
		\item \textit{Derivation of Weak Formulation.} Suppose $u \in C^2(\wbar{\Omega})$ is a solution of (\ref{eq:HDP}). Let $\varphi \in C^\infty_c(\Omega)$. Then integration by parts (see \cite[436]{lee:smooth_manifolds:2013}) yields
			\begin{equation*}
				-\int_\Omega L(u) \varphi = - \sum_{j = 1}^n \int_\Omega \mathrm{div}(X_j)\varphi = \sum_{i,j = 1}^n \int_\Omega a_{ij} \frac{\partial u}{\partial x^j}\frac{\partial \varphi}{\partial x^i} = \sum_{i,j = 1}^n \int_\Omega a_{ij} \frac{\partial u}{\partial x^i}\frac{\partial \varphi}{\partial x^j},
			\end{equation*}
			\noindent where $X_j := \del[2]{a_{ij}\frac{\partial}{\partial x^j}}_i$. Thus we get the weak formulation:
			\begin{equation}
				\label{eq:HDPweak}
				\sum_{i,j = 1}^n \int_\Omega a_{ij} \frac{\partial u}{\partial x^i}\frac{\partial \varphi}{\partial x^j} = \int_\Omega f \varphi \qquad \forall \varphi \in C^\infty_c(\Omega).
			\end{equation}
		\item \textit{Existence and Uniqueness of Weak Solutions.} Since $L$ is uniformly elliptic, there exists $\lambda > 0$ such that 
			\begin{equation*}
				\sum_{i,j = 1}^n a_{ij}(x) \xi_i \xi_j \geq \lambda \abs{\xi}^2
			\end{equation*}
			\noindent holds for any $x \in \Omega$ and $\xi \in \mathbb{R}^n$. Moreover, since $a_{ij} \in C^0(\wbar{\Omega})$, we get that $L$ is uniformly bounded, i.e. there exists $\Lambda > 0$ such that
			\begin{equation*}
				\sum_{i,j = 1}^n a_{ij}(x) \xi_i \xi_j \leq \Lambda \abs{\xi}^2
			\end{equation*}
			\noindent holds for any $x \in \Omega$ and $\xi \in \mathbb{R}^n$. Now define a bilinear form $\langle \cdot,\cdot \rangle_a : H^1_0(\Omega) \times H^1_0(\Omega) \to \mathbb{R}$ by
			\begin{equation}
				\label{eq:inner_prod}
				\langle u,v \rangle_a := \sum_{i,j = 1}^n \int_\Omega a_{ij} \frac{\partial u}{\partial x^i}\frac{\partial v}{\partial x^j}
			\end{equation}
			Then it is easy to see, that $\langle \cdot,\cdot\rangle_a$ is symmetric. Also, $\langle \cdot,\cdot \rangle_a$ is positive definite since
			\begin{equation*}
				\langle u,u\rangle_a = \sum_{i,j = 1}^n \int_\Omega a_{ij} \frac{\partial u}{\partial x^i}\frac{\partial u}{\partial x^j} \geq \lambda \int_\Omega\abs{\nabla u}^2 \geq C \lambda \int_\Omega \abs{u}^2
			\end{equation*}
			\noindent using ellipticity and Poincar\'e's inequality. Moreover by Poincar\'e's inequality we have that
			\begin{equation*}
				C \lambda \norm{u}^2_{H^1_0(\Omega)} \leq \norm{u}_a \leq \Lambda \norm{u}_{H^1_0(\Omega)}^2
			\end{equation*}
			\noindent for the induced norm $\norm{\cdot}_a$. Hence the induced norm is equivalent to the standard norm on $H^1_0(\Omega)$ and thus $(H^1_0(\Omega),\norm{\cdot}_a)$ is a Hilbert space. Thus an application of Riesz representation theorem \ref{thm:RRT} yields the existence of a unique $u \in H^1_0(\Omega)$, such that
			\begin{equation*}
				\langle u,\varphi \rangle_a = l(\varphi) := \int_\Omega f \varphi
			\end{equation*}
			\noindent holds for all $\varphi \in H^1_0(\Omega)$, since $l \in (H^1_0(\Omega))^*$ This proves part (a).
		\item \textit{$H^1$-Estimate.} The main idea in proving part (b) is an induction on $k \in \omega$. So let us assume that $k = 0$. Let $u \in H^1_0(\Omega)$ denote the unique solution of part (a).
			\begin{lemma}
				\label{lem:H^1_estimate}

			\end{lemma}
	\end{enumerate}
\end{proof}

\printbibliography
\end{document}
